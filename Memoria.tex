\documentclass[a4paper,12pt]{book}

\usepackage[utf8]{inputenc}
\usepackage[spanish]{babel}
%\usepackage{listings}
\usepackage{graphicx}
\usepackage{hyperref}

\usepackage{color}
\definecolor{gray97}{gray}{.97}
\definecolor{gray75}{gray}{.75}
\definecolor{gray45}{gray}{.45}

\usepackage{listings}
\lstset{ frame=Ltb,
framerule=0pt,
aboveskip=0.5cm,
framextopmargin=3pt,
framexbottommargin=3pt,
framexleftmargin=0.4cm,
framesep=0pt,
rulesep=.4pt,
backgroundcolor=\color{gray97},
rulesepcolor=\color{black},
%
stringstyle=\ttfamily,
showstringspaces = false,
basicstyle=\small\ttfamily,
commentstyle=\color{gray45},
keywordstyle=\bfseries,
%
numbers=left,
numbersep=15pt,
numberstyle=\tiny,
numberfirstline = false,
breaklines=true,
tabsize=2,
}

% minimizar fragmentado de listados
\lstnewenvironment{listing}[1][]
{\lstset{#1}\pagebreak[0]}{\pagebreak[0]}

% estilo para poner comandos de consola
\lstdefinestyle{consola}
{basicstyle=\scriptsize\bf\ttfamily,
backgroundcolor=\color{gray75},
}

% estilo para código Java
\lstdefinestyle{Java}
{language=Java,
}

% estilo para código XML
\lstdefinestyle{XML}
{language=XML,
}

% estilo para código HTML
\lstdefinestyle{HTML}
{language=HTML,
}



\title{\Huge Plataforma de firma digital para la Universidad de Málaga.}
	
 	\author{Juan Antonio Pérez Ariza \\
		Escuela Técnica Superior Ingeniería Informática \\
		Universidad de Málaga \\
		\\
		Profesor: \\
		Isaac Agudo Ruíz\\
		Departamento Lenguajes y Ciencias de la Comunicación \\
		Universidad de Málaga}

\begin{document}
	\maketitle	
	
	%Table of contents	
	\tableofcontents
	
	%separacion entre parrafos
	\parskip=5mm
	
	\mainmatter
	%introducción
	\chapter{Introducción.}
En este primer capítulo de la memoria vamos a explicar las motivaciones que nos llevaron a pensar en realizar dicho proyecto, los objetivos que nos marcamos cuando lo diseñamos, los materiales usados y la organización de esta memoria.

\section{¿Por qué decidimos hacer el proyecto?}
Al igual que muchos estudiantes de la Universidad de Málaga, yo suelo comer habitualmente en la cafetería de la facultad y hay mucha personas que se molesta cuando le piden que firme el papel con el que se lleva el recuento de los estudiantes que comen en las cafeterías para el descuento por ser estudiante. Además de este inconveniente hay un par de problemas más, que son lo molesto que es tener que firmar todos los días o habitualmente y las colas que se forman al tener que rellenar el nombre y la firma, por eso se decidió hacer una aplicación para terminales móviles con la que agilizar todo este proceso de firma y control de estudiantes, mediante la lectura de un código QR que tendría la información necesaria para la realización de la firma.

A medida que avanzaba el proyecto se vio que se podía ampliar la funcionalidad de la aplicación, no solo al comedor, si no también a alquiler de pistas o cualquier documento necesario en la Universidad de Málaga.

Otra razón para la realización de este proyecto es que a mi me gusta la seguridad informática y vi en este proyecto una buena forma de aprender más sobre criptografía, particularmente en la criptografía de clave pública. También vi una buena forma de aprender a programar para terminales Android, debido al gran auge que tienen en este momento los terminales móviles, y a crear aplicaciones web de las que no tenía ninguna idea. Al inicio la aplicación web se pensó en hacer directamente en java sin ninguna ayuda, pero se descartó ante la dificultad de encontrar un servicio de hosting gratuito, por lo que se decidió cambiar a una sugerencia que hizo director de proyecto de usar una plataforma que proporciona Google llamada Google App Engine, que es gratuita y se pueden crear aplicaciones web programadas en el lenguaje de programación Java y así tener la posibilidad de aprender otras APIS, no solo Java2EE.

\section{Objetivos que queríamos conseguir.}

El principal objetivo que queríamos conseguir era que la forma de firmar fuera muy fácil y que no fuera un mecanismo muy engorroso. Para ellos decidimos realizar una aplicación para smartphone Android y una aplicación web para el almacenamiento y posterior comprobación de las firmas.

Una vez teníamos clara la idea, empezamos a diseñar un sistema con el que se pudiera realizar la firma digital inicialmente, solo de un recibo y finalmente cualquier documento de la UMA agilizando de esta forma dicho proceso.

%TODO: Hay poner algo de criptografía... xDDD

En la parte de la aplicación de Android se decidió hacer una aplicación clara y que fuese fácil de usar. Para eso usamos la API nivel 14 que equivale a la versión 4.0 de Android, llamada Ice Cream Sandwich. Se eligió porque proporciona una nueva forma de diseño de las interfaces, un nuevo tema  llamado Holo y proporciona muchas nuevas herramientas como por ejemplo son los ActionBar, que es una barra que permanece siempre en la parte superior de la pantalla, la que se va acomodando a las necesidades que tiene la aplicación cambiando los botones con diferentes funcionalidades, por ejemplo si estamos en la pantalla principal pues tendremos siempre visible el botón de añadir un nuevos recibo que abrirá el lector de códigos QR, se puede observar en la primera barra de la figura~\ref{fig:actionBar}, sin embargo si estamos visualizando un recibo solo tendremos el botón de volver atrás, como se puede ver en la segunda barra de la figura~\ref{fig:actionBar}, en la que se puede ver que al lado del icono de la aplicación una flechita que indica que es el botón para volver atrás.

\begin{figure}
  \centering
    \includegraphics[scale=0.3]{./Introduccion/imagenes/actionbar.png}
  \caption{Action Bar.}
  \label{fig:actionBar}
\end{figure}

Al tomar la decisión de programar para terminales con Android 4.0 o mayor estuvimos sopesando los pros y los contras, y al final decidimos que la implantación de Android 4.0 cada vez es mayor y que cada día hay más terminales con dicha versión, como se puede ver en este gráfico de la figura~\ref{fig:graficoEvolucionAndroid} y podemos observar que a finales de agosto de este año la cantidad de usuarios afectados sería de más del 15\% de terminales como podemos ver en la figura~\ref{fig:graficoUsoAndroid}, aunque todavía sigue reinando la versión 2.3.3, aunque creemos que el cambio a la versión 4.0 o superior será rápida debido a todas las ventajas que aporta y mucho más ahora que hace unos meses Google sacó una nueva versión, la 4.1, llamada Jelly Bean y casi todas las compañías querrán actualizar sus terminales a la última versión, por lo que a pesar de dejar a un gran número de usuarios sin poder usar la aplicación preferimos funcionalidad y elegancia frente a gran cantidad de usuarios, ya que estos llegarán a medida que sus compañías actualicen sus terminales.


\begin{figure}
  \centering
    \includegraphics[scale=0.8]{./Introduccion/imagenes/graficoEvolucionAndroid.png}
  \caption{Gráfico de las versiones de Android a principio de octubre del 2012.}
  \label{fig:graficoEvolucionAndroid}
\end{figure}

\begin{figure}
  \centering
    \includegraphics[scale=0.5]{./Introduccion/imagenes/graficoUsoAndroid.png}
  \caption{Gráfico del uso de las versiones de Android a principios de octubre del 2012.}
  \label{fig:graficoUsoAndroid}
\end{figure}

En la parte del servidor al elegir la plataforma de Google, hubo muchas cosas que resultaron más fáciles a costa de tener que aprender a usar el SDK que Google proporciona, que como era una de las cosas por la que elegimos dicha plataforma no nos importó. Una de las cosas que nos facilitaba es la gestión de usuarios, que los gestiona Google directamente al tener que usar una cuenta de Google Account para usar la aplicación web. Toda la seguridad y mantenimiento de los servidores, copias de seguridad, balanceos de carga y otra gran cantidad de funciones también las realiza Google, por lo que nos eximen de su realización y no tendremos que preocuparnos de ellas.

\section{Organización de la memoria.}

En el primer capítulo, que es el actual, hemos realizado una introducción al proyecto, explicado la idea inicial y los materiales usados. Para continuar en el capítulo segundo de la memoria explicaremos los conocimientos básicos sobre todas las tecnologías usadas en el proyecto, como pueden ser Android, Java, criptografía, HTML, GIT, etc. En el capítulo tercero de la memoria explicaremos todo el diseño e implementación realizados en el proyecto. En el cuarto capítulo comentaremos todos los conocimientos necesarios para entender la criptografía usada en el proyecto, así como una breve historia de la criptografía, desde la época romana hasta la actual. En el quinto capítulo vamos a documentar todos los conocimientos necesarios y que hemos tenido que usar para realizar la aplicación de Android. En el capítulo sexto contaremos los conocimientos básicos y los necesarios para realizar las aplicaciones web del proyecto. En el último capítulo tendremos el desenlace del proyecto en el que explicaremos el uso de las aplicaciones, expondremos las conclusiones, problemas y posibles trabajos futuros que hemos observado que se podrían realizar y la biografía usada. Para finalizar tendremos tres anexos en los que explicaremos la configuración de la plataforma Eclipse para el uso de la API Google App Engine como la de Android, la creación de los certificados de clave pública y el contenido del CD.  

\section{Material usado.}
Para la realización del proyecto hemos usado un ordenador personal para todo el proceso de programación y un smartphone con Android para la depuración y prueba de la aplicación. 

El ordenador es un ordenador portátil, con Ubuntu 12.04 como sistema operativo, un procesador Pentium Dual Core a 2.2Ghz, 4 Gb de memoria ram.

El móvil usado es un Samsung Galaxy Nexus (figura~\ref{fig:nexus}) que fue el móvil presentado cuando Google lanzó la versión 4.0, por lo que es el primer terminal en usar Android 4.0 y meses después el primero en recibir la actualización de Android 4.1. Sus característica son una pantalla de 4.65 pulgadas Super Amoled con resolución de 1280 x 720, procesador dual-core a 1.2Ghz, HSPA+, NFC, Wifi, GPS, etc.

\begin{figure}
  \centering
    \includegraphics[scale=0.3]{./Introduccion/imagenes/nexus.png}
  \caption{Samsung Galaxy Nexus.}
  \label{fig:nexus}
\end{figure}

No tenemos datos sobre las máquinas usadas por los servidores donde Google da el servicio de Google App Engine.


	%conocimientos previos sobre las tecnologías
	\chapter[Conocimientos previos]{Conocimientos previos sobre las tecnologías usadas.}\label{cap:conocimientos}
\markboth{CAPÍTULO \ref{cap:conocimientos}. CONOCIMIENTOS PREVIOS.}{}

En este segundo capítulo vamos a explicar cuáles son las ideas previas de las que partimos y los conocimientos que poseíamos sobre las tecnologías que hemos usado en el proyecto. Haciendo una breve explicación sobre su funcionamiento, su uso, su historia y sus recientes versiones.

El principal elemento usado en el proyecto es el lenguaje de programación Java, que lo hemos usado para programar tanto la aplicación web como la aplicación en el móvil. Pero aparte de los diferentes SDK de Android y de Google App Engine, hemos recurrido y empleado muchas otras tecnologías y lenguajes como pueden ser SQL, XML, UML, GIT. Además hemos aplicado los conocimientos básicos sobre criptografía de clave pública necesarios para realizar todo el proceso de firma digital.   

\section{El lenguaje de programación Java}

Java es un lenguaje de programación orientado a objetos que fue diseñado por James Gosling\footnote{ Para más información sobre James Gosling: \url{http://en.wikipedia.org/wiki/James\_Gosling}} para Sun Microsystems y que recientemente ha sido comprado por Oracle Corporation. Fue lanzado en 1995 y ha sido el centro de toda la plataforma Java de Sun Microsystems. Es un lenguaje con una sintaxis muy parecida a C o C++, pero con la gran ventaja de que el manejo de punteros y objetos es automático, al igual que la recogida de basura.

Java es un lenguaje en el que hay que compilar los códigos fuentes para crear unos archivos intermedios llamados bytecodes, los archivos *.class, que luego serán interpretados por la máquina virtual de Java (JVM). Esta dependerá de la arquitectura en la que se quiera ejecutar la aplicación Java. Gracias a esto se puede decir que Java es un lenguaje multiplataforma, lo que significa que un mismo código Java se puede ejecutar en Linux, en Windows, en Mac o cualquier otro sistema para el cual exista una máquina virtual, lo que en inglés se llama ``write once, run anywhere" (WORA). Además de esta importante ventaja Java es un lenguaje de propósito general, concurrente, basado en clases y orientado a objetos. Java es el segundo lenguaje de programación más popular de 2012, gracias a las aplicaciones web cliente-servidor que tienen tanto auge en estos momentos, como podemos ver en la figura \ref{fig:indicetiobe}.

\begin{figure}[h]
  \centering
    \includegraphics[scale=0.9]{./ConocimientosPrevios/imagenes/indiceTiobe.png}
  \caption{Índice Tiobe en septiembre del 2012. \url{http://www.tiobe.com/}}
  \label{fig:indicetiobe}
\end{figure} 

La implementación original y las referencias del compilador de Java, máquinas virtuales y las librerías de clases fueron desarrolladas por Sun en 1995, pero en el 2007 gracias a la contribución de la comunidad, Sun Microsystems cambió la licencia de todas las tecnologías Java a GNU General Public License, por lo que se abría la posibilidad de que se crearan versiones alternativas de compiladores bajo licencia GNU como por ejemplo GNU Compiler para Java o GNU Classpath.

En el proyecto la versión usada fue la versión \textbf{Java SE}.

\subsection{Historia}

Originalmente Java nació como un proyecto de James Gosling, Mike Sheridan y Patrick Naughton en 1991 y estaba diseñado para una televisión interactiva, pero era muy avanzado para lo que la industria televisiva de la época podía necesitar. En su origen fue llamado Oak, pero por problemas con el nombre, ya que era una marca registrada de otra empresa, lo cambiaron a Green y posteriormente ya lo renombraron al definitivo Java. Hay muchas teorías sobre por qué se llama Java, pero una de ellas es que había una cafetería llamada Java Coffe donde James, Mike and Patrick pasaron muchas horas consumiendo café.

La idea de Gosling era crear una máquina virtual donde funcionara un lenguaje de programación con la sintaxis y la estructura de C/C++ para que la curva de aprendizaje fuera muy suave para los programadores que en la época sabían C/C++.

Sun Microsystems lanzó Java 1.0 en 1995, con la principal característica de que una vez escrito un código fuente no había que modificarlo para que funcionara en las diferentes máquinas, lo que anteriormente hemos llamado con el acrónimo en inglés WORA (Write Once, Run Anywhere). Rápidamente todos los navegadores de la época empezaron a soportar applets Java en las páginas web, por lo que Java se volvió muy popular en la época. La nueva versión Java 2 fue lanzada en 1998-1999 y con ella llegaron las distinciones en diferentes plataformas, como por ejemplo Java2EE para aplicaciones corporativas o una versión ligera llamada Java2ME que estaba diseñada para funcionar en los diferentes teléfonos de la época, y el resto se agrupan en la versión Java2SE, que es la versión estándar.

En 1997, Sun Microsystems intentó formalizar Java mediante una norma ISO/IEC pero se retiró del proceso y dio todo el control a la comunidad. Sun ofrecia implementaciones gratuitas y generaba dinero vendiendo algunas licencias de productos como Java Enterprise System. Una cosa importante es que Sun distingue entre el SDK (Kit de desarrollo) y el JRE (Entorno de ejecución) en el que van incluidos los compiladores, debuggers, etc.

El 13 de noviembre del 2006, Sun lanzó Java gratis y como software libre, bajo la licencia GNU General Public License (GPL). El proceso finalizó el 8 de mayo del 2007.

En 2009-2010 Oracle Corporation compró Sun Microsystems por lo que Java actualmente pertenece a Oracle Corporation.

\subsection{Versiones}

\begin{itemize}

	\item \textbf{JDK 1.0} (23 de enero de 1996): Primer lanzamiento
	
	\item \textbf{JDK 1.1} (19 de febrero de 1997): Las primeras características añadidas fueron una reestructuración intensiva del modelo de eventos AWT (Abstract Windowing Toolkit), clases internas (inner classes), JavaBeans, JDBC (Java Database Connectivity), para la integración de bases de datos y RMI (Remote Method Invocation).
	
     \item \textbf{JDK 1.2}(8 de diciembre de 1998): Recibió el nombre en clave Playground. Esta y las siguientes versiones fueron recogidas bajo la denominación Java 2 y el nombre ``J2SE" (Java 2 Platform, Standard Edition), reemplazó a JDK para distinguir la plataforma base de J2EE (Java 2 Platform, Enterprise Edition) y J2ME (Java 2 Platform, Micro Edition). 
    Se añadieron las siguientes mejoras, la palabra reservada strictfp, reflexión en la programación, la API gráfica (Swing) fue integrada en las clases básicas, la máquina virtual (JVM) de Sun fue equipada con un compilador JIT (Just in Time) por primera vez, Java Plug-in, Java IDL, una implementación de IDL (Lenguaje de Descripción de Interfaz) para la interoperabilidad con CORBA y Colecciones.

    \item \textbf{J2SE 1.3} (8 de mayo de 2000): Recibió el nombre en clave Kestrel. Los cambios más notables fueron: la inclusión de la máquina virtual HotSpot JVM, RMI fue cambiado para que se basara en CORBA, JavaSound, se incluyó el Java Naming and Directory Interface (JNDI) en el paquete de bibliotecas principales (anteriormente disponible como una extensión), Java Platform Debugger Architecture (JPDA).

    \item \textbf{J2SE 1.4} (6 de febrero de 2002): Recibió el nombre en clave Merlin. Este fue el primer lanzamiento de la plataforma Java desarrollado bajo el Proceso de la Comunidad Java como JSR 59. Las principales características que se le añadieron fueron palabra reservada assert, expresiones regulares modeladas al estilo de las expresiones regulares Perl, encadenación de excepciones, non-blocking NIO (New Input/Output), logging API, API I/O para la lectura y escritura de imágenes en formatos como JPEG o PNG, parser XML integrado y procesador XSLT (JAXP), seguridad integrada y extensiones criptográficas (JCE, JSSE, JAAS), Java Web Start incluido.
    
    \item \textbf{J2SE 5.0} (30 de septiembre de 2004): Recibió el nombre en clave Tiger. Estos fueron los cambios más importantes, plantillas (genéricos), metadatos, también llamados anotaciones, permite a estructuras del lenguaje, como las clases o los métodos, ser etiquetados con datos adicionales que pueden ser procesados posteriormente por utilidades de proceso de metadatos, autoboxing/unboxing, conversiones automáticas entre tipos primitivos (Como los int) y clases de envoltura primitivas (Como Integer), enumeraciones, varargs (número de argumentos variable), el último parámetro de un método puede ser declarado con el nombre del tipo seguido por tres puntos (por ejemplo \lstinline{void drawtext(String... lines)}). En la llamada al método, puede usarse cualquier número de parámetros de ese tipo, que serán almacenados en un array para pasarlos al método, bucle for mejorado, la sintaxis para el bucle for se ha extendido con una sintaxis especial para iterar sobre cada miembro de un array o sobre cualquier clase que implemente Iterable, como la clase estándar Collection, de la siguiente forma:

\begin{lstlisting}[style=Java]
void displayWidgets (Iterable<Widget> widgets) {
	for (Widget w : widgets) {
		w.display();
	}
}
\end{lstlisting}

    \item \textbf{Java SE 6} (11 de diciembre de 2006): Recibió el nombre en clave Mustang. En esta versión, Sun cambió el nombre ``J2SE" por Java SE y eliminó el ``.0" del número de versión. Los cambios más importantes introducidos en esta versión son un nuevo marco de trabajo y APIS que hacían posible la combinación de Java con lenguajes dinámicos como PHP, Python, Ruby y JavaScript, el motor Rhino, de Mozilla, una implementación de Javascript en Java, un cliente completo de Servicios Web y soporta las últimas especificaciones para Servicios Web, mejoras en la interfaz gráfica y en el rendimiento.
    
    \item \textbf{Java SE 7} (Julio 2011): Su nombre en clave es Dolphin. Y las principales nuevas características son: soporte para XML dentro del propio lenguaje, un nuevo concepto de superpaquete, soporte para closures, e introducción de anotaciones estándar para detectar fallos en el software.

\end{itemize}

\section{El entorno de programación Eclipse.}

Eclipse es un entorno integral de desarrollo que consta de un entorno de desarrollo integrado (IDE) y es extensible mediante plugins que están escritos en Java. Puede ser usado para una larga lista de lenguajes de programación como pueden ser C, C++, Haskell, Perl, PHP, Python, Android y un largo etcétera. Fue originalmente desarrollado por IBM y fue lanzado con la licencia de software Eclipse Public License\footnote{ Para más información visite: \url{http://en.wikipedia.org/wiki/Eclipse\_Public\_License}} la cual es una licencia de software libre. El SDK de Eclipse es libre y tiene licencia Open Source por lo que cualquier persona con los conocimientos necesarios puede programar el plugin que necesite para Eclipse. Fue el primer entorno de programación que funcionó bajo GNU Classpath y que funcionaba sin problemas con IcedTea. En la figura \ref{fig:pantallaEclipse} se puede ver el aspecto que tiene.

\begin{figure}
  \centering
    \includegraphics[scale=0.5]{./ConocimientosPrevios/imagenes/pantallaEclipse.png}
  \caption{Eclipse 4.2 Juno.}
  \label{fig:pantallaEclipse}
\end{figure} 

En el proyecto hemos usado la versión \textbf{Indigo}, que equivale a la versión 3.7 de Eclipse.

\subsection{Historia}

Eclipse comenzó como un proyecto de IBM Canadá. En noviembre de 2001 se creó un grupo de empresas para promover el desarrollo de Eclipse como software libre, los miembros iniciales eran Borland, IBM, Merant, QNX Software Systems, Rational Software, Red Hat, SuSE, TogetherSoft and WebGain. Finalmente en enero de 2004 se creó la Eclipse Foundation. 

\subsection{Versiones}

\begin{itemize}

	\item \textbf{Versión 3.0} (21 de junio de 2004)
	
	\item \textbf{Versión 3.1} (28 de junio de 2005)
	
	\item \textbf{Versión 3.2} (30 de junio de 2006): recibió el nombre de Callisto.
	
	\item \textbf{Versión 3.3} (29 de junio de 2007): recibió el nombre de Europa.
	
	\item \textbf{Versión 3.4} (25 de junio de 2008): recibió el nombre de Ganymede.
	
	\item \textbf{Versión 3.5} (24 de junio de 2009): recibió el nombre de Galileo.
	
	\item \textbf{Versión 3.6} (23 de junio de 2010): recibió el nombre de Helios.
	
	\item \textbf{Versión 3.7} (22 de junio de 2011): recibió el nombre de Indigo.
	
	\item \textbf{Versión 4.2} (27 de junio de 2012): recibió el nombre de Juno.
	
	\item \textbf{Versión 4.3} (26 de junio de 2013): esta será la próxima versión, que saldrá el próximo año y recibirá el nombre de Kepler.

\end{itemize}

\section{Criptografía.}\label{lbl:criptografia}

La criptografía es la ciencia que se encarga del estudio y creación de técnicas para la protección de una comunicación, para que solamente los usuarios autorizados puedan verla, leerla y entenderla. En la actualidad la criptografía es un término que se usa de forma similar a encriptación, que es el proceso para transformar una información mediante diferentes algoritmos, en un mensaje que no pueda entender un atacante que intercepte una comunicación. 

En el proyecto hemos usado una criptografía llamada \textbf{Criptografía de Clave Pública}, que como veremos a continuación en la historia brevemente y posteriormente en el capítulo~\ref{cap:criptografia}, en el que se explicará la criptografía usada a lo largo del proyecto con mas profundidad, consta de dos claves, ambas enlazadas matemáticamente y si conocemos una no podremos de ninguna forma conseguir la otra. La pública es la que tendría la persona que quiera desencriptar el mensaje, que a su vez da nombre a este algoritmo y otra privada que sólo conocerá la persona que quiere encriptar el mensaje.

La criptografía ha evolucionado mucho y actualmente no solo se usan para proteger mensajes, si no que también se usa para proteger la integridad de ellos. Este es uno de los usos más común de la criptografía de clave pública.

\subsection{Historia}

Podemos hacer dos grandes grupos dentro de la historia de la criptografía, la criptografía clásica y la criptografía durante la época de los ordenadores.

\subsection{Criptografía clásica.}

	Durante la época de la criptografía clásica sólo se quería proteger el mensaje que se enviaba de la mirada de curiosos y enemigos por lo que únicamente existían algoritmos de encriptación, la integridad del mensaje no importaba en esa época.  
	
	En dicha época todos los algoritmos de cifrados que existían eran por transposición o sustitución de caracteres. A continuación exponer unos ejemplos de los algoritmos utilizados más famosos. 
\begin{itemize}

	\item \textbf{Cifrado Cesar:} dicho cifrado es famoso porque los usaban las centurias romanas para comunicarse entre ellas de manera que si un mensaje era interceptado no pudiera ser leído. Consiste en sustituir cada carácter del mensaje por el que hay tres lugares a la derecha. Por ejemplo si tenemos el mensaje ``Hola" si lo ciframos con este sistema conseguimos ``Krod", en la figura \ref{fig:cifradoCesar} podemos ver como es el cifrado.
	
	Para desencriptar solo habría que intercambiar por la tercera letra anterior.

\begin{figure}[h]
  \centering
    \includegraphics[scale=0.6]{./ConocimientosPrevios/imagenes/cifradoCesar.png}
  \caption{Ejemplo Cifrado Cesar}
  \label{fig:cifradoCesar}
\end{figure} 

	\item \textbf{Cifrado Homofónico:} Es una evolución del siguiente, pero en vez de sustituir siempre por el mismo carácter lo que se hace es tener la posibilidad de poder realizar varios cambios posibles, por lo que un mismo mensaje podría generar varios textos cifrados, complicando así su desencriptación. En la figura \ref{fig:cifradoHomofonico} podemos ver una tabla sencilla de sustitución para realizar el cifrado. Por ejemplo si ciframos la palabra ``PLATON" nos daría de resultado ``882110772963", pero podríamos sustituir la P no solo por 88 si no por cualquier valor de la tabla dando lugar a que pudiéramos crear varios mensajes cifrados.
	
\begin{figure}[h]
  \centering
    \includegraphics[scale=0.7]{./ConocimientosPrevios/imagenes/cifradoHomofonico.png}
  \caption{Tabla para cifrado homofónico}
  \label{fig:cifradoHomofonico}
\end{figure}

	\item \textbf{Cifrado por Transposición:} consiste en realizar una permutación de las posiciones que ocupan las letras escritas, un ejemplo podría ser escribir todo el texto con una cierta longitud preestablecida y luego leerlo por columnas en vez de por filas. En la figura \ref{fig:cifradoTransposicion} podemos ver un ejemplo del mecanismo de cifrado.  

\begin{figure}[h]
  \centering
    \includegraphics[scale=0.4]{./ConocimientosPrevios/imagenes/cifradoTransposicion.png}
  \caption{Ejemplo de cifrado por sustitución}
  \label{fig:cifradoTransposicion}
\end{figure}	

	\item \textbf{Cifrado Producto:} Es un cifrado que combina sustitución y transposición y se puede considerar como un encadenamiento de varios cifrados. Esto da lugar a cifrados complejos, seguros y difíciles de atacar, ya que tendríamos que averiguar no sólo el método de cifrado utilizado, sino que también tendríamos que saber el orden en el que se ha ejecutado las encriptaciones.

	\item \textbf{Cifrado Vernam:} es un tipo de cifrado que se denomina cifrado de flujo. El texto en claro se combina con una cadena, del mismo tamaño del texto en claro, de número aleatorios o pseudoaleatorio por medio de la función XOR. Lo inventó Gilbert Vernam que era un ingeniero de AT\&T en 1917. Es también conocido como RC4 en internet.  
	
\end{itemize}

\subsection{Criptografía durante la época de los ordenadores.}

La criptografía dio un gran salto en cuanto a calidad en el momento en el que se empezaron a usar ordenadores para encriptar y desencriptar textos, debido a que los ordenadores son máquinas que las tareas repetitivas las hacen muy bien y muy rápidos.

Se empezaron a idear nuevos algoritmos de cifrado mucho más complejos, los cuales se pueden dividir en dos grandes grupos, la criptografía de clave simétrica y la criptografía de clave pública. A continuación vamos a explicar brevemente los algoritmos más famosos de ambos.

\subsubsection*{Criptografía de Clave simétrica}

La principal característica de esta técnica de criptografía es que usan la misma clave para encriptar y desencriptar.

\begin{itemize}

	\item \textbf{Data Encryption Standard (DES):} Fue presentado por IBM en 1974, para generar un estandar de cifrado para transmisión de datos y cifrado de almacenamiento de datos y que fuera usado por gobiernos, empresas privadas o cualquier usuario. IBM comenzó el desarrollo basándose en un dispositivo de cifrado llamado Lucifer el cual tenia una clave de 128 bits. DES es un criptosistema de clave secreta que cifra en bloques de 64 bits del texto en claro y genera otros bloques de 64 bits del texto cifrado. La clave utilizada también es de 64 bits, pero el bit final de cada octeto de los 64 bits de la clave se usa como bit de paridad para control de errores. El cifrado se realiza en 16 iteraciones en las que se usan varias operaciones como son operaciones XOR, permutaciones y sustituciones. El esquema para cifrar se puede ver en la figura \ref{fig:cifradoDes}.
	
\begin{figure}
  \centering
    \includegraphics{./ConocimientosPrevios/imagenes/cifradoDes.png}
  \caption{Iteraciones en el cifrado DES}
  \label{fig:cifradoDes}
\end{figure}

\item \textbf{AES (Rijndael):} Fue presentado al concurso AES el 2 de enero de 1997 y anunciado ganador en 2001. Fue diseñado por dos criptólogos llamados Joan Daemen y Vincent Rijmen, ambos estudiantes de la Katholieke Universiteit Leuven de Bélgica. Al contrario que DES, AES es una red de sustituciones y permutaciones no una red de Feistel, se transformó en estándar efectivo el 26 de mayo de 2002 y en la actualidad es uno de los algoritmos de encriptación más famosos. Opera con bloques de 128 bits y tiene claves de 128, 192 y 256 bits.

\end{itemize}

El mayor problema que tiene este tipo de criptografía es que para que el destinatario pueda leer el mensaje necesita saber la clave y el intercambio de clave puede ser una dificultad muy grande si ambos usuarios no se pueden comunicar directamente, ya que usando cualquier otro método la clave podría ser interceptada y todo el proceso de encriptación sería inutil.

\subsubsection*{Criptografía de Clave Pública}

La criptografía de clave pública fue inventada por Diffie y Hellman y paralelamente por Merkle y ambos grupos aportaron a la criptografía el concepto de la utilización de pares de claves.

La característica principal es que cada usuario posee dos claves, una privada que sólo conoce el dueño de la clave y será usada para descifrar todo lo que otros usuarios cifren con su otra clave y otra la clave pública que es conocida por el resto de usuarios y será la que estos usarán para encriptar el mensaje que queremos que sea secreto. Así de esta forma si una persona quiere comunicarse con otra de forma secreta sólo tiene que conocer su clave pública, cifrar con ella y el destinatario podrá descifrar el mensaje con su clave privada. 

Otra característica es que las claves son imposibles de deducir una a partir de la otra, ambas claves son de una gran longitud y son generadas mediante exponenciación y/o productos de números primos grandes.

En los primeros años de existencia de la criptografía de clave pública se inventaron tres sistemas, Algoritmo de la mochila de Merkle-Hellman que fue roto, el esquema de McEliece que está considerado imposible de llevar a la práctica y un tercero que es el que explicaremos a continuación llamado RSA cuyo uso de ha impuesto actualmente.

\begin{itemize}

	\item \textbf{RSA:} Su nombre proviene de sus creadores que son Rivest, Shamir y Adleman y se basaron en la idea: \textit{``es muy fácil multiplicar dos números enteros primos grandes, pero extremadamente difícil hallar la factorización del producto"}, cuando inventaron el RSA en 1997. Es un algoritmo exponencial. Una característica de RSA es que tanto el mensaje, como el texto cifrado tienen que ser un código decimal, por lo que se tendría que usar el valor ASCII de la letra por ejemplo. Un ejemplo de uso sería el siguiente, lo primero que se debe de hacer antes de enviar el mensaje es acordar el algoritmo que se va a usar, lo siguiente el emisor cifra el mensaje usando la clave pública del receptor y se lo envía. Acto seguido el receptor descifra el mensaje que ha enviado el receptor usando su propia clave privada. La gran ventaja de este método es que en ningún momento la clave privada se tiene que enviar, por lo que solucionamos el gran problema que dijimos que tenían los algoritmos de cifrado simétrico, que antes de nada había que intercambiar la clave con la vulnerabilidad que eso implicaba. 

\end{itemize}

El algoritmo RSA será explicado con más profundidad en el capítulo~\ref{cap:criptografia}.

Los algoritmos de clave pública tienen un gran problema, es que son muy lentos realizando el proceso de cifrado y descifrado, por lo que en la situaciones reales se usan para realizar el intercambio de claves de algoritmos de clave simétrica que son mucho más rápidos y también igual de seguros, de esta forma solucionamos su principal problema.

\section{Android.}

Android es un sistema operativo basado en Linux especialmente diseñado para smartphone, tablet, smart TV y una infinidad de dispositivos, desarrollado por Google con Open Handset Alliance. Android empezó siendo desarrollado por la compañía llamada Android que inicialemente fue financiada y después comprada por Google en 2005. En 2007 cuando se presentó por primera vez Android también se anunció la fundación de Open Handset Alliance que es un conjunto de 86 empresas, entre las que hay compañías de hardware, software y telecomunicaciones, interesadas en el mundo de los dispositivos móviles. Android es código abierto y está distribuido bajo licencia Apache\footnote{ Para saber más sobre la licencia visite: \url{http://en.wikipedia.org/wiki/Apache\_License}}. La tarea del mantenimiento y desarrollo de Android es de Android Open Source Project (AOSP).

Android tiene una gran comunidad de desarrolladores que pueden extender las funcionalidad de los teléfonos o de cualquier dispositivos que pueda ejecutar la máquina virtual de Android, se puede desarrollar tanto en Java usando el SDK o en C++ usando el NDK, posee una tienda online llamada Google Play (anteriormente Android Market), donde se pueden comprar aplicaciones, películas, libros o música y en la que cualquier desarrollador por una pequeña cantidad de dinero (alrededor de 25\euro, por una cuenta vitalicia de desarrollador) puede subir todas las aplicaciones gratuitas o de pago que desee. En Junio de 2012 había alrededor de 600.000 aplicaciones en Google Play.

En el primer cuatrimestre de 2012, Android tenía el 59\% del mercado de smartphones en el mundo, de ahí la importancia de esta plataforma para los desarrolladores, ya que proporciona un mercado muy amplio y una forma muy fácil y barata de conseguir un gran número de usuarios.

Los detalles técnicos de Android se explicarán con más profundidad en el capítulo~\ref{cap:android}.

\subsection{Historia.}

Como hemos dicho anteriormente Android fue diseñado y creado originalmente por una compañía llamada Android que fue fundada en Palo Alto, California en 2003 por Andy Rubin, Rich Miner, Nick Sears y Chris White. Originalmente solo estaba diseñado para funcionar con smartphones, ya que ellos pensaban que un smartphone era algo más que un dispositivo que sirviera para usar el GPS y tener preferencias. 

Google compró Android el 17 de agosto de 2005, con la intención de entrar en el mercado de los teléfonos móviles. Después de varios años de rumores el 5 de noviembre de 2007, Google presentó la Open Handset Alliance, un grupo de empresas que incluian a Broadcom Corporation, Google, HTC, Intel, LG, Marvell Technology Group, Motorola, Nvidia, Qualcomm, Samsung Electronics, Sprint Nextel, T-Mobile and Texas Instruments entre otras muchas empresas que estaban interesadas en generar estándares para dispositivos móviles. Ese mismo día también se lanzó el primer producto Android basado en el kernell de Linux 2.6.

Android ha sido muy criticado por la gran fragmentación que tiene debido al gran número de versiones que posee, que son compatibles hacia versiones abajo pero no hacia versiones posteriores, esto quiere decir que la versión 4.0 es compatible con todo el software que funcionase con las versiones 2.0, 2.3 o cualquiera inferior, pero no será compatibles con el software diseñado para la versión 4.1. Este problema hace necesario que los fabricantes actualicen el software de sus teléfono, lo que es un gran problema debido a que muchos no lo hacen, hubo un pequeño intento de solucionar esta problemática haciendo que los fabricantes estuvieran obligados a actualizar sus terminales al menos en los 18 meses posteriores a la salida al mercado, pero no hubo ningún acuerdo.   

\subsection{Versiones.}

Como curiosidad todas las versiones de Android se denominan con un nombre en clave que es un postre.
\begin{itemize}

	\item \textbf{1.0 (Apple Pie):} primera versión lanzada el 23 de septiembre del 2008.
	
	\item \textbf{1.1 (Banana Bread):} lanzada el 9 de febrero del 2009.
	
	\item \textbf{1.5 (Cupcake):} fue presentada el 30 de abril del 2009, esta fue la primera versión con la que Android empezó a despuntar y entrar en el mundo de los teléfonos móviles, anteriormente apenas si era conocido. Tenía características nuevas muy interesantes como poder grabar y reproducir vídeo, podía subir videos a Youtube e imágenes a Picasa directamente desde el teléfono, un nuevo teclado predictivo, nuevos widget y carpetas para colocar en la pantalla de inicio y transiciones animadas.

	\item \textbf{1.6 (Donut):} fue presentada el 15 de septiembre de 2009. Se le añadieron las siguientes características nuevas como una interfaz integrada para la cámara, la grabadora de vídeo y la galería, se actualizó la búsqueda por voz añadiendo soporte a más aplicaciones nativas y la posibilidad de llamar a contactos, se añadió un buscador general en la pantalla de inicio donde se podía buscar contactos, historiales y páginas web, se añadió un nuevo framework de gestos y las herramientas de desarrollo llamado GestureBuilder.

	\item \textbf{2.0 / 2.1 (Eclair):} la versión 2.0 fue presentada el 26 de octubre de 2009 y la 2.1 fue liberada el 3 de diciembre del 2009. Se añadieron un gran número de mejoras, se optimizó la velocidad de hardware, se soportaron más tamaño de pantallas y resoluciones, se rediseñó la interfaz de usuario, el navegador también fue renovado y se le añadió soporte para HTML5, nueva lista de contactos, se añadió soporte para el flash de la cámara, zoom digital, soporte para bluetooth 2.1, se mejoraron la captura de eventos multi-touch con MotionEvent y fondos de pantalla animados.
	
	\item \textbf{2.2 (Froyo):} fue lanzada el 20 de mayo de 2010. Se optimizó el sistema Android, la memoria y el rendimiento, se mejoró la velocidad de las aplicaciones gracias a la implementación de JIT, se implementó el motor JavaScript V8 de Google Chrome en el navegador del móvil, nueva funcionalidad de WiFi hotspot y tethering por USB, se actualizó el Android market para que tuviera actualizaciones automáticas, marcación por voz y compartir contactos por Bluetooth, soporte para contraseñas numéricas y alfanuméricas, soporte para Adobe Flash 10 y soporte para pantallas de HDPI, como pueden ser pantallas de 4" y resolución de 720p. 

	\item \textbf{2.3 (Gingerbread):} fue presentado el 6 de diciembre del 2010. Cambiaron el diseño de la interfaz de usuario, añadieron soporte para pantallas extra grandes y resoluciones WXGA, soporte nativo para VoIP SIP, reproducción nativa de vídeos WebM/VP8 un formato de vídeo patrocinado por Google que es la alternativa al H264 en la reproducción de vídeo en HTML5 y decodificación de audios en AAC, se añadió soporte a NFC (Near Field Communication), nuevo teclado multitáctil, soporte mejorado para programar en código nativo, soporte nativo de más sensores como pueden ser acelerómetros o barómetros, soporte para múltiples cámaras y cambio del sistema de archivos YAFFS a ext4. La versión 2.3.3 sigue siendo la versión de Android más usada actualmente. 

	\item \textbf{3.0 / 3.1 / 3.2 (Honeycomb):} Esta versión fue diseñada exclusivamente para tablet, por lo que no hubo smartphones que actualizaran a esta versión. Las características principales fueron un escritorio en 3D con widget rediseñado, sistema multitarea mejorado, mejoras en el navegador de internet, videochat mediante Google Talk, mejoras en el soporte de redes WiFi, añadidos soporte para gran cantidad de periféricos y conexión USB.
	
	\item \textbf{4.0 (Ice Cream Sandwich):} Fue una de las actualizaciones más importantes que ha recibido Android y fue lanzada el 19 de octubre de 2011, en ella se unificaron todas las versiones y se tenía una sola versión para smartphne, televisores, tablets, netbooks, etc. Se añadió una nueva versión de interfaz mucho más limpia y usable llamada Holo, una nueva fuente llamada Roboto, se da la opción de utilizar botones virtuales en la interfaz de usuario en vez de botones físicos, soporte para aceleración gráfica por hardware, por lo que la interfaz es manejada y dibujada por la GPU, aumentando notablemente el rendimiento, multitarea mejorada, se ha añadido un nuevo corrector ortográfico, en la lista de notificaciones se pueden eliminar las que no sean interesantes, capturas de pantalla pulsando el botón de encendido y el de bajar volumen, mejorada la aplicación encargada de hacer fotografías, añadida una nueva opción para crear fotos panorámicas, Android Beam, una nueva característica que nos permite compartir contenidos entre teléfonos mediante NFC, reconocimiento de voz del usuario, reconocimiento facial, para bloqueo y desbloqueo del teléfono, añadidas nuevas carpetas que se crean sólo con arrastrar y soltar, un único y nuevo framework para crear aplicaciones y soporte para contenedor MKV.

	\item \textbf{4.1 (Jelly Bean):} Esta es la última versión de Android que hay en el mercado, fue lanzada el 27 de junio de 2012 durantes la última Google I/O. Se mejoró la fluidez y la estabilidad gracias al proyecto ``Project Butter", ajuste automático de widget cuando se añaden al escritorio, se añadió soporte para lenguas no occidentales, mejora de Android Beam para poder enviar video por NFC, dictado de voz mejorada y sin tener que tener conexión a internet para usarlo, nuevas notificaciones en las que se puede añadir botones para controlar o tener acceso a opciones más comunes, como puede ser responder a un email, pulsar pause o pasar de canción, nueva función Google Now que intenta ser el competidor de SIRI del iPhone en Android, cifrado de aplicaciones y nuevas actualizaciones incrementales, en las que no es necesario volver a bajar toda la aplicación para actualizarla, sólo se baja las partes nuevas, Google Chrome se convierte en el navegador por defecto de Android y se pone fin al soporte de Adoble Flash Player, se añade una nueva función llamada Sound Search que permite identificar la canción que está sonando, se ha añadido una nueva función llamada Gestual Mode para personas discapacitadas visualmente.
\end{itemize}

En el proyecto se ha usado la versión \textbf{Android 4.0} para el desarrollo.

En la figura \ref{fig:Android41} se puede ver como es visualemte Android en la actualidad, con la versión 4.1.

\begin{figure}
  \centering
    \includegraphics[scale=0.2]{./ConocimientosPrevios/imagenes/android41.jpeg}
  \caption{Versión de Android 4.1}
  \label{fig:Android41}
\end{figure}


\section{Google App Engine.}

Google App Engine es una plataforma de cloud computing para desarrollar y almacenar aplicaciones web que ofrece Google. Las aplicaciones web pueden ser escalables y si necesitan más recursos automáticamente le son asignados para poder seguir ofreciendo servicio. Google App Engine es gratis para un cierto número de peticiones y almacenamiento y la primera versión fue lanzada en abril de 2008.

Actualmente se puede desarrollar en tres lenguajes, que son Java, Python y Go, este último un lenguaje creado por Google. En el proyecto hemos usado Java para desarrollar en la plataforma. 

Google App Engine para Java soporta muchos estándares y framework, y el Core está hecho con la tecnología Servlet 2.5 usando el servidor de software libre llamado Jetty Web Server, acompañado con otras tecnologías como JSP. El almacén de datos puede ser muy poco intuitivo desde el punto de vista de los desarrolladores, pero se puede acceder fácilmente con JPA (Java Persistence API) y los métodos JDO (Java Data Objects) para escritura y lectura de datos. También se pueden usar tecnologías como Spring Framework.

Google garantiza que aplicación estará disponible el 99.95\% del tiempo, para ello ofrece una alta replicación.

La base de datos a la que nos dan acceso no es una base de datos estándar, como pueden ser MySQL, Oracle o SQLServer, pero tiene una sintaxis muy parecida a SQL, llamada GQL. Una de las principales diferencias es que no admite sentencias join, debido a la ineficiencia de la misma. La versión de Java soporta consultas asíncronas no bloqueantes, ofreciendo así una forma de procesamiento paralelo de datos.

Para más información se puede visitar la web diseñada para desarrolladores que proporciona Google, \url{https://developers.google.com/appengine/}. 

En el capítulo \ref{cap:GAE} veremos más ampliamente todo lo relacionado con Google App Engine que hemos usado en el proyecto.

\section{SQLite.}

SQLite es un sistema gestión de base de datos relacionales compatibles con ACID, ACID es el acrónimo de Atomicity, Consistency, Isolation and Durability, que son las características que debe tener una base de datos para que se consideren base de datos relacional. Su característica principal es que ocupa muy poco espacio, alrededor de 275 Kb y fue escrita en el lenguaje C por Richard Hipp. Está distribuida bajo licencia de dominio público.

A diferencia de los sistemas de gestión de base de datos clientes-servidor, el motor de SQLite pasa a ser parte del programa que quiere usarlo, ya que se integra con él. Esto hace que tenga mayor rendimiento debido a que la comunicación es por medio de funciones, que es mucho más eficiente que mediante comunicación de procesos. La totalidad de la base de datos, tablas, índices y datos, se guardan en un solo fichero estándar en la máquina host. La versión 3 de SQLite permite base de datos de hasta 2 Terabytes y permite campos del tipo BLOB.

SQLite está muy extendido y se puede programar en infinidad de lenguajes de programación como pueden ser C, C++, Perl, Python, PHP, Java, etc.

Es utilizado en infinidad de programas y sistemas, que van desde editores de imagen como puede ser Adobe Photoshop Elements, reproductores de sonido como Clementine o navegadores como Firefox, Chrome u Opera.

Esta es una de las tres formas que proporciona Android para guardar datos, al estar embebida en cada aplicación para mejorar el rendimiento, cada aplicación debe de tener una. 

\section{XML.}

XML son las siglas en inglés de e\textbf{X}tensible \textbf{M}arkup \textbf{L}anguage que es un lenguaje de marcas desarrollado por el World Wide Web Consortium (W3C), deriva del lenguaje SGML y permite definir la gramática de lenguajes específicos para estructurar documentos grandes.

Es una de las formas de intercambio estructurado de información más extendidas en internet, ya que se puede usar en base de datos, hojas de cálculo o en casi cualquier información que se quiera usar.

XML es un lenguaje que puede ser analizado sintácticamente para averiguar si está bien construido o no, por lo que cualquier parser (analizador sintáctico) puede confirmar si tiene la estructura bien definida según el estándar. Todos los documentos tiene que tener las siguientes partes: prólogo, cuerpo, elementos y atributos.

Un ejemplo podemos verlo en el siguiente un trozo de código XML, para almacenar un libro en una librería.

\begin{lstlisting}[language=XML]   
<?xml version="1.0"?>
<libro>
<titulo> A Game of Thrones </titulo>
<disponible tiempo="24" unidad="horas"/>
<autor> George R. R. Martin </autor>
<formato> Rustica </formato>
<publicacion> 1996 </publicacion>
<precio cantidad="9.99" moneda="euro"/>
<descuento cantidad="5"/>
<enlacelibro href="/exec/ISBN/0-553-10354-7"/>
</libro>
\end{lstlisting}

En el proyecto hemos usado XML, en los archivos de configuración o para el diseño de las interfaces en Android. En una de la aplicaciones web realizadas para los archivos de configuración también hemos usado XML, en la otra hemos usado un lenguaje parecido llamado YALM, que es equivalente a XML.

\section{UML.}

UML es un lenguaje de modelado de propósito general más usado en la actualidad para el diseño de software. UML son las siglas de Unified Modeling Language. UML tiene la ventaja de que se puede observar visualmente el diseño del software. Se puede desde especificar, construir o documentar un sistema o un software.

\begin{figure}
  \centering
    \includegraphics[scale=0.4]{./ConocimientosPrevios/imagenes/UMLDiagrams.jpg}
  \caption{Ejemplos de diseños realizados con UML.}
  \label{fig:UMLDiagrams}
\end{figure}

En la figura~\ref{fig:UMLDiagrams} podemos ver los diferentes diagramas que podemos realizar mediantes UML.

Hemos usado UML para diseñar las clases usadas a lo largo de todo el proyecto, además de para mostrar en la memoria del proyecto la relación de las clases, los casos de uso, etc.

\section{GIT.}

Git es un software de control de versiones, a la vez que mantenedor de la coherencia y cohesión del código fuente orientado a la velocidad. Fue desarrollado para el manejo del código fuente de Linux y al principio fue diseñado por Linus Torvalds. GIT es un repositorio con un completo historial y una capacidad de identificación de cada cambio realizado sin depender del acceso a la red o a un servidor central. Está liberado con licencia GNU versión 2.

En el proyecto hemos estado usando GIT como control de versiones, ya que si en algún cambio ocurría algún problema poder volver atrás y para tener una copia de seguridad del proyecto, alojado en todo momento en un servidor externo por si ocurría algún problema en el ordenador donde desarrollamos el proyecto.

A lo largo del proyecto hemos usado dos servicios gratuitos  que proporcionan servidores GIT, como pueden ser \url{https://github.com/} y \url{https://bitbucket.org/}. El primero es muy conocido actualmente y hay una gran comunidad de software libre en dicha plataforma, tiene un pequeño problema para nuestro proyecto y es que el código tiene que ser libre y en la versión gratuita no se pueden tener repositorios privados, cosa que el segundo servicio si te lo ofrece. Para todo el código fuente usando, tanto en la aplicación de Android como en las dos aplicaciones hemos usado bitbucket y para la memoria de proyecto hemos usado github. La dirección del repositorio de la memoria es la siguiente: \url{https://github.com/t321/memoriaPFC}, el repositorio de las aplicaciones web \url{https://bitbucket.org/t321/pfcaeg} y el de la aplicación de Android \url{https://bitbucket.org/t321/pfcandroid}.

La configuración básica para el uso de GIT con Eclipse se puede ver en el apendice~\ref{cap:apendiceA}.  















	%criptografía usada en el proyecto
	\chapter{Criptografía usada en el proyecto.}

En la sección \ref{lbl:criptografia} del capítulo \ref{cap:conocimientos} hemos hecho una breve introducción a la criptografía de clave pública, a continuación vamos a explicarla con más extensión y los usos que le hemos dado en el proyecto.

Toda la criptografía que hemos usado es criptografía de clave pública, y en concreto el criptosistema llamado RSA. Se pensó en usar DSA pero finalmente se usó RSA. En el proyecto se ha usado la criptografía no para cifrar nada, si no para firmar digitalmente.

\section{RSA.}

RSA se basa en la idea de que múltiplicar dos números enteros primos, pero muy dificil sabiendo el resultado de la multiplicación averiguar dichos dos números.

El proceso para generar las claves es el siguiente:
\begin{itemize}

	\item Se eligen dos números primos grandes que llamaremos \textbf{p} y \textbf{q}, dichos números se múltiplican y obtenemos \textbf{n}, $n = p \cdot q$, dicho valor de $n$ es el que se usa como módulo de la clave privada y pública. 

	\item Calculamos $\varphi(n)$ de la siguiente forma $\varphi(n)=(p-1)\cdot(q-1)$, donde $\varphi(n)$ es la función $\varphi$ de Euler.

	\item Se elige un entero positivo $e$ menor que $\varphi(n)$ y que sea coprimo con $\varphi(n)$. El valor $e$ es el exponente de la clave pública.
	
	\item Se busca un valor $d$ que satisfaga la congruencia $d=e^{-1}\,mod\,\varphi(n)$. Este valor se suele calcular con el algoritmo de Euclides extendido y el valor $d$ es el exponente para la clave privada. 
\end{itemize}

La clave pública será $(n,e)$ y la clave privada será $(n,d)$.

El proceso de cifrado y descifrado será el siguiente.

\begin{itemize}
	\item \textbf{Cifrado:} el receptor (Alice) envia su clave pública $(n,e)$ al emisor (Bob) y guarda la clave privada, $(n,d)$, en secreto, a partir de este momento Alice usando la clave pública de Bob puede comunicarse seguramente. El mensaje que Bob quiere enviar a Alice será $M$. Bob primero convierte $M$ en un número menor que $n$, y que sea de una forma reversible de forma que sabiendo $n$ podamos volver a conseguir $M$. Acto seguido calcula $c$ de esta forma, $c\equiv m^e\,(mod\,n)$. Bob mandaría $c$ a Alice y la comunicación finalizaría.
	
	\item \textbf{Descifrado:} Alice empezará el proceso para recuperar $m$ a partir de $c$ y $d$. $m\equiv c^d\,(mod\,n)$, una vez ha calculado $m$ puede conseguir $M$.

\end{itemize}

El proceso de descifrado funciona porque $c^d=(m^e)^d\equiv m^{ed}\, (mod\, n)$ y hemos elegido $d$ y $e$ de forma que $e\cdot d =1+k\cdot\varphi(n)$, se cumple que $ m^{ed}\cdot m^{1+k\cdot\varphi(n)} = m(m^{\varphi(n)})^k = m\, (mod\, n)$, esta última congruencia se obtiene directamente del teorema de Euler cuando $m$ y $n$ son coprimos.

Un ejemplo de cifrado y descifrado es el siguiente:



Aquí tenemos un ejemplo de cifrado/descifrado con RSA. Los parámetros usados aquí son pequeños y orientativos con respecto a los que maneja el algoritmo, pero podemos usar también OpenSSL para generar y examinar un par de claves reales.
p=61 	1º nº primo privado
q=53 	2º nº primo privado
n=pq=3233 	producto p*q
e=17 	exponente público
d=2753 	exponente privado

La clave pública (e, n). La clave privada es (d, n). La función de cifrado es:

        \mbox{encrypt}(m) = m^e\pmod{n} = m^{17}\pmod{3233}

Donde m es el texto sin cifrar. La función de descifrado es:

        \mbox{decrypt(c)} = c^d\pmod{n} = c^{2753}\pmod{3233}

Donde c es el texto cifrado. Para cifrar el valor del texto sin cifrar 123, nosotros calculamos:

        \mbox{encrypt(123)} = 123^{17}\pmod{3233} = 855

Para descifrar el valor del texto cifrado, nosotros calculamos:

        \mbox{decrypt(855)} = 855^{2753}\pmod{3233} = 123

Ambos de estos cálculos pueden ser eficientemente usados por el algoritmo de multiplicación cuadrática para exponenciación modular



	%android
	\input{./Android/android.tex}
	%google app engine
	\chapter{Google App Engine.}
En este apartado de la memoria voy a explicar lo que es, la configuración y el como usar la plataforma Google App Engine.

\section{Introducción.}
Google App Engine es una conjunto de apis que proporciona Google para construir tus propias aplicaciones web, que pueden ser alojadas y usadas en su servicio Google App y vendidas en Google Apps Marketplace. Además de alojamiento gratuito Google, ofrecen un dominio, que es: \url{http://nombre\_de\_la\_aplicacion.appspot.com} y una base de datos propietaria de Google que se accede transparentemente a través de la api, gestión de usuarios mediante autentificación con cuentas Google del tipo: \textit{usuario@gmail.com}, autentificación por federación o \textit{openID}.

Además de todas esas características Google proporciona apis para Java, Python y Go, este último un lenguaje experimental del propio Google. Para usar dicha API, Google también da un plugin para Eclipse, en caso de que el lenguaje elegido sea Java, que ayuda al despliegue de la aplicación web, autocompletado y gestión de de las aplicaciones creadas. 
%TODO:En el anexo 1 se puede ver como instalar el puglins para eclipse de google app engine.

En el proyecto solo he usado la API de Google App Engine de Java, por lo que todo lo que puedo comentar es de dicha API, la parte de Python y Go no se han estudiado.

En general el uso de Google App Engine para crear aplicaciones web es idéntico a crear una aplicación web con Java 2 Enterprise Edition (Java2EE), se pueden crear servlet que recogen valores \textbf{GET} o \textbf{POST} y además clases java para hacer operaciones con dichos valores. A su vez para mostrar la infomación se pueden generar archivos \textit{*.jsp}, que son archivos html con bloques o líneas de código java que se introducen con estas etiquetas: $<$\%= línea de código Java \%$>$ o $<$\% Bloque de código Java \%$>$. A parte de archivos \textit{*.java} y \textit{*.jsp}, debemos tener una carpeta llamada \textit{war} en la que tiene que ir toda la información de la aplicación web que queremos deplegar. En dicha carpeta hay varias subcarpetas como pueden ser \textit{css} en la que tiene que ir el estilo de la web o \textit{WEB-INF} en la que están todos los archivos de configuración, como pueden ser los permisos que tenemos que tener para poder acceder al uso de un servlet, si la web tiene conexión https, la configuración de la base de datos, etc.

Para este proyecto se han tenido que desarrollar dos aplicaciones web, una que es un servidor de timestamp y otra que es una aplicación para gestión de las firmas digitales que realice cada usuario. A continuación vamos a explicar en profundida la tecnología usada y ambas aplicaciones web.

\section{Explicación de una aplicación web genérica en Google App Engine.}
En esta parte voy a explicar en profundidad que es un servlet, los archivos de configuración, los archivos \textit{*.jsp} y el resto de archivos necesarios para poder desplegar una aplicación en Google Apps.
 
\subsection{¿Qué es un servlet?.}
Un servlet es la evolución de los antiguos applet, su uso más comun es generar páginas web dinámente con los parámetros que recibe mediante una petición realizada por el navegador web y datos que están almacenados en el servidor web.

Un servlet es un objeto java que tiene que ser ejecutado en un servidor web o contenedor J2EE, que recibe unos parámetros, realiza una o varias acciones y devuelve un resultado que puede ser desde un código html, un JSP que genera dinámicamente un código html, un JSON o una simple cadena de texto.

Los servlets, junto con JSP, son la solución de Oracle a la generación de contenido dinámico equivalente al lenguaje PHP, ASP de Microsoft, Ruby, etc.

Los servlet forman parte de Java Enterprise Edition (JEE) que a su vez es una amplicación de Java Standard Edition (JSE), para usarlos necesita un servidor web que pueda interpretar código java, el más famoso es Apache Tomcat que está desarrollado y mantenido por Apache Foundation, que son los encagardos también de mantener y desarrollar el famoso servidor web Apache, aunque existen otro como JBoss, Jetty o GlassFish, pero como veremos en este proyecto no son los únicos, ya que el propio Google Apps también funciona internamente a base de servlets y JSP.

Para crear un servlet hay que generar una clase java que implemente la interfaz \textit{javax.servlet.Servlet} o que extienda cualquier clase que herede de una clase o que implemente la interfaz anterior, como puede ser \textit{javax.servlet.http.HttpServlet} que es específico para conexiones HTTP. 
Una vez generada la clase hay que implementar el método \textbf{doGet} para peticiones tipo \textbf{GET} o el método \textbf{doPost} para peticiones de tipo \textbf{POST}. En el siguiente trozo se código se puede ver la implementación mas básica de los métodos \textbf{doGet} y \textbf{doPost}, con las llamadas sus respectivas llamadas a \textit{super}.
\begin{small}
 \begin{flushleft}
\begin{scriptsize}
\begin{lstlisting}[language=Java] 

@Override
protected void doGet(HttpServletRequest req, 
	HttpServletResponse resp) throws ServletException, IOException {
	// TODO Auto-generated method stub
	super.doGet(req, resp);
}

@Override
protected void doPost(HttpServletRequest req, 
HttpServletResponse resp) throws ServletException, IOException {
	// TODO Auto-generated method stub
	super.doPost(req, resp);
}
\end{lstlisting}
\end{scriptsize}
\end{flushleft}
\end{small}

Una vez implementados los métodos que se necesiten se pueden usar el parámetro \textit{HttpServletRequest req} para recibir los valores que queramos enviar a la aplicación web y podemos usar \textit{HttpServletResponse resp} para enviar lo que queramos desde una redirección a JSP o una página web a una JSON o cadena de texto. 
Un ejemplo de como se reciben los parámetros sería: 

\begin{lstlisting}[language=Java]  
String num_sec = req.getParameter("sec");
\end{lstlisting}

Y si queremos enviar algo por \textit{HttpServletResponse resp} podríamos usar:

\begin{lstlisting}[language=Java]   
PrintWriter out = resp.getWriter();
out.print(jsonArray);
out.flush();
\end{lstlisting}

Como podemos ver el objeto \textit{resp} nos da la posibilidad de conseguir un objeto \textit{java.io.PrintWriter} por el que podemos enviar lo que necesitemos.

%TODO: añadir la referencia... xD donde pone capítulo proximos xD
La forma de acceder a un servlet mandandole peticiones \textbf{GET} sería la siquiente: \url{https://servertimestamp.appspot.com/search?id=63&texto=Prueba}, como se puede ver la dirección base seria: \url{https://servertimestamp.appspot.com/}, el servlets estaría mapeado internamente en el servidor web, como ya veremos en próximas sección, en la dirección \url{/search} y el primer parámetro va precedido de \url{?id\_parámetro} y el resto de \url{\&id\_parámetro}. En nuestro ejemplo tendría dos parámetros que son \textit{id} y \textit{texto}, con sus respectivos valores después del =.

El método \textbf{POST} es el utilizado para pasar parámetros por medio de formularios.

\subsection{¿Qué es JSP?.}
JSP es el acrónimo de JavaServer Pages y es una tecnología que ayuda a crear dinámicamente páginas web badas en HTML o XML y es la solución equivalente a PHP de Oracle. En la figura~\ref{fig:modoJSP} se puede observar el proceso que sigue desde que se hace la petición en el navegador hasta que se muestra.

\begin{figure}
  \centering
    \includegraphics[scale=0.5]{./GoogleAppEngine/imagenes/JSP_Model.png}
  \caption{Modo de interpretación de un archivo JSP}
  \label{fig:modoJSP}
\end{figure}

Un fichero \textit{*.jsp} es la unión de código HTML con código java, el cual es interpretado en el momento de visualización de la página web. Un ejemplo es el siguiente:
 
\begin{lstlisting}[language=HTML]   
<!DOCTYPE html>
<html>
<body>
<table>
<tr>
	<th>ID</th>
	<th>Num sec</th>
	<th>Token de tiempo</th>
	<th>Mensaje</th>
	<th>URL para ver la firma</th>
	<th>Fecha</th>
	<th>Usuario</th>
	<th id="filadestino">Destino</th>
	<th>Verificado?</th>
</tr>

<% for (RowRepositorioGeneral row : rows) {%>
<tr>
	<td><%=row.getId()%></td>
	<td><%= row.getNum_sec()%></td>
	<td><%=row.getToken_tiempo()%></td>
	<td><%=row.getTexto_claro()%></td>
	<td><a href=<%=row.getUrl_firma()%>>URL para ver el token
			de tiempo</a></td>
	<td><%=row.getFecha()%></td>
	<td><%=row.getUsuario()%></td>
	<td id="filadestino"><%=row.getDestino()%></td>
	<td>
		<%
		Boolean confirmado = row.getConfirmado();
		if (!(confirmado == null) && confirmado) {
		%>
		<center>
			<img src="ok.png" />
		</center> <%} else  %>
	</td>
</tr>
<%}%>
</table>
</body>
</html>
\end{lstlisting}

Como se puede ver en este trozo de código este jsp genera una tabla que se rellena dinámicamente con los valores que devuelve un objeto java, se puede observar que se entrelazan trozos de código Java con etiquetas HTML. Si mostramos esta web y acto seguido introducimos otro objeto RowRepositorioGeneral en la estructura, cuando recarguemos la tabla tendrá una fila nueva.

\subsection{La carpeta WAR.}
La carpeta WAR es la carpeta principal para el despliegue de una aplicación web, ya que en ella es donde tienen que ir todos los archivos que necesitemos, desde archivos HTML, CSS, JSP, imágenes, etc. En la figura~\ref{fig:carpetawar} se puede ver un ejemplo de la carpeta WAR de mi aplicación web.

\begin{figure}
  \centering
    \includegraphics{./GoogleAppEngine/imagenes/carpetawar.png}
  \caption{Carpeta WAR}
  \label{fig:carpetawar}
\end{figure}

Se puede ver las diferentes carpetas y ficheros que la forman. Se ve la carpeta css que contiene los archivos de estilo que la página web usará, también se pueden ver los archivos \textit{web.xml} y \textit{app.yalm} que son archivos de configuración del servidor que se verán en el proximo apartado~\ref{ref_archivos_configuracion_google_app_engine} y además los archivos \textit{jsp} que se usan en la aplicación junto con los archivos \textit{html} y \textit{javascript} que se necesiten.

\subsection{Archivos de configuración.\label{ref_archivos_configuracion_google_app_engine}}
Los principales archivos de configuración son \textit{web.xml} y \textit{app.yalm}, este segundo es solo una forma de escribir de forma más legible el xml, para que nos sea más sencillo escribirlo y leerlo a los humanos.

Un ejemplo de un archivo \textit{web.xml} es el siguiente:

\begin{lstlisting}[language=XML]
<?xml version="1.0" encoding="utf-8"?>
<web-app xmlns:xsi="http://www.w3.org/2001/XMLSchema-instance"
xmlns="http://java.sun.com/xml/ns/javaee"
xmlns:web="http://java.sun.com/xml/ns/javaee/web-app_2_5.xsd"
xsi:schemaLocation="http://java.sun.com/xml/ns/javaee
http://java.sun.com/xml/ns/javaee/web-app_2_5.xsd" version="2.5">

	<servlet>
		<servlet-name>AddRow</servlet-name>
		<servlet-class>pfc.ServletCreateRow</servlet-class>
	</servlet>
	<servlet-mapping>
		<servlet-name>AddRow</servlet-name>
		<url-pattern>/add</url-pattern>
	</servlet-mapping>

	<welcome-file-list>
		<welcome-file>ServerTimestampApplication.jsp</welcome-file>
	</welcome-file-list>
</web-app>
\end{lstlisting}
%TODO: poner la referencia al código si se puede
Como se puede observar en el código se ha definido un servlet que se llamará \textbf{AddRow} que usará la clase \textbf{ServletCreateRow} y que estará mapeado en la dirección web \textbf{/add}, también podemos observar que el fichero que nos mostrará el servidor será \textbf{ServerTimestampApplication.jsp} si entramos a la url principal.

A continuación veremos como es un archivo \textit{app.yalm}:
\begin{lstlisting}[language=YAML]
application: repositoriorecibos
version: 1
runtime: java

handlers:
  - url: /add
    servlet: pfc.ServletCreateRow
    secure: always
welcome_files:
  - RepositorioGeneralApplication.jsp
\end{lstlisting}

Como podemos observar es mucho más fácil de entender y de escribir, el único problema que tienen los archivos YALM es que son sensibles a los espacios en blanco y tabuladores, por lo que hay que tener cuidado al redactarlos. En este archivo se crea un servlet en la ruta \textbf{/add}, que es la clase java \textbf{ServletCreateRow} del paquete \textbf{pfc} y que siempre hay que estar registrado en la aplicación para poder acceder a él. También podemos observar el fichero de bienvenida para cuando accedemos a la aplicación web. 
Al tener el archivo \textit{app.yalm} en la carpeta WEB-INF el parseador de YALM interpreta dicho archivo y genera un archivo \textit{web.xml} que es el que usará el servidor web para su configuración automáticamente.
\\
\\
Para ver todas las opciones de configuración que se pueden modificar en \textit{app.yalm} o en \textit{web.xml} se puede consultar estos enlaces \url{https://developers.google.com/appengine/docs/java/config/}, \url{https://developers.google.com/appengine/docs/java/configyaml/}. En el primero podemos ver todas las opciones configurables de \textit{web.xml} y el la segunda las de \textit{app.yalm}.

\section{Servidor de timestamp.}
En este apartado voy a explicar en profundidad todo lo relacionado con la aplicación de timestamp que he tenido que desarrollar, desde el diseño que se ha seguido hasta los problemas que me han surgido.

En principio me gustaría explicar para que se usa un servidor de timestamp en general. Un servidor de timestamp es un registro donde toda persona puede subir un documento y el servidor guarda ese documento añadiendole la fecha en la que se realizó la subida, dicha aplicación luego ofrece el servicio de consultar a que hora fue subido dicho documento. Un ejemplo podría ser \url{https://seguro.ips.es/servidortimestamp/index.asp} que se puede ver una captura de pantalla en la figura~\ref{fig:server_ips_timestamp}. En dicha captura podemos ver que tiene las opciones básicas de un servidor de timestamping como puede ser generar un sello, consultar su validez, etc. 

\begin{figure}[hbt]
  \centering
    \includegraphics[scale=0.5]{./GoogleAppEngine/imagenes/server_ips_timestamp.png}
  \caption{Servidor Timestamp https://seguro.ips.es}
  \label{fig:server_ips_timestamp}
\end{figure}

La veracidad de que el sellado de dicho documento fue en el instante que dice ser, depende de la confianza que se tenga en ese servicio. Es similar a cuando se necesita que te sellen un documento físico, que dependiendo de para quien lo necesite, necesitas que lo firme un notario o un empleado público si es para una entidad pública. Normalmente suelen existir servidores de timestamping en los que se tiene confianza y los documentos sellados se consideran verdaderos.

Existen tres modelos principales de servidor de timestamping que son los siguientes:
\begin{itemize}
\item \textbf{Solución Arbitrada básica:} En esta solución el usuario que quiere sellar algo mandaría una copia del documento que quiere sellar a la entidad de sellado, que a su vez pondría el sello de tiempo y a su vez guardaría una copia de dicho documento, este es el modelo mas parecido a la vida real. Esta solución tiene un par de problemas grandes como puede ser que la privacidad del documento se pierde, tenemos que tener en cuenta que el servidor de timestamping puede estar en España, EEUU o en cualquier otro país y a su vez la base de datos para almacenar todos los documentos tiene que ser enorme, por lo que almacenar todos los documentos nos puede acarrear muchos problemas.

\item \textbf{Solución Arbitrada avanzada:} Esta solución es una evolución de la anterior, en ella el cambio que se hace es que el usuario que quiere que le sellen el documento manda el hash de dicho documento, un hash es el resultado de una función unidireccional que recibe un documento y devuelve un valor único, teniendo dicho valor no se puede saber el documento original, pero dicho documento siempre creará ese valor único, y la entidad solo tendría que almacenar dicho hash junto con el sello de tiempo que se ha generado. Esta solución no tiene los inconvenientes de la anterior, ya que el tamaño de los documentos se reduciría a unos pocos bytes, y la privacidad del documento no se ve comprometida. El problema que si persiste es que el usuario conozca a la entidad de certificación y puedan generar timestamp falsos, pero este problema ya va dentro de la confianza que queramos darle a ese servicio. Suponemos que si es un servicio oficial y serio este problema no va a suceder, de todas formas existen otras soluciones que arreglan este problema.

\item \textbf{Solución Arbitrada avanzada y distribuida:} Esta forma consigue arreglar el problema de la anterior que se produzca un uso fraudulento del timestamping es usando varias entidades de timestamping, por lo que el usuario mandaría el hash a varias entidades de sellado y el usuario guarde los reguardos que están firmados digitalmente de todas las entidades. Así si en una hay un problema tiene varias copias que certifican que se selló dicho instante.

\item \textbf{Solución mediante enlaces:} Esta solución es la mas compleja y a su vez la que soluciona todos los problemas anteriores, además tiene la ventaja de que no tiene que usar multitud de entidades de certificación. Consiste en que cuando un usuario quiere sellar algo, manda el hash del documento, la entidad añade el número de serie del documento anterior, el timestamp y lo firma digitalmente, entonces el problema de que se introduzcan valores fraudulentos por mitad se anula, ya que cada recibo está enlazado con el anterior.
\end{itemize}

En nuestro caso hemos desarrollado un servidor de timestamping en su versión solución arbitrada avanzada.

A continuación voy a explicar la aplicación web, los servlets que la componen y sus funciones.

\subsection{Explicación de la aplicación web.}
%TODO: falta explicar como funciona la BD...
En este capítulo voy a explicar todas las partes que componen la aplicación web que he desarrollado para la implementación del servidor timestamp.

En la figura~\ref{fig:paquete_pfc} se puede ver las clases que forman el paquete \textit{pfc}.

\begin{figure}
  \centering
    \includegraphics[scale=0.5]{./GoogleAppEngine/imagenes/UML_pfc.png}
  \caption{Detalles del paquete pfc}
  \label{fig:paquete_pfc}
\end{figure}

A continuación voy a explicar una a una las clases desarrolladas.

\begin{description}
\item \textbf{Dao.java: \label{prueba}} Esta clase es la encargada de todos los accesos a la base de datos, desde insercción, borrado y listado de las filas, hasta consultas que se necesiten hacer. Se puede observar que las consultas que son de listado de columnas se ejecutan con una sentencia SQL, un ejemplo es la siguiente:
\begin{lstlisting}[language=Java]
EntityManager em = EMFService.get().createEntityManager();
Query q = em.createQuery("select t from RowTimerstamp t where t.num_sec = :num_sec");
q.setParameter("num_sec", id);
RowTimerstamp RowTimerstamps = (RowTimerstamp)q.getSingleResult();
\end{lstlisting}

Pero las consultas que implican borrado o inclusión de filas no se realizan con sentencias SQL convencionales, se añaden con métodos que proporciona la API, un ejemplo es el siguiente:
\begin{lstlisting}[language=Java]
EntityManager em = EMFService.get().createEntityManager();
RowTimerstamp RowTimerstamp = new RowTimerstamp(num_sec, firma, fecha);
em.persist(RowTimerstamp);
em.close();
\end{lstlisting}

\item \textbf{RowTimerstamp.java:} En esta clase se diseña el formato de las filas de la base de datos, que como se ha explicado anteriormente no se crea con sentencias SQL, se usa un modelo de programación llamado JPA. Para dicho modelo hay que crear una clase que contenga como variables de clase las columnas de la tabla de la base de datos. Como podemos ver mediante anotaciones Java se le indica que campo es la clave primaría, también se puede indicar que ese campo es autoincrementado y otras opciones que habría que indicar en la creación de la tabla.   

\begin{lstlisting}[language=Java]
@Id
@GeneratedValue(strategy = GenerationType.SEQUENCE)
private Long id;
private Long num_sec;
private String firma;
private Date fecha;
\end{lstlisting}

Se puede ver que el campo id será la clave primaría que se indica con \textit{@id} y que es autoincremental, a su vez también podemos ver el resto de datos que se van a guardar, el campo \textit{num\_sec} es el número de secuencia, ya que el campo \textit{@id} lo usa la base de datos para organizarse ella, el campo \textit{firma} es hash firmado por el usuario y que se quiere tener constancia de que se subío en la fecha que indica el campo \textit{fecha}.
El resto de métodos que tiene esta clase es un contructor, getter para consultar los campos y setter para insertar valores.

\item \textbf{ServletCreateRow.java:} Esta clase es un servet que se encagar de recibir todos los parámetros necesarios y añadirlos a la base de datos. Al recibir los parámetros mediante \textbf{GET} tiene que implementar el método \textit{doGet}, casi todos los servlets implementados en el proyecto mandan los parámetros mediante \textbf{GET}. La forma de recibir parámetros es la siguiente:

\begin{lstlisting}[language=Java]
String firma = req.getParameter("firma");
\end{lstlisting}

El resto de parámetros que se necesitan se generan en el servidor para que no puedan ser falseados, como es el número de secuencia y la fecha.

Si la insercción se produce correctamente se devuelve una cadena que tiene el siguiente formato: ``ok;;num\_sec;;fecha" que será interpretado en la aplicación android y que parseará dicha cadena para conseguir los valores que necesitemos.

\item \textbf{ServletDeleteAll.java:} Es un servlet ``secreto" que se usa para borrar todas las filas del servidor, cosa que no se debería poder para no poder falsear los datos introducidos en el servidor de timestamp. Hay que llamarlo con un parámetro que es \textbf{borrar} con valor \textbf{5}.

\item \textbf{ServletSearchDate.java:} Es un servlet que devuelve una cadena con la fecha de una fila que tiene el número de secuencia que se le pasa en el parámetro \textbf{token}.

\item \textbf{ServletSearchRow.java:} Es un servlet que devuelve una página web donde se puede observar en una única columna toda la infomación almacenada que corresponde con el número de secuencia que se le pasa en el parámetro \textbf{id}. La forma de hacerlo es la siguiente:

\begin{lstlisting}[language=Java]
PrintWriter pw = resp.getWriter();
pw.print("<!DOCTYPE html>");
pw.print("<html><head><title>Lista Time Stamp</title><link rel=\"stylesheet\" type=\"text/css\" " +
		"href=\"css/main.css\"/> <meta charset=\"utf-8\"> </head>");
pw.print("<body><table><tr><th>ID</th><th>Num sec</th><th>Firma</th><th>Date</th></tr><tr> " +
		"<td>"+ row.getId() +"</td><td>"+ row.getNum_sec() +"</td><td>"+ row.getFirma() +"</td><td>" +
		row.getFecha() + "</td></tr> </table></body>");
pw.flush();
\end{lstlisting}

Como se puede ver se crea una tabla en una web que su fila se generan dinámicamente dependiendo del número de secuencia que se le pase.

\item \textbf{ServletSearchSign.java:} Es un servlet que devuelve una cadena con la firma que corresponde al número de secuencia que se pasa por el parámetro \textbf{token}.
\end{description}
%TODO: falta los jsp

\section{Servidor de registro de firmas.}
El servidor de registro de firmas que hemos desarrollado es una aplicación web en la que se pueden consultar las firmas que tu has realizado, las firmas en las que el destinatario eres tú, gestión del certificado de clave pública, verificar una firma, exportar una cadena con la que cualquier persona puede mirar si la firma que has realizado es válida para comprobaciones en caso de algún problema, y generar códigos QR para que que alguien que lo necesite pueda firmarlo.

El sistema de gestión de usuarios la proporciona google, y para entrar en la aplicación web hay que tener una cuenta de google, si no se produce una redirección a la página de logueo que se puede ver en la figura~\ref{fig:logueoRepoGeneral}.
\begin{figure}
  \centering
    \includegraphics[scale=0.5]{./GoogleAppEngine/imagenes/login_repositorio_general.png}
  \caption{Login en Repositorio General}
  \label{fig:logueoRepoGeneral}
\end{figure}
La parte de la seguridad de los usuarios, logueo y mantenimiento de las base de datos ya las proporciona el mismo Google.

La aplicación web se puede ver en la figura~\ref{fig:repositorio_general}.
\begin{figure}[gae]
  \centering
    \includegraphics[scale=0.5]{./GoogleAppEngine/imagenes/repositorio_general.png}
  \caption{Repositorio General}
  \label{fig:repositorio_general}
\end{figure}

La aplicación tiene varias pestañas, que se explicarán posteriormente cuando expliquemos cada archivo \textit{*.jsp}, pero principalmente cada una de ellas se encarga de hacer una de las funciones que hemos comentado anteriormente.

\subsection{Explicación de la aplicación web.}
%TODO: falta explicar como funciona la BD...
En la figura~\ref{fig:clasesReposotorioGeneral} se puede ver las clases que forman el paquete \textit{pfc} de la aplicación web repositorio general.

\begin{figure}

    \includegraphics[scale=0.5]{./GoogleAppEngine/imagenes/UML_repositorio.png}
  \caption{Detalles de las clases Repositorio General.}
  \label{fig:clasesReposotorioGeneral}
\end{figure}

A continuación vamos a explicar una a una las clases desarrolladas.

\begin{description}
\item \textbf{Dao.java:} Al igual en el servidor de timestamp esta clase es la encargada de hacer todas operaciones contra la base de datos. Para mas información mirar el apartado~\ref{prueba}.

\item \textbf{DaoUserCert.java:} En este aplicación hemos utilizado dos bases de datos, una para guardar las firmas y otra para guardar los certificados de clave pública que se necesitan para verificar si una firma es correcta o no. Esta clase es la encargada de todos los accesos, tanto inserciones como consultas, a dicha tabla. 

\item \textbf{RowRepositorioGeneral.java:} Esta es la clase con la que se crea la tabla en la que se almacenan las firmas de los usuario, tiene los siguientes campos:  

\begin{lstlisting}[language=Java]
@Id
@GeneratedValue(strategy = GenerationType.SEQUENCE) //	 GenerationType.IDENTITY
private Long id;
private Long num_sec;
private String url_firma;
private String texto_claro;
private Long token_tiempo;
private String usuario;
private Boolean confirmado;
private String destino;
private BlobKey blobKey;
private Date fecha;
\end{lstlisting}

Tiene una clave primaría que es \textit{id} que es usada por Google internamente para el almacenado de la información, \textit{num\_sec} es el número de secuencia dentro la tabla que va incrementandose automáticamente, \textit{url\_firma} es la dirección en la cual se puede consultar la firma del texto en claro que está en el campo \textit{texto\_claro}. También se guarda el \textit{token\_tiempo} que es el \textit{num\_sec} de en la aplicación web del servidor de timestamp. La columna \textit{usuario} almacena el usuario que ha subido la firma, y en La columna \textit{destino} se guarda a quien va dirigido la firma, ya que todas firmas tienen un destinatario. En la columna \textit{blobkey} se guarda la referencia al certificado de clave pública que estaba en activo cuando fue subido a la aplicación web, la columna \textit{confirmado} puede valer true or false e indica si al subir la firma se pudo verificar y el texto en claro coincidía con el texto cifrado, en \textit{fecha} está la fecha en la que se almacenó.

\item \textbf{RowUserCert.java:} Esta clase es la encagargada de crear la tabla que usamos para guardar los archivos con la clave pública.
%TODO: poner el tipo de archivo que necesitamos usar y esas cosas.
Los campos que usaremos para almacenarlos serán los que se pueden ver en 

\begin{lstlisting}[language=Java]
@Id
@GeneratedValue(strategy = GenerationType.SEQUENCE)
private Long id;
private String usuario;
private Date fecha;
private BlobKey certificado;
\end{lstlisting}

Como podemos observar el campo \textit{id} será la clave pública y como hemos explicado será usado por la base de datos de Google para autogestión de las filas, el campo \textit{usuario} guardará una cadena con el email de la persona que ha subido ese archivo, el campo \textit{fecha} es la fecha en la que se subió el archivo, dicho campo se usará para comprobar que no se puedan falsear firmas, con varias comprobaciones y para anular certificados en caso de perdidas o que se necesite reemplazarlo, \textit{certificado} es un campo del tipo \textbf{BlobKey} que es como la ruta al archivo de certificado.

\end{description}
A continuación vamos a explicar los diferentes servlets que hemos desarrollado para la aplicación web.

\begin{description}
\item \textbf{ServletCreateCertificate.java:} Este servlet es el encargado de añadir a la base de datos el certificado de clave pública. Es usado en la pestaña de certificados de la aplicación web y es llamado cuando se pulsa subir certificado. Se puede observar el botón subir certificado en la figura~\ref{fig:pestanhaCertificados}.

\begin{figure}
  \centering
    \includegraphics[scale=0.5]{./GoogleAppEngine/imagenes/certificadosRepositorioGeneral.png}
  \caption{Pantallazo de la pestaña certificados.}
  \label{fig:pestanhaCertificados}
\end{figure}

\item \textbf{ServletCreateRowRepositorio.java:} Este servlet está mapeado en la dirección: \url{https://repositoriorecibos.appspot.com/add} y recibe los siguientes parámetros: \textit{texto}, \textit{url\_firma}, \textit{token}, \textit{destino} y \textit{fecha}. Es el encargado de añadir una fila por cada llamada a dicha dirección, a dicho dirección no hay forma de acceder desde la aplicación web, solo se pueden comprobar las firmas ya introducidas. A su vez antes de introducir la fila comprueba que la firma se puede validar y marca como verdadero o falso la columna verificado que posteriormente en el archivo \textit{RepositorioGeneralApplication.jsp} se cambiará por una imagen para hacer la verificación más visual. Si se hemos podido insertar la fila, el servlet devuelve la cadena \textbf{OK}, si no se devuelven varias cadenas con los fallos que se han producido.

\item \textbf{ServletDeleteAll.java:} Servlet ``secreto" que borra todas las filas de firmas almacenadas, hay que llamarlo con un parámetro que es \textit{borrar} con valor \textit{7}

\item \textbf{ServletExport.java:} Este servlet es el encargado de exportar una fila de nuestras filas para que otra persona pueda comprobar si es válida. Este servlet es llamado cuando se pulsa el botón exportar de la pestaña principal de la aplicación web. Se puede observar en la figura~\ref{fig:botonExportar}

\begin{figure}
  \centering
    \includegraphics[scale=0.5]{./GoogleAppEngine/imagenes/botonExportar.png}
  \caption{Detalle del botón exportar.}
  \label{fig:botonExportar}
\end{figure}

El servlet recibe los siguiente parámetros:

\begin{lstlisting}[language=Java]
String mensaje = checkNull(req.getParameter("mensaje"));
String url_firma = checkNull(req.getParameter("token"));
String id_blob = checkNull(req.getParameter("id_blob"));
String user = checkNull(req.getParameter("usuario"));
\end{lstlisting}

Una vez se tienen esos parámetros creamos una cadena de texto en la que unimos los siguiente campos y cada parámetro va separado por el separador: \textit{;/:}.
\begin{lstlisting}[language=Java]
String cadACodificar = mensaje + ";/:" + url_firma + ";/:" + id_blob + ";/:" + user;
\end{lstlisting}

Acto seguido codificamos la cadena con \textbf{Base64}, que es una forma simple de codificar los caracteres para que no viajen en texto claro.
\begin{lstlisting}[language=Java]
String cadCodificada = Base64.encode(cadACodificar.getBytes("UTF-8"));
\end{lstlisting}

También añadimos unos limitadores para que cuando tengamos que decodificar ese mensaje podamos saber donde empiezan y donde termina la exportación.
\begin{lstlisting}[language=Java]
pw.println("BEGIN EXPORT");
pw.println("--------------------------");
pw.println(cadCodificada);
pw.println("--------------------------");
pw.println("END EXPORT");
\end{lstlisting}

\item \textbf{ServletListRow.java:} Este servlet es el encargado en devolver todas las filas de la tabla que pertenecen a un usuario. La forma de hacerlo es la siguiente, primero se identifica el usuario con el que se ha logueado de esta forma:

\begin{lstlisting}[language=Java]
UserService userService = UserServiceFactory.getUserService();
User user = userService.getCurrentUser();
\end{lstlisting}

Una vez se consigue el usuario se llama a la función \textit{public List<RowRepositorioGeneral> getRowRepositorioGeneralList(String userId, Long num\_sec)} de la clase \textit{Dao.java}, esta última función nos devuelve una lista con todas las filas. Al llamar al servlet le pasaremos el último número de secuencia que tenemos guardado en el telefono móvil, para así agilizar las transferencias de datos, de esta forma solo nos devolverá las filas nuevas. La forma de devolvernos las filas será en un JSONArray, que es un objeto que dentro contiene varios objetos JSON\footnote{Para saber que es un objeto JSON pueden consultar los siguientes enlaces: \url{http://www.json.org/} o \url{http://en.wikipedia.org/wiki/JSON}}.

La creación de los objetos JSON la realizamos de la siguiente forma:
\begin{lstlisting}[language=Java]
JSONObject jsonObject = new JSONObject();

jsonObject.put("num_sec", rowRepositorioGeneral.getNum_sec().toString());
jsonObject.put("texto", rowRepositorioGeneral.getTexto_claro());
jsonObject.put("url_firma", rowRepositorioGeneral.getUrl_firma());
jsonObject.put("token_tiempo", rowRepositorioGeneral.getToken_tiempo().toString());
jsonObject.put("usuario",rowRepositorioGeneral.getUsuario());
jsonObject.put("fecha", rowRepositorioGeneral.getFecha().toString());
jsonObject.put("verificado", rowRepositorioGeneral.getConfirmado().toString());
jsonObject.put("destino", rowRepositorioGeneral.getDestino());
\end{lstlisting}

Ese objeto JSON se añade un objeto JSONArray que a su vez es el que devolveremos como respuesta al final de la ejecución de nuestro servlet y que espera la aplicación android que lo ha pedido.

\item \textbf{ServletVerify.java:} Este servlet es el utilizado en la pestaña verificar de nuestra aplicación web, como se puede observar en la figura~\ref{fig:pestañaVerificar}

\begin{figure}

    \includegraphics[scale=0.5]{./GoogleAppEngine/imagenes/pestanhaVerificar.png}
  \caption{Detalle de la pestaña verificación.}
  \label{fig:pestañaVerificar}
\end{figure}

Como podemos observar hay un cuadro de texto para introducir la cadena que devolvería al pulsar el botón exportar. Cualquier usuario puede verificar si una firma es correcta o no. En este servlet se hace el proceso contrario que hicimos en exportar, quitamos los indicadores de inicio y final de exportación, desencriptamos la cadena en Base64 y hacemos varias comprobaciones. Comprobamos que en la fecha en la que se firmó el certificado era válido y que no habiamos revocado ese certificado, también se comprueba que no fuera reemplazado por otro certificado antes de su expiración, ya que entonces la firma no sería válida. También comprobamos la integridad del mensaje, que la cadena no esté mal formada y que siga el formato que hemos obligado anteriormente.
\end{description}

Los archivos \textit{*.jsp} que hemos creado en su mayoría solo rellenan tablas dinámicamente haciendo llamadas a funciones de la clase \textit{Dao.java}. Solo habría uno que no realiza esas funciones que es el siguiente:
\begin{description}
\item \textbf{GenerarQR.jsp:} Este archivo JSP es el que se muestra en la pestaña \textit{Generar QR}, se puede ver en la figura~\ref{fig:pestanhaQR}. Su función es generar un código QR para que pueda ser leido por la aplicación del móvil. Hay que rellenar los campos de destino y el texto que queremos que firme dicha persona. Al darle a \textit{Generar código QR} se hace una llamada a la API Google Chart y se genera un código QR que contiene dichas cadenas y se muestra en la parte de la derecha, como se puede ver en la figura~\ref{fig:codigoQR}.

\begin{figure}
    \includegraphics[scale=0.5]{./GoogleAppEngine/imagenes/pestanhaQR.png}
  \caption{Pestaña para generar el código QR.}
  \label{fig:pestanhaQR}
\end{figure}

\begin{figure}

    \includegraphics[scale=0.5]{./GoogleAppEngine/imagenes/codigoQR.png}
  \caption{Pestaña con el código QR generado.}
  \label{fig:codigoQR}
\end{figure}

\end{description}



	\appendix
	%apendice con la configuración de eclipse tanto para Android como para Google App Engine
	\chapter[Configuración de Eclipse.]{Configuracion del entorno de programación Eclipse.}\label{cap:apendiceA}

A continuación vamos a explicar como instalar y configurar Eclipse para que podamos programar para Android y para Google App Engine, vamos a explicar como instalar los plugins que proporciona Google en ambos casos.

\section{Configuración de Eclipse para Android.}\label{cap:configuracionAndroidEclipse}

Lo primero que tenemos que hacer es descargarnos Eclipse y el SDK de Android. Para el primero vamos a la web \url{http://www.eclipse.org/downloads/} y bajamos la versión clasic de Eclipse. Yo recomiendo la versión Eclipse Classic porque es la versión básica que trae todo lo necesario para programar para Android. A continuación bajamos el SDK de Android de la siguiente web \url{http://developer.android.com/sdk/index.html}. Al haber realizado el proyecto en Ubuntu toda esta configuración será para Linux, pero es equivalente a como se debería de realizar en Windows o Mac ya que todo el software es multiplataforma, solo tendríamos que descargar las versiones específicas para nuestro sistema.

Una vez descargados ambos archivos procedemos descomprimirlos, con lo que obtenemos dos carpetas, una con el SDK de Android y otra con Eclipse. A continuación vamos a instalar el SDK de Android, para ello abrimos una terminal y nos desplazamos a la carpeta \textit{\\tools} dentro de la carpeta donde hemos descomprimido el SDK y ejecutamos el siguiente comando: 

\begin{lstlisting}[style=consola]
./android sdk
\end{lstlisting}

El resultado de ejecutar este comando podemos verlo en la figura~\ref[hola]{fig:androidSDK}

\begin{figure}
  \centering
    \includegraphics[scale=0.4]{./ConfiguracionEclipse/imagenes/androidSDK.png}
  \caption{Ventana de instalación del SDK de Android.}
  \label{fig:androidSDK}
\end{figure}

En la pantalla que aparece podemos elegir que versión del SDK queremos instalar dependiendo de la versión sobre la que queramos desarrollar, nosotros instalaremos la versión 4.0. Seleccionamos la versión 4.0 y pulsamos en instalar. Aceptamos las licencias y pulsamos en Install (figura~\ref{fig:licencias}), acto seguido el instalador empieza a descargar e instalar todos los paquetes que hemos seleccionado.

\begin{figure}
  \centering
    \includegraphics[scale=0.6]{./ConfiguracionEclipse/imagenes/licencias.png}
  \caption{Ventana para aceptar las licencias.}
  \label{fig:licencias}
\end{figure}

\begin{figure}
  \centering
    \includegraphics[scale=0.6]{./ConfiguracionEclipse/imagenes/SDKfinalizado.png}
  \caption{Instalación del SDK finalizada.}
  \label{fig:SDKfinalizado}
\end{figure}

Una vez hemos instalado el SDK (figura~\ref{fig:SDKfinalizado}) tenemos que instalar el puglins para Eclipse, para ello abrimos Eclipse y vamos a \textbf{Help -> Install New Software}, pulsamos en \textbf{Add} y copiamos la siguiente dirección web, \url{https://dl-ssl.google.com/android/eclipse/} y pulsamos \textbf{Ok}. En la figura~\ref{fig:instalacionADT} podemos ver el resultado y la opción que tenemos que marcar. Pulsamos \textbf{Siguiente} y aceptamos las licencias y pulsamos \textbf{Finalizar}, a continuación Eclipse empezará a descargar los paquetes que le hemos solicitado y cuando termina los instala.

\begin{figure}
  \centering
    \includegraphics[scale=0.6]{./ConfiguracionEclipse/imagenes/instalacionADT.png}
  \caption{Instalación del ADT en Eclipse.}
  \label{fig:instalacionADT}
\end{figure}

Una vez instalados nos pide que reiniciemos Eclipse, al iniciar de nuevo nos aparece un asistente para la configuración del SDK que vamos a usar en Eclipse, tenemos que saber la ruta donde hemos instalado anteriormente el SDK, podemos ver el asistente en la siguiente figura~\ref{fig:configuracionSDK}.
 
\begin{figure}
  \centering
    \includegraphics[scale=0.6]{./ConfiguracionEclipse/imagenes/configuracionSDK.png}
  \caption{Instalación del SDK en Eclipse.}
  \label{fig:configuracionSDK}
\end{figure}

Como podemos observar en la figura~\ref{fig:nuevoProyectoAndroid} ya tenemos configurado Eclipse para programar en Android.

\begin{figure}
  \centering
    \includegraphics[scale=0.6]{./ConfiguracionEclipse/imagenes/nuevoProyectoAndroid.png}
  \caption{Creación de un nuevo proyecto Android.}
  \label{fig:nuevoProyectoAndroid}
\end{figure}

Para el uso de Git hemos usado el plugins eGit, que se puede descargar gratuitamente de la web \url{http://www.eclipse.org/egit/}. En dicha web también podemos ver la forma de instalación y configuración, que es muy sencilla y solo se necesita la URL del servidor, el nombre de usuario y el password.

\section{Configuración de Eclipse para Google App Engine.}\label{cap:configuracionGAEEclipse}

Al igual que en el apéndice~\ref{cap:configuracionAndroidEclipse} necesitamos tener una versión de Eclipse Classic, yo recomiendo tener un Eclipse diferente para cada plugins que queramos utilizar para evitar incompatibilidades entre los diferentes plugins. Comenzamos como en la instalación del ADT para Eclipse, abriendo Eclipse y pulsamos en \textbf{Help -> Install New Software}, luego en \textbf{Add} y copiamos la siguiente dirección web, \url{http://dl.google.com/eclipse/plugin/3.7} y pulsamos \textbf{Ok}. En la figura~\ref{fig:instalacionGAE} podemos ver el resultado y los paquetes que nos da la posibilidad de instalar, a nosotros nos interesa dentro de \textbf{SDK}, la opción \textbf{Google App Engine Java SDK}, también necesitamos el \textbf{Google Plugin for Eclipse 3.7} y si queremos hacer diseño de la interfaz web de la aplicación podemos instalar \textbf{GWT Designer for GPE}, nosotros no lo usaremos en el proyecto. A continuación pulsamos \textbf{Siguiente}, aceptamos las licencias y pulsamos finalizar. Eclipse acto seguido empezará a descargar los puglins y a instalarlos (figura~\ref{fig:instalacionPlugins}), cuando termine nos pedirá que reiniciemos Eclipse.

\begin{figure}
  \centering
    \includegraphics[scale=0.6]{./ConfiguracionEclipse/imagenes/instalacionGAE.png}
  \caption{Instalación puglins Google App Engine.}
  \label{fig:instalacionGAE}
\end{figure}

\begin{figure}
  \centering
    \includegraphics[scale=0.6]{./ConfiguracionEclipse/imagenes/instalacionPlugins.png}
  \caption{Detalle de la descarga de plugins.}
  \label{fig:instalacionPlugins}
\end{figure}

Cuando se vuelve a abrir Eclipse ya vemos que hay varias cosas que han cambiado como podemos ver en la figura~\ref{fig:eclipseGAE}. Como podemos ver ya estamos logueados con nuestra cuenta de usuario y nos da la posibilidad de crear aplicaciones web en la plataforma. En el botón de Google también hay una acción muy importante que explicaremos posteriormente que es la de desplegar una aplicación web en el servicio de Google App Engine. Todo lo que hagamos podemos probarlo en local simplemente pulsado el botón play y Eclipse deplegará un servidor Tomcat en el que probar la aplicación web que desarrollemos.

\begin{figure}
  \centering
    \includegraphics[scale=0.6]{./ConfiguracionEclipse/imagenes/eclipseGAE.png}
  \caption{Eclipse con el puglin de Google App Engine instalado.}
  \label{fig:eclipseGAE}
\end{figure}

Una vez preparado Eclipse para que podamos programar, hay que genera los dominios donde correrán nuestras aplicaciones web. De forma gratuita Google proporciona 10 dominios diferentes por cuenta. Para crearlos hay que ir a la siguiente dirección web: \url{https://appengine.google.com/}, en ella nos logueamos con nuestra cuenta de Google, podemos ver en la figura~\ref{fig:appEngine} el panel de administración de las aplicaciones.  

\begin{figure}
  \centering
    \includegraphics[scale=0.6]{./ConfiguracionEclipse/imagenes/appEngine.png}
  \caption{Panel de administración Google App Engine.}
  \label{fig:appEngine}
\end{figure}

Pulsando el botón \textbf{Create Application} tenemos acceso a todos los parámetros de configuración para la aplicación web. Podemos ver todos los parámetros que nos pide en la imagen~\ref{fig:createGAE}. Nos pide un nombre que será el dominio que nos proporcionará Google, el título de la aplicación web y la forma de logueo que queremos que tenga nuestra aplicación en caso de necesitarla.

\begin{figure}
  \centering
    \includegraphics[scale=0.6]{./ConfiguracionEclipse/imagenes/createGAE.png}
  \caption{Parámetros para la creación de una nueva aplicación web.}
  \label{fig:createGAE}
\end{figure}

Una vez creada la aplicación web, tenemos acceso a un dashboard donde podemos configurar la aplicación, ver estadísticas, los log, administrar la base de datos, etc. Podemos verlo en la figura~\ref{fig:dashboardGAE}.

\begin{figure}
  \centering
    \includegraphics[scale=0.6]{./ConfiguracionEclipse/imagenes/dashboardGAE.png}
  \caption{Dashboard para la configuración de una aplicación web.}
  \label{fig:dashboardGAE}
\end{figure}

Con esto habríamos terminado la parte de la creación de la aplicación web y a continuación vamos a explicar como hacer el despliegue de una aplicación con Eclipse.

Una vez tenemos un proyecto creado y testeado procederemos a la subida, para eso tendríamos que pulsar en el botón de Google que vimos antes y luego pulsar en la opción \textbf{Deploy to App Engine}. Nos aparece una nueva ventana en la que debemos configurar unos parámetros básicos, pero si es la primera vez que vamos a hacer el despliegue tenemos pulsar en \textbf{App Engine project settings...} (figura~\ref{fig:GAESettings}), en ventana que nos aparece podemos configurar muchos más parámetros, el más importante el \textbf{Application ID} y \textbf{Version}. En \textbf{Application ID} tenemos que poner el nombre con el que hemos creado la aplicación web, en nuestro caso \textit{repositoriorecibos}. También podemos cambiar la versión del SDK que queremos usar, que según hemos podido observar durante el desarrollo del proyecto se ha actualizado varias veces, si la base de datos es altamente replicada, etc.

\begin{figure}
  \centering
    \includegraphics[scale=0.6]{./ConfiguracionEclipse/imagenes/GAESettings.png}
  \caption{Despliegue de una aplicación web.}
  \label{fig:GAESettings}
\end{figure}

Una vez realizado el despliegue ya podemos tener acceso a la aplicación web mediante el link \url{http://nombre\_aplicacion.appspot.com}.























	%apendice para la creación de los certificados usados con XCA
	\input{./CreacionCertificado/creacionCertificado.tex}
	
	\backmatter
		\listoffigures
	
\end{document}
