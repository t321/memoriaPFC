\documentclass[a4paper,12pt]{book}

\usepackage[utf8]{inputenc}
\usepackage[spanish]{babel}
%\usepackage{listings}
\usepackage{graphicx}
%\usepackage{amssymb}
\usepackage{hyperref}
\usepackage{eurosym}

\usepackage{color}
\definecolor{gray97}{gray}{.97}
\definecolor{gray75}{gray}{.75}
\definecolor{gray45}{gray}{.45}

\usepackage{listings}
\lstset{ frame=Ltb,
framerule=0pt,
aboveskip=0.5cm,
framextopmargin=3pt,
framexbottommargin=3pt,
framexleftmargin=0.4cm,
framesep=0pt,
rulesep=.4pt,
backgroundcolor=\color{gray97},
rulesepcolor=\color{black},
%
stringstyle=\ttfamily,
showstringspaces = false,
basicstyle=\small\ttfamily,
%\footnotesize\ttfamily,
%\small\ttfamily,
commentstyle=\color{gray45},
keywordstyle=\bfseries,
%
numbers=left,
numbersep=15pt,
numberstyle=\tiny,
numberfirstline = false,
breaklines=true,
tabsize=2,
}

% minimizar fragmentado de listados
\lstnewenvironment{listing}[1][]
{\lstset{#1}\pagebreak[0]}{\pagebreak[0]}

% estilo para poner comandos de consola
\lstdefinestyle{consola}
{basicstyle=\scriptsize\bf\ttfamily,
backgroundcolor=\color{gray75},
}

% estilo para código Java
\lstdefinestyle{Java}
{language=Java,
basicstyle=\footnotesize\ttfamily,
}

% estilo para código XML
\lstdefinestyle{XML}
{language=XML,
}

% estilo para código HTML
\lstdefinestyle{HTML}
{language=HTML,
}

%para los márgenes
%\usepackage{geometry} % See geometry.pdf to learn the layout options. There are lots.
%\geometry{a4paper}
%\marginsize{2.5cm}{2.5cm}{2.5cm}{2.5cm}

\usepackage[left=2.5cm,top=3.5cm,right=2.5cm,bottom=3cm]{geometry} 


\title{\Huge Plataforma de firma digital móvil en el contexto universitario.}
	
 	\author{Juan Antonio Pérez Ariza \\
		Escuela Técnica Superior Ingeniería Informática \\
		Universidad de Málaga \\
		\\
		Profesor: \\
		Isaac Agudo Ruíz\\
		Departamento Lenguajes y Ciencias de la Comunicación \\
		Universidad de Málaga}

\begin{document}
	\maketitle	
	
	%Table of contents	
	\tableofcontents
	
	%separacion entre parrafos
	\parskip=5mm
	
	\mainmatter
	%introducción
	\chapter{Introducción.}
En este primer capítulo de la memoria vamos a explicar las motivaciones que nos llevaron a pensar en realizar dicho proyecto, los objetivos que nos marcamos cuando lo diseñamos, los materiales usados y la organización de esta memoria.

\section{¿Por qué decidimos hacer el proyecto?}
Al igual que muchos estudiantes de la Universidad de Málaga, yo suelo comer habitualmente en la cafetería de la facultad y hay mucha personas que se molesta cuando le piden que firme el papel con el que se lleva el recuento de los estudiantes que comen en las cafeterías para el descuento por ser estudiante. Además de este inconveniente hay un par de problemas más, que son lo molesto que es tener que firmar todos los días o habitualmente y las colas que se forman al tener que rellenar el nombre y la firma, por eso se decidió hacer una aplicación para terminales móviles con la que agilizar todo este proceso de firma y control de estudiantes, mediante la lectura de un código QR que tendría la información necesaria para la realización de la firma.

A medida que avanzaba el proyecto se vio que se podía ampliar la funcionalidad de la aplicación, no solo al comedor, si no también a alquiler de pistas o cualquier documento necesario en la Universidad de Málaga.

Otra razón para la realización de este proyecto es que a mi me gusta la seguridad informática y vi en este proyecto una buena forma de aprender más sobre criptografía, particularmente en la criptografía de clave pública. También vi una buena forma de aprender a programar para terminales Android, debido al gran auge que tienen en este momento los terminales móviles, y a crear aplicaciones web de las que no tenía ninguna idea. Al inicio la aplicación web se pensó en hacer directamente en java sin ninguna ayuda, pero se descartó ante la dificultad de encontrar un servicio de hosting gratuito, por lo que se decidió cambiar a una sugerencia que hizo director de proyecto de usar una plataforma que proporciona Google llamada Google App Engine, que es gratuita y se pueden crear aplicaciones web programadas en el lenguaje de programación Java y así tener la posibilidad de aprender otras APIS, no solo Java2EE.

\section{Objetivos que queríamos conseguir.}

El principal objetivo que queríamos conseguir era que la forma de firmar fuera muy fácil y que no fuera un mecanismo muy engorroso. Para ellos decidimos realizar una aplicación para smartphone Android y una aplicación web para el almacenamiento y posterior comprobación de las firmas.

Una vez teníamos clara la idea, empezamos a diseñar un sistema con el que se pudiera realizar la firma digital inicialmente, solo de un recibo y finalmente cualquier documento de la UMA agilizando de esta forma dicho proceso.

%TODO: Hay poner algo de criptografía... xDDD

En la parte de la aplicación de Android se decidió hacer una aplicación clara y que fuese fácil de usar. Para eso usamos la API nivel 14 que equivale a la versión 4.0 de Android, llamada Ice Cream Sandwich. Se eligió porque proporciona una nueva forma de diseño de las interfaces, un nuevo tema  llamado Holo y proporciona muchas nuevas herramientas como por ejemplo son los ActionBar, que es una barra que permanece siempre en la parte superior de la pantalla, la que se va acomodando a las necesidades que tiene la aplicación cambiando los botones con diferentes funcionalidades, por ejemplo si estamos en la pantalla principal pues tendremos siempre visible el botón de añadir un nuevos recibo que abrirá el lector de códigos QR, se puede observar en la primera barra de la figura~\ref{fig:actionBar}, sin embargo si estamos visualizando un recibo solo tendremos el botón de volver atrás, como se puede ver en la segunda barra de la figura~\ref{fig:actionBar}, en la que se puede ver que al lado del icono de la aplicación una flechita que indica que es el botón para volver atrás.

\begin{figure}
  \centering
    \includegraphics[scale=0.3]{./Introduccion/imagenes/actionbar.png}
  \caption{Action Bar.}
  \label{fig:actionBar}
\end{figure}

Al tomar la decisión de programar para terminales con Android 4.0 o mayor estuvimos sopesando los pros y los contras, y al final decidimos que la implantación de Android 4.0 cada vez es mayor y que cada día hay más terminales con dicha versión, como se puede ver en este gráfico de la figura~\ref{fig:graficoEvolucionAndroid} y podemos observar que a finales de agosto de este año la cantidad de usuarios afectados sería de más del 15\% de terminales como podemos ver en la figura~\ref{fig:graficoUsoAndroid}, aunque todavía sigue reinando la versión 2.3.3, aunque creemos que el cambio a la versión 4.0 o superior será rápida debido a todas las ventajas que aporta y mucho más ahora que hace unos meses Google sacó una nueva versión, la 4.1, llamada Jelly Bean y casi todas las compañías querrán actualizar sus terminales a la última versión, por lo que a pesar de dejar a un gran número de usuarios sin poder usar la aplicación preferimos funcionalidad y elegancia frente a gran cantidad de usuarios, ya que estos llegarán a medida que sus compañías actualicen sus terminales.


\begin{figure}
  \centering
    \includegraphics[scale=0.8]{./Introduccion/imagenes/graficoEvolucionAndroid.png}
  \caption{Gráfico de las versiones de Android a principio de octubre del 2012.}
  \label{fig:graficoEvolucionAndroid}
\end{figure}

\begin{figure}
  \centering
    \includegraphics[scale=0.5]{./Introduccion/imagenes/graficoUsoAndroid.png}
  \caption{Gráfico del uso de las versiones de Android a principios de octubre del 2012.}
  \label{fig:graficoUsoAndroid}
\end{figure}

En la parte del servidor al elegir la plataforma de Google, hubo muchas cosas que resultaron más fáciles a costa de tener que aprender a usar el SDK que Google proporciona, que como era una de las cosas por la que elegimos dicha plataforma no nos importó. Una de las cosas que nos facilitaba es la gestión de usuarios, que los gestiona Google directamente al tener que usar una cuenta de Google Account para usar la aplicación web. Toda la seguridad y mantenimiento de los servidores, copias de seguridad, balanceos de carga y otra gran cantidad de funciones también las realiza Google, por lo que nos eximen de su realización y no tendremos que preocuparnos de ellas.

\section{Organización de la memoria.}

En el primer capítulo, que es el actual, hemos realizado una introducción al proyecto, explicado la idea inicial y los materiales usados. Para continuar en el capítulo segundo de la memoria explicaremos los conocimientos básicos sobre todas las tecnologías usadas en el proyecto, como pueden ser Android, Java, criptografía, HTML, GIT, etc. En el capítulo tercero de la memoria explicaremos todo el diseño e implementación realizados en el proyecto. En el cuarto capítulo comentaremos todos los conocimientos necesarios para entender la criptografía usada en el proyecto, así como una breve historia de la criptografía, desde la época romana hasta la actual. En el quinto capítulo vamos a documentar todos los conocimientos necesarios y que hemos tenido que usar para realizar la aplicación de Android. En el capítulo sexto contaremos los conocimientos básicos y los necesarios para realizar las aplicaciones web del proyecto. En el último capítulo tendremos el desenlace del proyecto en el que explicaremos el uso de las aplicaciones, expondremos las conclusiones, problemas y posibles trabajos futuros que hemos observado que se podrían realizar y la biografía usada. Para finalizar tendremos tres anexos en los que explicaremos la configuración de la plataforma Eclipse para el uso de la API Google App Engine como la de Android, la creación de los certificados de clave pública y el contenido del CD.  

\section{Material usado.}
Para la realización del proyecto hemos usado un ordenador personal para todo el proceso de programación y un smartphone con Android para la depuración y prueba de la aplicación. 

El ordenador es un ordenador portátil, con Ubuntu 12.04 como sistema operativo, un procesador Pentium Dual Core a 2.2Ghz, 4 Gb de memoria ram.

El móvil usado es un Samsung Galaxy Nexus (figura~\ref{fig:nexus}) que fue el móvil presentado cuando Google lanzó la versión 4.0, por lo que es el primer terminal en usar Android 4.0 y meses después el primero en recibir la actualización de Android 4.1. Sus característica son una pantalla de 4.65 pulgadas Super Amoled con resolución de 1280 x 720, procesador dual-core a 1.2Ghz, HSPA+, NFC, Wifi, GPS, etc.

\begin{figure}
  \centering
    \includegraphics[scale=0.3]{./Introduccion/imagenes/nexus.png}
  \caption{Samsung Galaxy Nexus.}
  \label{fig:nexus}
\end{figure}

No tenemos datos sobre las máquinas usadas por los servidores donde Google da el servicio de Google App Engine.


	%conocimientos previos sobre las tecnologías
	\chapter[Conocimientos previos]{Conocimientos previos sobre las tecnologías usadas.}\label{cap:conocimientos}
\markboth{CAPÍTULO \ref{cap:conocimientos}. CONOCIMIENTOS PREVIOS.}{}

En este segundo capítulo vamos a explicar cuáles son las ideas previas de las que partimos y los conocimientos que poseíamos sobre las tecnologías que hemos usado en el proyecto. Haciendo una breve explicación sobre su funcionamiento, su uso, su historia y sus recientes versiones.

El principal elemento usado en el proyecto es el lenguaje de programación Java, que lo hemos usado para programar tanto la aplicación web como la aplicación en el móvil. Pero aparte de los diferentes SDK de Android y de Google App Engine, hemos recurrido y empleado muchas otras tecnologías y lenguajes como pueden ser SQL, XML, UML, GIT. Además hemos aplicado los conocimientos básicos sobre criptografía de clave pública necesarios para realizar todo el proceso de firma digital.   

\section{El lenguaje de programación Java}

Java es un lenguaje de programación orientado a objetos que fue diseñado por James Gosling\footnote{ Para más información sobre James Gosling: \url{http://en.wikipedia.org/wiki/James\_Gosling}} para Sun Microsystems y que recientemente ha sido comprado por Oracle Corporation. Fue lanzado en 1995 y ha sido el centro de toda la plataforma Java de Sun Microsystems. Es un lenguaje con una sintaxis muy parecida a C o C++, pero con la gran ventaja de que el manejo de punteros y objetos es automático, al igual que la recogida de basura.

Java es un lenguaje en el que hay que compilar los códigos fuentes para crear unos archivos intermedios llamados bytecodes, los archivos *.class, que luego serán interpretados por la máquina virtual de Java (JVM). Esta dependerá de la arquitectura en la que se quiera ejecutar la aplicación Java. Gracias a esto se puede decir que Java es un lenguaje multiplataforma, lo que significa que un mismo código Java se puede ejecutar en Linux, en Windows, en Mac o cualquier otro sistema para el cual exista una máquina virtual, lo que en inglés se llama ``write once, run anywhere" (WORA). Además de esta importante ventaja Java es un lenguaje de propósito general, concurrente, basado en clases y orientado a objetos. Java es el segundo lenguaje de programación más popular de 2012, gracias a las aplicaciones web cliente-servidor que tienen tanto auge en estos momentos, como podemos ver en la figura \ref{fig:indicetiobe}.

\begin{figure}[h]
  \centering
    \includegraphics[scale=0.9]{./ConocimientosPrevios/imagenes/indiceTiobe.png}
  \caption{Índice Tiobe en septiembre del 2012. \url{http://www.tiobe.com/}}
  \label{fig:indicetiobe}
\end{figure} 

La implementación original y las referencias del compilador de Java, máquinas virtuales y las librerías de clases fueron desarrolladas por Sun en 1995, pero en el 2007 gracias a la contribución de la comunidad, Sun Microsystems cambió la licencia de todas las tecnologías Java a GNU General Public License, por lo que se abría la posibilidad de que se crearan versiones alternativas de compiladores bajo licencia GNU como por ejemplo GNU Compiler para Java o GNU Classpath.

En el proyecto la versión usada fue la versión \textbf{Java SE}.

\subsection{Historia}

Originalmente Java nació como un proyecto de James Gosling, Mike Sheridan y Patrick Naughton en 1991 y estaba diseñado para una televisión interactiva, pero era muy avanzado para lo que la industria televisiva de la época podía necesitar. En su origen fue llamado Oak, pero por problemas con el nombre, ya que era una marca registrada de otra empresa, lo cambiaron a Green y posteriormente ya lo renombraron al definitivo Java. Hay muchas teorías sobre por qué se llama Java, pero una de ellas es que había una cafetería llamada Java Coffe donde James, Mike and Patrick pasaron muchas horas consumiendo café.

La idea de Gosling era crear una máquina virtual donde funcionara un lenguaje de programación con la sintaxis y la estructura de C/C++ para que la curva de aprendizaje fuera muy suave para los programadores que en la época sabían C/C++.

Sun Microsystems lanzó Java 1.0 en 1995, con la principal característica de que una vez escrito un código fuente no había que modificarlo para que funcionara en las diferentes máquinas, lo que anteriormente hemos llamado con el acrónimo en inglés WORA (Write Once, Run Anywhere). Rápidamente todos los navegadores de la época empezaron a soportar applets Java en las páginas web, por lo que Java se volvió muy popular en la época. La nueva versión Java 2 fue lanzada en 1998-1999 y con ella llegaron las distinciones en diferentes plataformas, como por ejemplo Java2EE para aplicaciones corporativas o una versión ligera llamada Java2ME que estaba diseñada para funcionar en los diferentes teléfonos de la época, y el resto se agrupan en la versión Java2SE, que es la versión estándar.

En 1997, Sun Microsystems intentó formalizar Java mediante una norma ISO/IEC pero se retiró del proceso y dio todo el control a la comunidad. Sun ofrecia implementaciones gratuitas y generaba dinero vendiendo algunas licencias de productos como Java Enterprise System. Una cosa importante es que Sun distingue entre el SDK (Kit de desarrollo) y el JRE (Entorno de ejecución) en el que van incluidos los compiladores, debuggers, etc.

El 13 de noviembre del 2006, Sun lanzó Java gratis y como software libre, bajo la licencia GNU General Public License (GPL). El proceso finalizó el 8 de mayo del 2007.

En 2009-2010 Oracle Corporation compró Sun Microsystems por lo que Java actualmente pertenece a Oracle Corporation.

\subsection{Versiones}

\begin{itemize}

	\item \textbf{JDK 1.0} (23 de enero de 1996): Primer lanzamiento
	
	\item \textbf{JDK 1.1} (19 de febrero de 1997): Las primeras características añadidas fueron una reestructuración intensiva del modelo de eventos AWT (Abstract Windowing Toolkit), clases internas (inner classes), JavaBeans, JDBC (Java Database Connectivity), para la integración de bases de datos y RMI (Remote Method Invocation).
	
     \item \textbf{JDK 1.2}(8 de diciembre de 1998): Recibió el nombre en clave Playground. Esta y las siguientes versiones fueron recogidas bajo la denominación Java 2 y el nombre ``J2SE" (Java 2 Platform, Standard Edition), reemplazó a JDK para distinguir la plataforma base de J2EE (Java 2 Platform, Enterprise Edition) y J2ME (Java 2 Platform, Micro Edition). 
    Se añadieron las siguientes mejoras, la palabra reservada strictfp, reflexión en la programación, la API gráfica (Swing) fue integrada en las clases básicas, la máquina virtual (JVM) de Sun fue equipada con un compilador JIT (Just in Time) por primera vez, Java Plug-in, Java IDL, una implementación de IDL (Lenguaje de Descripción de Interfaz) para la interoperabilidad con CORBA y Colecciones.

    \item \textbf{J2SE 1.3} (8 de mayo de 2000): Recibió el nombre en clave Kestrel. Los cambios más notables fueron: la inclusión de la máquina virtual HotSpot JVM, RMI fue cambiado para que se basara en CORBA, JavaSound, se incluyó el Java Naming and Directory Interface (JNDI) en el paquete de bibliotecas principales (anteriormente disponible como una extensión), Java Platform Debugger Architecture (JPDA).

    \item \textbf{J2SE 1.4} (6 de febrero de 2002): Recibió el nombre en clave Merlin. Este fue el primer lanzamiento de la plataforma Java desarrollado bajo el Proceso de la Comunidad Java como JSR 59. Las principales características que se le añadieron fueron palabra reservada assert, expresiones regulares modeladas al estilo de las expresiones regulares Perl, encadenación de excepciones, non-blocking NIO (New Input/Output), logging API, API I/O para la lectura y escritura de imágenes en formatos como JPEG o PNG, parser XML integrado y procesador XSLT (JAXP), seguridad integrada y extensiones criptográficas (JCE, JSSE, JAAS), Java Web Start incluido.
    
    \item \textbf{J2SE 5.0} (30 de septiembre de 2004): Recibió el nombre en clave Tiger. Estos fueron los cambios más importantes, plantillas (genéricos), metadatos, también llamados anotaciones, permite a estructuras del lenguaje, como las clases o los métodos, ser etiquetados con datos adicionales que pueden ser procesados posteriormente por utilidades de proceso de metadatos, autoboxing/unboxing, conversiones automáticas entre tipos primitivos (Como los int) y clases de envoltura primitivas (Como Integer), enumeraciones, varargs (número de argumentos variable), el último parámetro de un método puede ser declarado con el nombre del tipo seguido por tres puntos (por ejemplo \lstinline{void drawtext(String... lines)}). En la llamada al método, puede usarse cualquier número de parámetros de ese tipo, que serán almacenados en un array para pasarlos al método, bucle for mejorado, la sintaxis para el bucle for se ha extendido con una sintaxis especial para iterar sobre cada miembro de un array o sobre cualquier clase que implemente Iterable, como la clase estándar Collection, de la siguiente forma:

\begin{lstlisting}[style=Java]
void displayWidgets (Iterable<Widget> widgets) {
	for (Widget w : widgets) {
		w.display();
	}
}
\end{lstlisting}

    \item \textbf{Java SE 6} (11 de diciembre de 2006): Recibió el nombre en clave Mustang. En esta versión, Sun cambió el nombre ``J2SE" por Java SE y eliminó el ``.0" del número de versión. Los cambios más importantes introducidos en esta versión son un nuevo marco de trabajo y APIS que hacían posible la combinación de Java con lenguajes dinámicos como PHP, Python, Ruby y JavaScript, el motor Rhino, de Mozilla, una implementación de Javascript en Java, un cliente completo de Servicios Web y soporta las últimas especificaciones para Servicios Web, mejoras en la interfaz gráfica y en el rendimiento.
    
    \item \textbf{Java SE 7} (Julio 2011): Su nombre en clave es Dolphin. Y las principales nuevas características son: soporte para XML dentro del propio lenguaje, un nuevo concepto de superpaquete, soporte para closures, e introducción de anotaciones estándar para detectar fallos en el software.

\end{itemize}

\section{El entorno de programación Eclipse.}

Eclipse es un entorno integral de desarrollo que consta de un entorno de desarrollo integrado (IDE) y es extensible mediante plugins que están escritos en Java. Puede ser usado para una larga lista de lenguajes de programación como pueden ser C, C++, Haskell, Perl, PHP, Python, Android y un largo etcétera. Fue originalmente desarrollado por IBM y fue lanzado con la licencia de software Eclipse Public License\footnote{ Para más información visite: \url{http://en.wikipedia.org/wiki/Eclipse\_Public\_License}} la cual es una licencia de software libre. El SDK de Eclipse es libre y tiene licencia Open Source por lo que cualquier persona con los conocimientos necesarios puede programar el plugin que necesite para Eclipse. Fue el primer entorno de programación que funcionó bajo GNU Classpath y que funcionaba sin problemas con IcedTea. En la figura \ref{fig:pantallaEclipse} se puede ver el aspecto que tiene.

\begin{figure}
  \centering
    \includegraphics[scale=0.5]{./ConocimientosPrevios/imagenes/pantallaEclipse.png}
  \caption{Eclipse 4.2 Juno.}
  \label{fig:pantallaEclipse}
\end{figure} 

En el proyecto hemos usado la versión \textbf{Indigo}, que equivale a la versión 3.7 de Eclipse.

\subsection{Historia}

Eclipse comenzó como un proyecto de IBM Canadá. En noviembre de 2001 se creó un grupo de empresas para promover el desarrollo de Eclipse como software libre, los miembros iniciales eran Borland, IBM, Merant, QNX Software Systems, Rational Software, Red Hat, SuSE, TogetherSoft and WebGain. Finalmente en enero de 2004 se creó la Eclipse Foundation. 

\subsection{Versiones}

\begin{itemize}

	\item \textbf{Versión 3.0} (21 de junio de 2004)
	
	\item \textbf{Versión 3.1} (28 de junio de 2005)
	
	\item \textbf{Versión 3.2} (30 de junio de 2006): recibió el nombre de Callisto.
	
	\item \textbf{Versión 3.3} (29 de junio de 2007): recibió el nombre de Europa.
	
	\item \textbf{Versión 3.4} (25 de junio de 2008): recibió el nombre de Ganymede.
	
	\item \textbf{Versión 3.5} (24 de junio de 2009): recibió el nombre de Galileo.
	
	\item \textbf{Versión 3.6} (23 de junio de 2010): recibió el nombre de Helios.
	
	\item \textbf{Versión 3.7} (22 de junio de 2011): recibió el nombre de Indigo.
	
	\item \textbf{Versión 4.2} (27 de junio de 2012): recibió el nombre de Juno.
	
	\item \textbf{Versión 4.3} (26 de junio de 2013): esta será la próxima versión, que saldrá el próximo año y recibirá el nombre de Kepler.

\end{itemize}

\section{Criptografía.}\label{lbl:criptografia}

La criptografía es la ciencia que se encarga del estudio y creación de técnicas para la protección de una comunicación, para que solamente los usuarios autorizados puedan verla, leerla y entenderla. En la actualidad la criptografía es un término que se usa de forma similar a encriptación, que es el proceso para transformar una información mediante diferentes algoritmos, en un mensaje que no pueda entender un atacante que intercepte una comunicación. 

En el proyecto hemos usado una criptografía llamada \textbf{Criptografía de Clave Pública}, que como veremos a continuación en la historia brevemente y posteriormente en el capítulo~\ref{cap:criptografia}, en el que se explicará la criptografía usada a lo largo del proyecto con mas profundidad, consta de dos claves, ambas enlazadas matemáticamente y si conocemos una no podremos de ninguna forma conseguir la otra. La pública es la que tendría la persona que quiera desencriptar el mensaje, que a su vez da nombre a este algoritmo y otra privada que sólo conocerá la persona que quiere encriptar el mensaje.

La criptografía ha evolucionado mucho y actualmente no solo se usan para proteger mensajes, si no que también se usa para proteger la integridad de ellos. Este es uno de los usos más común de la criptografía de clave pública.

\subsection{Historia}

Podemos hacer dos grandes grupos dentro de la historia de la criptografía, la criptografía clásica y la criptografía durante la época de los ordenadores.

\subsection{Criptografía clásica.}

	Durante la época de la criptografía clásica sólo se quería proteger el mensaje que se enviaba de la mirada de curiosos y enemigos por lo que únicamente existían algoritmos de encriptación, la integridad del mensaje no importaba en esa época.  
	
	En dicha época todos los algoritmos de cifrados que existían eran por transposición o sustitución de caracteres. A continuación exponer unos ejemplos de los algoritmos utilizados más famosos. 
\begin{itemize}

	\item \textbf{Cifrado Cesar:} dicho cifrado es famoso porque los usaban las centurias romanas para comunicarse entre ellas de manera que si un mensaje era interceptado no pudiera ser leído. Consiste en sustituir cada carácter del mensaje por el que hay tres lugares a la derecha. Por ejemplo si tenemos el mensaje ``Hola" si lo ciframos con este sistema conseguimos ``Krod", en la figura \ref{fig:cifradoCesar} podemos ver como es el cifrado.
	
	Para desencriptar solo habría que intercambiar por la tercera letra anterior.

\begin{figure}[h]
  \centering
    \includegraphics[scale=0.6]{./ConocimientosPrevios/imagenes/cifradoCesar.png}
  \caption{Ejemplo Cifrado Cesar}
  \label{fig:cifradoCesar}
\end{figure} 

	\item \textbf{Cifrado Homofónico:} Es una evolución del siguiente, pero en vez de sustituir siempre por el mismo carácter lo que se hace es tener la posibilidad de poder realizar varios cambios posibles, por lo que un mismo mensaje podría generar varios textos cifrados, complicando así su desencriptación. En la figura \ref{fig:cifradoHomofonico} podemos ver una tabla sencilla de sustitución para realizar el cifrado. Por ejemplo si ciframos la palabra ``PLATON" nos daría de resultado ``882110772963", pero podríamos sustituir la P no solo por 88 si no por cualquier valor de la tabla dando lugar a que pudiéramos crear varios mensajes cifrados.
	
\begin{figure}[h]
  \centering
    \includegraphics[scale=0.7]{./ConocimientosPrevios/imagenes/cifradoHomofonico.png}
  \caption{Tabla para cifrado homofónico}
  \label{fig:cifradoHomofonico}
\end{figure}

	\item \textbf{Cifrado por Transposición:} consiste en realizar una permutación de las posiciones que ocupan las letras escritas, un ejemplo podría ser escribir todo el texto con una cierta longitud preestablecida y luego leerlo por columnas en vez de por filas. En la figura \ref{fig:cifradoTransposicion} podemos ver un ejemplo del mecanismo de cifrado.  

\begin{figure}[h]
  \centering
    \includegraphics[scale=0.4]{./ConocimientosPrevios/imagenes/cifradoTransposicion.png}
  \caption{Ejemplo de cifrado por sustitución}
  \label{fig:cifradoTransposicion}
\end{figure}	

	\item \textbf{Cifrado Producto:} Es un cifrado que combina sustitución y transposición y se puede considerar como un encadenamiento de varios cifrados. Esto da lugar a cifrados complejos, seguros y difíciles de atacar, ya que tendríamos que averiguar no sólo el método de cifrado utilizado, sino que también tendríamos que saber el orden en el que se ha ejecutado las encriptaciones.

	\item \textbf{Cifrado Vernam:} es un tipo de cifrado que se denomina cifrado de flujo. El texto en claro se combina con una cadena, del mismo tamaño del texto en claro, de número aleatorios o pseudoaleatorio por medio de la función XOR. Lo inventó Gilbert Vernam que era un ingeniero de AT\&T en 1917. Es también conocido como RC4 en internet.  
	
\end{itemize}

\subsection{Criptografía durante la época de los ordenadores.}

La criptografía dio un gran salto en cuanto a calidad en el momento en el que se empezaron a usar ordenadores para encriptar y desencriptar textos, debido a que los ordenadores son máquinas que las tareas repetitivas las hacen muy bien y muy rápidos.

Se empezaron a idear nuevos algoritmos de cifrado mucho más complejos, los cuales se pueden dividir en dos grandes grupos, la criptografía de clave simétrica y la criptografía de clave pública. A continuación vamos a explicar brevemente los algoritmos más famosos de ambos.

\subsubsection*{Criptografía de Clave simétrica}

La principal característica de esta técnica de criptografía es que usan la misma clave para encriptar y desencriptar.

\begin{itemize}

	\item \textbf{Data Encryption Standard (DES):} Fue presentado por IBM en 1974, para generar un estandar de cifrado para transmisión de datos y cifrado de almacenamiento de datos y que fuera usado por gobiernos, empresas privadas o cualquier usuario. IBM comenzó el desarrollo basándose en un dispositivo de cifrado llamado Lucifer el cual tenia una clave de 128 bits. DES es un criptosistema de clave secreta que cifra en bloques de 64 bits del texto en claro y genera otros bloques de 64 bits del texto cifrado. La clave utilizada también es de 64 bits, pero el bit final de cada octeto de los 64 bits de la clave se usa como bit de paridad para control de errores. El cifrado se realiza en 16 iteraciones en las que se usan varias operaciones como son operaciones XOR, permutaciones y sustituciones. El esquema para cifrar se puede ver en la figura \ref{fig:cifradoDes}.
	
\begin{figure}
  \centering
    \includegraphics{./ConocimientosPrevios/imagenes/cifradoDes.png}
  \caption{Iteraciones en el cifrado DES}
  \label{fig:cifradoDes}
\end{figure}

\item \textbf{AES (Rijndael):} Fue presentado al concurso AES el 2 de enero de 1997 y anunciado ganador en 2001. Fue diseñado por dos criptólogos llamados Joan Daemen y Vincent Rijmen, ambos estudiantes de la Katholieke Universiteit Leuven de Bélgica. Al contrario que DES, AES es una red de sustituciones y permutaciones no una red de Feistel, se transformó en estándar efectivo el 26 de mayo de 2002 y en la actualidad es uno de los algoritmos de encriptación más famosos. Opera con bloques de 128 bits y tiene claves de 128, 192 y 256 bits.

\end{itemize}

El mayor problema que tiene este tipo de criptografía es que para que el destinatario pueda leer el mensaje necesita saber la clave y el intercambio de clave puede ser una dificultad muy grande si ambos usuarios no se pueden comunicar directamente, ya que usando cualquier otro método la clave podría ser interceptada y todo el proceso de encriptación sería inutil.

\subsubsection*{Criptografía de Clave Pública}

La criptografía de clave pública fue inventada por Diffie y Hellman y paralelamente por Merkle y ambos grupos aportaron a la criptografía el concepto de la utilización de pares de claves.

La característica principal es que cada usuario posee dos claves, una privada que sólo conoce el dueño de la clave y será usada para descifrar todo lo que otros usuarios cifren con su otra clave y otra la clave pública que es conocida por el resto de usuarios y será la que estos usarán para encriptar el mensaje que queremos que sea secreto. Así de esta forma si una persona quiere comunicarse con otra de forma secreta sólo tiene que conocer su clave pública, cifrar con ella y el destinatario podrá descifrar el mensaje con su clave privada. 

Otra característica es que las claves son imposibles de deducir una a partir de la otra, ambas claves son de una gran longitud y son generadas mediante exponenciación y/o productos de números primos grandes.

En los primeros años de existencia de la criptografía de clave pública se inventaron tres sistemas, Algoritmo de la mochila de Merkle-Hellman que fue roto, el esquema de McEliece que está considerado imposible de llevar a la práctica y un tercero que es el que explicaremos a continuación llamado RSA cuyo uso de ha impuesto actualmente.

\begin{itemize}

	\item \textbf{RSA:} Su nombre proviene de sus creadores que son Rivest, Shamir y Adleman y se basaron en la idea: \textit{``es muy fácil multiplicar dos números enteros primos grandes, pero extremadamente difícil hallar la factorización del producto"}, cuando inventaron el RSA en 1997. Es un algoritmo exponencial. Una característica de RSA es que tanto el mensaje, como el texto cifrado tienen que ser un código decimal, por lo que se tendría que usar el valor ASCII de la letra por ejemplo. Un ejemplo de uso sería el siguiente, lo primero que se debe de hacer antes de enviar el mensaje es acordar el algoritmo que se va a usar, lo siguiente el emisor cifra el mensaje usando la clave pública del receptor y se lo envía. Acto seguido el receptor descifra el mensaje que ha enviado el receptor usando su propia clave privada. La gran ventaja de este método es que en ningún momento la clave privada se tiene que enviar, por lo que solucionamos el gran problema que dijimos que tenían los algoritmos de cifrado simétrico, que antes de nada había que intercambiar la clave con la vulnerabilidad que eso implicaba. 

\end{itemize}

El algoritmo RSA será explicado con más profundidad en el capítulo~\ref{cap:criptografia}.

Los algoritmos de clave pública tienen un gran problema, es que son muy lentos realizando el proceso de cifrado y descifrado, por lo que en la situaciones reales se usan para realizar el intercambio de claves de algoritmos de clave simétrica que son mucho más rápidos y también igual de seguros, de esta forma solucionamos su principal problema.

\section{Android.}

Android es un sistema operativo basado en Linux especialmente diseñado para smartphone, tablet, smart TV y una infinidad de dispositivos, desarrollado por Google con Open Handset Alliance. Android empezó siendo desarrollado por la compañía llamada Android que inicialemente fue financiada y después comprada por Google en 2005. En 2007 cuando se presentó por primera vez Android también se anunció la fundación de Open Handset Alliance que es un conjunto de 86 empresas, entre las que hay compañías de hardware, software y telecomunicaciones, interesadas en el mundo de los dispositivos móviles. Android es código abierto y está distribuido bajo licencia Apache\footnote{ Para saber más sobre la licencia visite: \url{http://en.wikipedia.org/wiki/Apache\_License}}. La tarea del mantenimiento y desarrollo de Android es de Android Open Source Project (AOSP).

Android tiene una gran comunidad de desarrolladores que pueden extender las funcionalidad de los teléfonos o de cualquier dispositivos que pueda ejecutar la máquina virtual de Android, se puede desarrollar tanto en Java usando el SDK o en C++ usando el NDK, posee una tienda online llamada Google Play (anteriormente Android Market), donde se pueden comprar aplicaciones, películas, libros o música y en la que cualquier desarrollador por una pequeña cantidad de dinero (alrededor de 25\euro, por una cuenta vitalicia de desarrollador) puede subir todas las aplicaciones gratuitas o de pago que desee. En Junio de 2012 había alrededor de 600.000 aplicaciones en Google Play.

En el primer cuatrimestre de 2012, Android tenía el 59\% del mercado de smartphones en el mundo, de ahí la importancia de esta plataforma para los desarrolladores, ya que proporciona un mercado muy amplio y una forma muy fácil y barata de conseguir un gran número de usuarios.

Los detalles técnicos de Android se explicarán con más profundidad en el capítulo~\ref{cap:android}.

\subsection{Historia.}

Como hemos dicho anteriormente Android fue diseñado y creado originalmente por una compañía llamada Android que fue fundada en Palo Alto, California en 2003 por Andy Rubin, Rich Miner, Nick Sears y Chris White. Originalmente solo estaba diseñado para funcionar con smartphones, ya que ellos pensaban que un smartphone era algo más que un dispositivo que sirviera para usar el GPS y tener preferencias. 

Google compró Android el 17 de agosto de 2005, con la intención de entrar en el mercado de los teléfonos móviles. Después de varios años de rumores el 5 de noviembre de 2007, Google presentó la Open Handset Alliance, un grupo de empresas que incluian a Broadcom Corporation, Google, HTC, Intel, LG, Marvell Technology Group, Motorola, Nvidia, Qualcomm, Samsung Electronics, Sprint Nextel, T-Mobile and Texas Instruments entre otras muchas empresas que estaban interesadas en generar estándares para dispositivos móviles. Ese mismo día también se lanzó el primer producto Android basado en el kernell de Linux 2.6.

Android ha sido muy criticado por la gran fragmentación que tiene debido al gran número de versiones que posee, que son compatibles hacia versiones abajo pero no hacia versiones posteriores, esto quiere decir que la versión 4.0 es compatible con todo el software que funcionase con las versiones 2.0, 2.3 o cualquiera inferior, pero no será compatibles con el software diseñado para la versión 4.1. Este problema hace necesario que los fabricantes actualicen el software de sus teléfono, lo que es un gran problema debido a que muchos no lo hacen, hubo un pequeño intento de solucionar esta problemática haciendo que los fabricantes estuvieran obligados a actualizar sus terminales al menos en los 18 meses posteriores a la salida al mercado, pero no hubo ningún acuerdo.   

\subsection{Versiones.}

Como curiosidad todas las versiones de Android se denominan con un nombre en clave que es un postre.
\begin{itemize}

	\item \textbf{1.0 (Apple Pie):} primera versión lanzada el 23 de septiembre del 2008.
	
	\item \textbf{1.1 (Banana Bread):} lanzada el 9 de febrero del 2009.
	
	\item \textbf{1.5 (Cupcake):} fue presentada el 30 de abril del 2009, esta fue la primera versión con la que Android empezó a despuntar y entrar en el mundo de los teléfonos móviles, anteriormente apenas si era conocido. Tenía características nuevas muy interesantes como poder grabar y reproducir vídeo, podía subir videos a Youtube e imágenes a Picasa directamente desde el teléfono, un nuevo teclado predictivo, nuevos widget y carpetas para colocar en la pantalla de inicio y transiciones animadas.

	\item \textbf{1.6 (Donut):} fue presentada el 15 de septiembre de 2009. Se le añadieron las siguientes características nuevas como una interfaz integrada para la cámara, la grabadora de vídeo y la galería, se actualizó la búsqueda por voz añadiendo soporte a más aplicaciones nativas y la posibilidad de llamar a contactos, se añadió un buscador general en la pantalla de inicio donde se podía buscar contactos, historiales y páginas web, se añadió un nuevo framework de gestos y las herramientas de desarrollo llamado GestureBuilder.

	\item \textbf{2.0 / 2.1 (Eclair):} la versión 2.0 fue presentada el 26 de octubre de 2009 y la 2.1 fue liberada el 3 de diciembre del 2009. Se añadieron un gran número de mejoras, se optimizó la velocidad de hardware, se soportaron más tamaño de pantallas y resoluciones, se rediseñó la interfaz de usuario, el navegador también fue renovado y se le añadió soporte para HTML5, nueva lista de contactos, se añadió soporte para el flash de la cámara, zoom digital, soporte para bluetooth 2.1, se mejoraron la captura de eventos multi-touch con MotionEvent y fondos de pantalla animados.
	
	\item \textbf{2.2 (Froyo):} fue lanzada el 20 de mayo de 2010. Se optimizó el sistema Android, la memoria y el rendimiento, se mejoró la velocidad de las aplicaciones gracias a la implementación de JIT, se implementó el motor JavaScript V8 de Google Chrome en el navegador del móvil, nueva funcionalidad de WiFi hotspot y tethering por USB, se actualizó el Android market para que tuviera actualizaciones automáticas, marcación por voz y compartir contactos por Bluetooth, soporte para contraseñas numéricas y alfanuméricas, soporte para Adobe Flash 10 y soporte para pantallas de HDPI, como pueden ser pantallas de 4" y resolución de 720p. 

	\item \textbf{2.3 (Gingerbread):} fue presentado el 6 de diciembre del 2010. Cambiaron el diseño de la interfaz de usuario, añadieron soporte para pantallas extra grandes y resoluciones WXGA, soporte nativo para VoIP SIP, reproducción nativa de vídeos WebM/VP8 un formato de vídeo patrocinado por Google que es la alternativa al H264 en la reproducción de vídeo en HTML5 y decodificación de audios en AAC, se añadió soporte a NFC (Near Field Communication), nuevo teclado multitáctil, soporte mejorado para programar en código nativo, soporte nativo de más sensores como pueden ser acelerómetros o barómetros, soporte para múltiples cámaras y cambio del sistema de archivos YAFFS a ext4. La versión 2.3.3 sigue siendo la versión de Android más usada actualmente. 

	\item \textbf{3.0 / 3.1 / 3.2 (Honeycomb):} Esta versión fue diseñada exclusivamente para tablet, por lo que no hubo smartphones que actualizaran a esta versión. Las características principales fueron un escritorio en 3D con widget rediseñado, sistema multitarea mejorado, mejoras en el navegador de internet, videochat mediante Google Talk, mejoras en el soporte de redes WiFi, añadidos soporte para gran cantidad de periféricos y conexión USB.
	
	\item \textbf{4.0 (Ice Cream Sandwich):} Fue una de las actualizaciones más importantes que ha recibido Android y fue lanzada el 19 de octubre de 2011, en ella se unificaron todas las versiones y se tenía una sola versión para smartphne, televisores, tablets, netbooks, etc. Se añadió una nueva versión de interfaz mucho más limpia y usable llamada Holo, una nueva fuente llamada Roboto, se da la opción de utilizar botones virtuales en la interfaz de usuario en vez de botones físicos, soporte para aceleración gráfica por hardware, por lo que la interfaz es manejada y dibujada por la GPU, aumentando notablemente el rendimiento, multitarea mejorada, se ha añadido un nuevo corrector ortográfico, en la lista de notificaciones se pueden eliminar las que no sean interesantes, capturas de pantalla pulsando el botón de encendido y el de bajar volumen, mejorada la aplicación encargada de hacer fotografías, añadida una nueva opción para crear fotos panorámicas, Android Beam, una nueva característica que nos permite compartir contenidos entre teléfonos mediante NFC, reconocimiento de voz del usuario, reconocimiento facial, para bloqueo y desbloqueo del teléfono, añadidas nuevas carpetas que se crean sólo con arrastrar y soltar, un único y nuevo framework para crear aplicaciones y soporte para contenedor MKV.

	\item \textbf{4.1 (Jelly Bean):} Esta es la última versión de Android que hay en el mercado, fue lanzada el 27 de junio de 2012 durantes la última Google I/O. Se mejoró la fluidez y la estabilidad gracias al proyecto ``Project Butter", ajuste automático de widget cuando se añaden al escritorio, se añadió soporte para lenguas no occidentales, mejora de Android Beam para poder enviar video por NFC, dictado de voz mejorada y sin tener que tener conexión a internet para usarlo, nuevas notificaciones en las que se puede añadir botones para controlar o tener acceso a opciones más comunes, como puede ser responder a un email, pulsar pause o pasar de canción, nueva función Google Now que intenta ser el competidor de SIRI del iPhone en Android, cifrado de aplicaciones y nuevas actualizaciones incrementales, en las que no es necesario volver a bajar toda la aplicación para actualizarla, sólo se baja las partes nuevas, Google Chrome se convierte en el navegador por defecto de Android y se pone fin al soporte de Adoble Flash Player, se añade una nueva función llamada Sound Search que permite identificar la canción que está sonando, se ha añadido una nueva función llamada Gestual Mode para personas discapacitadas visualmente.
\end{itemize}

En el proyecto se ha usado la versión \textbf{Android 4.0} para el desarrollo.

En la figura \ref{fig:Android41} se puede ver como es visualemte Android en la actualidad, con la versión 4.1.

\begin{figure}
  \centering
    \includegraphics[scale=0.2]{./ConocimientosPrevios/imagenes/android41.jpeg}
  \caption{Versión de Android 4.1}
  \label{fig:Android41}
\end{figure}


\section{Google App Engine.}

Google App Engine es una plataforma de cloud computing para desarrollar y almacenar aplicaciones web que ofrece Google. Las aplicaciones web pueden ser escalables y si necesitan más recursos automáticamente le son asignados para poder seguir ofreciendo servicio. Google App Engine es gratis para un cierto número de peticiones y almacenamiento y la primera versión fue lanzada en abril de 2008.

Actualmente se puede desarrollar en tres lenguajes, que son Java, Python y Go, este último un lenguaje creado por Google. En el proyecto hemos usado Java para desarrollar en la plataforma. 

Google App Engine para Java soporta muchos estándares y framework, y el Core está hecho con la tecnología Servlet 2.5 usando el servidor de software libre llamado Jetty Web Server, acompañado con otras tecnologías como JSP. El almacén de datos puede ser muy poco intuitivo desde el punto de vista de los desarrolladores, pero se puede acceder fácilmente con JPA (Java Persistence API) y los métodos JDO (Java Data Objects) para escritura y lectura de datos. También se pueden usar tecnologías como Spring Framework.

Google garantiza que aplicación estará disponible el 99.95\% del tiempo, para ello ofrece una alta replicación.

La base de datos a la que nos dan acceso no es una base de datos estándar, como pueden ser MySQL, Oracle o SQLServer, pero tiene una sintaxis muy parecida a SQL, llamada GQL. Una de las principales diferencias es que no admite sentencias join, debido a la ineficiencia de la misma. La versión de Java soporta consultas asíncronas no bloqueantes, ofreciendo así una forma de procesamiento paralelo de datos.

Para más información se puede visitar la web diseñada para desarrolladores que proporciona Google, \url{https://developers.google.com/appengine/}. 

En el capítulo \ref{cap:GAE} veremos más ampliamente todo lo relacionado con Google App Engine que hemos usado en el proyecto.

\section{SQLite.}

SQLite es un sistema gestión de base de datos relacionales compatibles con ACID, ACID es el acrónimo de Atomicity, Consistency, Isolation and Durability, que son las características que debe tener una base de datos para que se consideren base de datos relacional. Su característica principal es que ocupa muy poco espacio, alrededor de 275 Kb y fue escrita en el lenguaje C por Richard Hipp. Está distribuida bajo licencia de dominio público.

A diferencia de los sistemas de gestión de base de datos clientes-servidor, el motor de SQLite pasa a ser parte del programa que quiere usarlo, ya que se integra con él. Esto hace que tenga mayor rendimiento debido a que la comunicación es por medio de funciones, que es mucho más eficiente que mediante comunicación de procesos. La totalidad de la base de datos, tablas, índices y datos, se guardan en un solo fichero estándar en la máquina host. La versión 3 de SQLite permite base de datos de hasta 2 Terabytes y permite campos del tipo BLOB.

SQLite está muy extendido y se puede programar en infinidad de lenguajes de programación como pueden ser C, C++, Perl, Python, PHP, Java, etc.

Es utilizado en infinidad de programas y sistemas, que van desde editores de imagen como puede ser Adobe Photoshop Elements, reproductores de sonido como Clementine o navegadores como Firefox, Chrome u Opera.

Esta es una de las tres formas que proporciona Android para guardar datos, al estar embebida en cada aplicación para mejorar el rendimiento, cada aplicación debe de tener una. 

\section{XML.}

XML son las siglas en inglés de e\textbf{X}tensible \textbf{M}arkup \textbf{L}anguage que es un lenguaje de marcas desarrollado por el World Wide Web Consortium (W3C), deriva del lenguaje SGML y permite definir la gramática de lenguajes específicos para estructurar documentos grandes.

Es una de las formas de intercambio estructurado de información más extendidas en internet, ya que se puede usar en base de datos, hojas de cálculo o en casi cualquier información que se quiera usar.

XML es un lenguaje que puede ser analizado sintácticamente para averiguar si está bien construido o no, por lo que cualquier parser (analizador sintáctico) puede confirmar si tiene la estructura bien definida según el estándar. Todos los documentos tiene que tener las siguientes partes: prólogo, cuerpo, elementos y atributos.

Un ejemplo podemos verlo en el siguiente un trozo de código XML, para almacenar un libro en una librería.

\begin{lstlisting}[language=XML]   
<?xml version="1.0"?>
<libro>
<titulo> A Game of Thrones </titulo>
<disponible tiempo="24" unidad="horas"/>
<autor> George R. R. Martin </autor>
<formato> Rustica </formato>
<publicacion> 1996 </publicacion>
<precio cantidad="9.99" moneda="euro"/>
<descuento cantidad="5"/>
<enlacelibro href="/exec/ISBN/0-553-10354-7"/>
</libro>
\end{lstlisting}

En el proyecto hemos usado XML, en los archivos de configuración o para el diseño de las interfaces en Android. En una de la aplicaciones web realizadas para los archivos de configuración también hemos usado XML, en la otra hemos usado un lenguaje parecido llamado YALM, que es equivalente a XML.

\section{UML.}

UML es un lenguaje de modelado de propósito general más usado en la actualidad para el diseño de software. UML son las siglas de Unified Modeling Language. UML tiene la ventaja de que se puede observar visualmente el diseño del software. Se puede desde especificar, construir o documentar un sistema o un software.

\begin{figure}
  \centering
    \includegraphics[scale=0.4]{./ConocimientosPrevios/imagenes/UMLDiagrams.jpg}
  \caption{Ejemplos de diseños realizados con UML.}
  \label{fig:UMLDiagrams}
\end{figure}

En la figura~\ref{fig:UMLDiagrams} podemos ver los diferentes diagramas que podemos realizar mediantes UML.

Hemos usado UML para diseñar las clases usadas a lo largo de todo el proyecto, además de para mostrar en la memoria del proyecto la relación de las clases, los casos de uso, etc.

\section{GIT.}

Git es un software de control de versiones, a la vez que mantenedor de la coherencia y cohesión del código fuente orientado a la velocidad. Fue desarrollado para el manejo del código fuente de Linux y al principio fue diseñado por Linus Torvalds. GIT es un repositorio con un completo historial y una capacidad de identificación de cada cambio realizado sin depender del acceso a la red o a un servidor central. Está liberado con licencia GNU versión 2.

En el proyecto hemos estado usando GIT como control de versiones, ya que si en algún cambio ocurría algún problema poder volver atrás y para tener una copia de seguridad del proyecto, alojado en todo momento en un servidor externo por si ocurría algún problema en el ordenador donde desarrollamos el proyecto.

A lo largo del proyecto hemos usado dos servicios gratuitos  que proporcionan servidores GIT, como pueden ser \url{https://github.com/} y \url{https://bitbucket.org/}. El primero es muy conocido actualmente y hay una gran comunidad de software libre en dicha plataforma, tiene un pequeño problema para nuestro proyecto y es que el código tiene que ser libre y en la versión gratuita no se pueden tener repositorios privados, cosa que el segundo servicio si te lo ofrece. Para todo el código fuente usando, tanto en la aplicación de Android como en las dos aplicaciones hemos usado bitbucket y para la memoria de proyecto hemos usado github. La dirección del repositorio de la memoria es la siguiente: \url{https://github.com/t321/memoriaPFC}, el repositorio de las aplicaciones web \url{https://bitbucket.org/t321/pfcaeg} y el de la aplicación de Android \url{https://bitbucket.org/t321/pfcandroid}.

La configuración básica para el uso de GIT con Eclipse se puede ver en el apendice~\ref{cap:apendiceA}.  















	%criptografía usada en el proyecto
	\chapter{Criptografía usada en el proyecto.}

En la sección \ref{lbl:criptografia} del capítulo \ref{cap:conocimientos} hemos hecho una breve introducción a la criptografía de clave pública, a continuación vamos a explicarla con más extensión y los usos que le hemos dado en el proyecto.

Toda la criptografía que hemos usado es criptografía de clave pública, y en concreto el criptosistema llamado RSA. Se pensó en usar DSA pero finalmente se usó RSA. En el proyecto se ha usado la criptografía no para cifrar nada, si no para firmar digitalmente.

\section{RSA.}

RSA se basa en la idea de que múltiplicar dos números enteros primos, pero muy dificil sabiendo el resultado de la multiplicación averiguar dichos dos números.

El proceso para generar las claves es el siguiente:
\begin{itemize}

	\item Se eligen dos números primos grandes que llamaremos \textbf{p} y \textbf{q}, dichos números se múltiplican y obtenemos \textbf{n}, $n = p \cdot q$, dicho valor de $n$ es el que se usa como módulo de la clave privada y pública. 

	\item Calculamos $\varphi(n)$ de la siguiente forma $\varphi(n)=(p-1)\cdot(q-1)$, donde $\varphi(n)$ es la función $\varphi$ de Euler.

	\item Se elige un entero positivo $e$ menor que $\varphi(n)$ y que sea coprimo con $\varphi(n)$. El valor $e$ es el exponente de la clave pública.
	
	\item Se busca un valor $d$ que satisfaga la congruencia $d=e^{-1}\,mod\,\varphi(n)$. Este valor se suele calcular con el algoritmo de Euclides extendido y el valor $d$ es el exponente para la clave privada. 
\end{itemize}

La clave pública será $(n,e)$ y la clave privada será $(n,d)$.

El proceso de cifrado y descifrado será el siguiente.

\begin{itemize}
	\item \textbf{Cifrado:} el receptor (Alice) envia su clave pública $(n,e)$ al emisor (Bob) y guarda la clave privada, $(n,d)$, en secreto, a partir de este momento Alice usando la clave pública de Bob puede comunicarse seguramente. El mensaje que Bob quiere enviar a Alice será $M$. Bob primero convierte $M$ en un número menor que $n$, y que sea de una forma reversible de forma que sabiendo $n$ podamos volver a conseguir $M$. Acto seguido calcula $c$ de esta forma, $c\equiv m^e\,(mod\,n)$. Bob mandaría $c$ a Alice y la comunicación finalizaría.
	
	\item \textbf{Descifrado:} Alice empezará el proceso para recuperar $m$ a partir de $c$ y $d$. $m\equiv c^d\,(mod\,n)$, una vez ha calculado $m$ puede conseguir $M$.

\end{itemize}

El proceso de descifrado funciona porque $c^d=(m^e)^d\equiv m^{ed}\, (mod\, n)$ y hemos elegido $d$ y $e$ de forma que $e\cdot d =1+k\cdot\varphi(n)$, se cumple que $ m^{ed}\cdot m^{1+k\cdot\varphi(n)} = m(m^{\varphi(n)})^k = m\, (mod\, n)$, esta última congruencia se obtiene directamente del teorema de Euler cuando $m$ y $n$ son coprimos.

Un ejemplo de cifrado y descifrado es el siguiente:



Aquí tenemos un ejemplo de cifrado/descifrado con RSA. Los parámetros usados aquí son pequeños y orientativos con respecto a los que maneja el algoritmo, pero podemos usar también OpenSSL para generar y examinar un par de claves reales.
p=61 	1º nº primo privado
q=53 	2º nº primo privado
n=pq=3233 	producto p*q
e=17 	exponente público
d=2753 	exponente privado

La clave pública (e, n). La clave privada es (d, n). La función de cifrado es:

        \mbox{encrypt}(m) = m^e\pmod{n} = m^{17}\pmod{3233}

Donde m es el texto sin cifrar. La función de descifrado es:

        \mbox{decrypt(c)} = c^d\pmod{n} = c^{2753}\pmod{3233}

Donde c es el texto cifrado. Para cifrar el valor del texto sin cifrar 123, nosotros calculamos:

        \mbox{encrypt(123)} = 123^{17}\pmod{3233} = 855

Para descifrar el valor del texto cifrado, nosotros calculamos:

        \mbox{decrypt(855)} = 855^{2753}\pmod{3233} = 123

Ambos de estos cálculos pueden ser eficientemente usados por el algoritmo de multiplicación cuadrática para exponenciación modular



	%lo relacionado con Android y Google App Engine
	\chapter{Android y Google App Engine.}\label{cap:androidYGAE}
\markboth{CAPÍTULO \ref{cap:androidYGAE}. ANDROID Y GOOGLE APP ENGINE.}{}

En este capítulo vamos a explicar más detenidamente las dos principales plataformas usadas durante el proyecto, Android en la parte móvil y Google App Engine en la parte de las aplicaciones web. 

\section{Android}\label{cap:android}

\subsection{Introducción.}

Android es un sistema operativo basado en Linux, libre y multiplataforma. Inicialmente empezó como un sistema operativo solo para móviles pero con el tiempo ya podemos encontrarlo en móviles, tablets, pc, neveras, relojes, cámaras de fotos y una gran cantidad de aparatos.

En el proyecto lo usaremos para diseñar y desarrollar una aplicación móvil, con la que poder firmar digitálmente un texto leido previamente mediante un lector de códigos QR.

Android es propiedad de Google actualmente y es el encargado de dar soporte y ayudar a los desarrolladores. Esta función la realiza muy bien, dando una muy buena API y una gran documentación. Podemos encontrar toda la información que podemos necesitar para desarrollar una aplicación en la siguiente web, \url{http://developer.android.com/index.html}. En la anterior web podemos encontrar la documentación de la API, consejos de diseño para que la aplicación tenga un aspecto bonito a la vez que usable, todas las novedades incluidas en las versiones nuevas del sistema operativo, etc.

\subsection{Arquitectura de la plataforma Android.}

Como hemos dicho anteriormente Android es una plataforma que engloba desde el sistema operativo, al software intermedio que comunica el sistema operativo y las aplicaciones, llamado en inglés middleware y las posibilidad de hacer funcionar las aplicaciones en la plataforma elegida, ya sea un telefono, un tablet o cualquier aparato con Android.

Para los desarrolladores Android proporciona dos kit de desarrollo, uno que usa la tecnología Java (SDK), el que hemos usado en el proyecto y otro que da la posibilidad de programar a más bajo nivel (NDK), este último desarrollado en C++.

El SDK de Android proporciona ayuda para las siguientes características, un navegador basado en WebKit, graficos optimizados en 2D, gráficos en 3D basados en OpenGL ES 1.0 con aceleración gráfica, una base de datos para almacenar datos que necesitemos, llamada SQLite, soporte para ficheros gráficos (JPG, PNG, GIF, etc), vídeo (MPEG4, H.264) y audio (MP3, AAC), telefonía GSM, tecnologías inalámbricas como son Bluetooth, 3G, Wifi, uso de la cámara, GPS, brújulas, etc. Además de todo esto, proporciona ayuda en la reutilización y remplazo de componentes, una máquina virtual optimizada para dispositivos móviles llamada Dalvik y un emulador donde poder probar la aplicación antes del lanzamiento sin tener que poseer un terminal Android. 

\begin{figure}
  \centering
    \includegraphics[scale=0.6]{./Android/imagenes/arquitecturaAndroid.jpg}
  \caption{Arquitectura de Android.}
  \label{fig:arquitecturaAndroid}
\end{figure} 

La arquitectura la podemos ver en la figura~\ref{fig:arquitecturaAndroid}. En la imagen podemos observar las cinco capas principales en las que se divide Android.
\begin{itemize}

\item \textbf{Applications:} en esta capa están todas las aplicaciones que nos proporciona el sistema operativo de base, como pueden ser la lista de contactos, un gestor de SMS, el navegador, el lanzador de aplicaciones, todas las aplicaciones de servicios de Google, como pueden ser Gmail, Google Maps, Google Calendar, Google Reader.   

\item \textbf{Application framework:} en esta capa está todo lo relacionado con los manejadores de activitys, llamadas de telefono, controladores de vistas. Es una capa que hay intermedia para manejo de hardware, con la que se pueden controlar tanto las notificaciones, como servicios que se ejucutan en segundo plano, etc. Los desarrolladores tienen el mismo acceso mediante esta capa de la API a todos los servicios que las aplicaciones nativas que proporciona el sistema el operativo. 

\item \textbf{Libraries:} como su nombre indica en esta capa están todas las librerías que el sistema operativo necesita, para manejo de archivos multimedia, manejo de gráficos 3D, renderizado web, etc.

\begin{itemize}
	\item \textbf{System C library:} una implementación del estandar C, optimizada para funcionar en sistemas móviles para las funciones del kernell de linux.
	\item \textbf{Media Libraries:} librerías basadas en PacketVideo's OpenCORE para grabación y reproducción de formatos de audio y video mas populares del momento como MP3, H.264, JPG o PNG.

	\item \textbf{Surface Manager:} librería para manejo de graficos 2D y 3D para varias aplicaciones.

	\item \textbf{LibWebCore:} motor de renderizado para navegadores embebidos de páginas web.

	\item \textbf{SGL:} motor para renderizado 2D.

	\item \textbf{3D libraries:} librería que implementa la API de OpenGL ES 1.0, que proporciona aceleración 3D por hardware si es posible o una alta optimización para renderizado por software en sistemas que no posean aceleración por hardware.

	\item \textbf{FreeType:} librería para manejo de fuentes, tanto bitmap como vectoriales.

	\item \textbf{SQLite:} librería para el manejo de la base de datos que proporciona Android.

\end{itemize}

\item \textbf{Android Runtime:} es una capa que está al mismo nivel que la capa de librerías. En esta capa se añaden muchas librerías para dotar de la mayoría de las funcionalidades que proporciona Java, también está en esta capa la máquina virtual (Dalvik) encargada de ejecutar el código Smali de los archivos DEX.

\item \textbf{Linux Kernel:} Android está basado en el kernel de Linux 2.6 y todo lo relatico a seguridad, manejo de memoria, control de procesos, pila de protocolos de red y modelo de drivers es el mismo. El kernel es el que proporciona una capa de abstracción entre el hardware y el software que usará Android.

\end{itemize}

Como ya hemos dicho antes Android corre cada aplicación en una máquina virtual, esta máquina virtual recibe el nombre de Dalvik. Dicho nombre viene de un pueblo de Islandía donde viven los familiares del creador de esta, Dan Bornstein. La máquina virtual ejecuta un byte code especial llamdo DEX (Dalvik Executable), que está especialmente diseñado y optimizado para funcionar en sistemas móviles, tablets, etc. En la figura~\ref{fig:maquinaVirtualDalvik} podemos ver todo el proceso desde la creación del archivo JAVA a la ejecución.

\begin{figure}[h]
  \centering
    \includegraphics[scale=0.8]{./Android/imagenes/maquinaVirtualDalvik.png}
  \caption{Proceso de ejecución en Android.}
  \label{fig:maquinaVirtualDalvik}
\end{figure}


Los ficheros DEX meten las cadenas duplicadas y las constantes en un mismo fichero para ahorrar espacio, normalmente los archivos DEX suelen ser más pequeños que los archivos JAR de la máquina virtual Java. Una vez instados los archivos DEX pueden ser modificados en el terminal para añadir optimizaciones, reordenado de byte en ciertos datos, quitado de clases vacias, etc. En la versión 2.2 de Android se añadió una nueva característica llamada JIT (Just-In-Time) que es compilación en tiempo real de los archivos DEX, por lo que se pueden añadir nuevas optimizaciones dependiendo de la plataforma.

Todas las aplicaciones de Android se distribuyen en unos archivos con extensión APK. Estos archivos no son más que archivos ZIP con la extensión cambiada. Todos deben tener una estructura idéntica, que se explica a continuación. Contiene diferentes carpetas en las que se incluyen ficheros de configuración, fiecheros necesarios para el funcionamiento de la aplicación y para comprobar la integridad de los mismos.

\begin{itemize}

\item \textbf{META-INF:} en un directorio que contiene tres archivos, \textit{MANIFEST.MF} que es el archivo de manifest, \textit{CERT.RSA} que es el certificado con el que está firmada la aplicación, \textit{CERT.SF} que contiene el hash en SHA-1 de todos los componentes de la aplicación, un ejemplo del archivo \textit{CERT.SF} es el siguiente:

\begin{verbatim}

Signature-Version: 1.0
Created-By: 1.0 (Android)
SHA1-Digest-Manifest: wxqnEAI0UA5nO5QJ8CGMwjkGGWE=
...
Name: res/layout/exchange_component_back_bottom.xml
SHA1-Digest: eACjMjESj7Zkf0cBFTZ0nqWrt7w=
...
Name: res/drawable-hdpi/icon.png
SHA1-Digest: DGEqylP8W0n0iV/ZzBx3MW0WGCA=
\end{verbatim}

\item \textbf{lib:} esta carpeta puede contener otras dependiendo de la plataforma para la que esté diseñada la aplicación. Si por ejemplo tiene código específicamente diseñado para x86 tendrá una carpeta llamada x86, si tiene código para MIPS una llamada mips donde se encontraría el código especialmente diseñado para esta plataforma. Puede que dicha carpeta no exista.

\item \textbf{res:} este directorio contiene todos los recursos que no tienen que ser compilados, como pueden ser imágenes, sonidos, etc.

\end{itemize}

Además de estas carpetas todos los ficheros APK incluyen estos tres archivos.

\begin{itemize}
\item \textbf{AndroidManifest.xml:} es un archivo que sirve para indicar la versión, los permisos que contiene la aplicación, las referencias a librerías, las activitys que contiene la aplicación. En el siguiente listado podemos observar un extracto de los permisos que usamos en la aplicación del proyecto.

\begin{lstlisting}[style=XML]
<uses-permission android:name="android.permission.INTERNET"/>
<uses-permission android:name="android.permission.WRITE_EXTERNAL_STORAGE"/>
<uses-permission android:name="android.permission.WRITE_SETTINGS"/>
<uses-permission android:name="android.permission.GET_ACCOUNTS"/>
<uses-permission android:name="android.permission.USE_CREDENTIALS" />
<uses-permission android:name="android.permission.MANAGE_ACCOUNTS" />
\end{lstlisting}

Podemos ver que con esto garantizamos que la aplicación pueda conectarse a internet, usar la tarjeta SD y manejar las cuentas almacenadas en el móvil, como puede ser la dirección de correo usada para dar de alta el móvil en Google Play.

\item \textbf{classes.dex:} el archivo DEX donde están todas las clases precompiladas en byte code para la máquina virtual Dalvik.

\item \textbf{resources.arsc:} es un fichero que contiene los recursos precompilados como pueden ser los XML de las interfaces de la aplicación, etc.

\end{itemize}

En la imagen~\ref{fig:estructuraAPK} se pueden observar los ficheros y las carpetas comentadas anteriormente del proyecto.

\begin{figure}[h]
  \centering
    \includegraphics[scale=0.8]{./Android/imagenes/estructuraAPK.png}
  \caption{Estructura del fichero APK del proyecto.}
  \label{fig:estructuraAPK}
\end{figure}

Toda aplicación en Android se construye con unos componentes básicos como pueden ser activities, intents, views, services, contents providers, widgets y un largo etcétera. A continuación vamos a explicar los más importantes.

\begin{itemize}

%poner sección
\item \textbf{Activity:} una activity es la entidad más básica de una interfaz de usuario donde se puede mostrar información en Android, podemos pensar que es como una ventana de cualquier aplicación de escritorio. Cada interfaz que vemos en una aplicación de Android es una activity. Android tiene un ciclo de manejo de activitys bastante complejo, se explicará en la sección~\ref{cap:desarrolandoAndroid} donde pondremos más incapié en el desarrollo de la aplicación móvil del proyecto.

\item \textbf{Intent:} un intent es un componente prácticamente imprescindible en cualquier aplicación de Android, es una forma de comunicación entre cualquier componente. Se pueden definir como mensajes o peticiones, ya que también puede comunicar aplicaciones entre sí. En el proyecto usamos intent para comunicar las activities y poder intercambiar datos entre ellas, también la usamos para abrir el lector de códigos QR que necesitamos, dentro de nuestra aplicación llamamos al intent que nos proporciona la aplicación lectora y cuando termine ella nos devuelve el valor que había en el código QR para que podamos tratarlo.

\item \textbf{View:} son los componentes básicos con los que podemos contruir las interfaces gráficas, como pueden ser botones, barras de texto, campos de texto, spinner, etc. Android pone a nuestra disposición una gran cantidad de estos elementos, y además brinda la posibilidad de crear nuevos, según los vayamos necesitando. Estos objetos normalmente se añaden a una vista y se pueden añadir, modificar o borrar en ella. Estos objetos tienen un ID único en la aplicación por el que podemos controlarlo y añadirle por ejemplo el texto si es un campo de texto o un listener si es un botón para que cuando se pulse realizar la acción que necesitemos.

\item \textbf{Services:} un servicio es un componente de Android que no tiene interfaz gráfica asociada y se ejecuta en segundo plano. Es similar a los servicios que ofrece cualquier sistema operativo. Pueden realizar cualquier acción, tanto recoger o actualizar datos, lanzar notificaciones cada cierto tiempo o mostrar activity para que el usuario introduzca algún valor que necesite.

\item \textbf{Content Provider:} es un mecanismo que posee Android para intercambiar información entre aplicaciones. Por ejemplo cuando usamos la opción de compartir en el teléfono, dentro de una aplicación, nos salen varias aplicaciones con las que podemos compartir directamente, estas son todas las aplicaciones que han implementado el content provider que necesita esta aplicación para compartir la información. En la figura~\ref{fig:contentProvider} podemos ver la opción de compartir de una aplicación y podemos observar como aparece por ejemplo GMail para mandar un email directamente desde aquí sin tener conocimiento de como se produce el intercambio de datos.
 
\begin{figure}[h]
  \centering
    \includegraphics[scale=0.2]{./Android/imagenes/contentProvider.png}
  \caption{Opción compartir de una aplicación.}
  \label{fig:contentProvider}
\end{figure}

\item \textbf{Broadcast Receiver:} es un componente de Android diseñado para actuar cuando ocurre un evento general del sistema, como puede ser la recepción de un SMS, la batería se está agotando, una llamada entrante, etc. También una aplicación puede generar eventos de este tipo para que cualquiera que implemente un Broadcast Receiver pueda recibirlo.

\item \textbf{Widget:} son elementos visuales y generalmente para que el usuario realice alguna acción, tales como poner en pausa la música, pasar de canción, revisar los feed RSS, mirar los correos pendientes, etc. Suele estar en alguna de las pantallas principales de Android. En la figura~\ref{fig:widget} podemos ver un widget de GMail.
 
\begin{figure}[h]
  \centering
    \includegraphics[scale=0.2]{./Android/imagenes/widget.png}
  \caption{Widget de GMail.}
  \label{fig:widget}
\end{figure}

\end{itemize}

\subsection{Desarrollando en Android.}\label{cap:desarrolandoAndroid}

Como ya hemos dicho anteriormente para el desarrollo de la aplicación móvil hemos usado el IDE de programación Eclipse. En el anexo~\ref{cap:apendiceA} podemos ver como realizar la configuración de Eclipse para la programación de aplicaciones Android. En la sección~\ref{cap:proyectoBasico} se explicará con más detenimiento la estructura de un proyecto básico en Android.

En este apartado vamos a explicar las entidades más importantes que existen en Android, como pueden ser las activity, fragment, y view usados en el proyecto.

En el proyecto la entidad más importante que hemos usado es la \lstinline{Activity}. Normalmente las activity están enfocadas a interactuar con el usuario de alguna forma, ya puede ser necesitando de alguna acción del usuario o mostrando información y normalmente ocupan toda la pantalla del terminal, pero puede flotar dentro de otra activity o agruparse con otras usando la clase \lstinline{ActivityGroup}. Todas las subclases que hereden de \lstinline{Activity} tienen que implementar dos métodos imprescindibles. Estos son \lstinline{onCreate(Bundle)} y \lstinline{onPause()}. El primero es donde se inicializa la activity y en el segundo es lo que ocurre cuando el usuario abandona la activity.

Habitualmente toda Activity tiene asociada una interfaz gráfica que se diseña en un fichero XML, como veremos posteriormente, este archivo se le indica a cada Activity con el método \lstinline{setContentView(int);} al cual hay que pasarle una constante que genera automáticamente la clase \lstinline{R} de Android. Una vez se le ha indicado el archivo XML se puede obtener cada uno de los componentes (botones, cuadros de texto, etc) que componen la interfaz gráfica con el método \lstinline{findViewById(int)}. Todo este proceso se tiene que realizar dentro del método \lstinline{onCreate();} y se puede observar en el siguiente trozo de código.

\begin{lstlisting}[style=Java]
@Override
protected void onCreate(Bundle savedInstanceState) {
	super.onCreate(savedInstanceState);
	setContentView(R.layout.initialconfigurescreen);

	TextView tAccountOk = (TextView) findViewById(R.id.tAccountOk);
	EditText tCertificate = (EditText) findViewById(R.id.
		tCertificate);
	ImageView image = (ImageView) findViewById(R.id.imageView1);

	Button bSelectAccount = (Button) findViewById(R.id.bConfigureScreenAccount);
}
\end{lstlisting}

A continuación vamos a explicar las características y su ciclo de vida de las Activity, ya que es un tema muy importante cuando desarrollamos una aplicación Android. En la figura~\ref{fig:cicloActivity} podemos ver todos los estados por los que pasa una Activity desde que se crea hasta que finaliza. 

\begin{figure}
  \centering
    \includegraphics[scale=0.8]{./Android/imagenes/cicloActivity.png}
  \caption{Ciclo de vida de una activity.}
  \label{fig:cicloActivity}
\end{figure}

Todos estos estados se pueden controlar mediante la implementación en la clase que hereda de \lstinline{Activity} de los siguientes métodos.

\begin{lstlisting}[language=Java]
public class Activity extends ApplicationContext {
	protected void onCreate(Bundle savedInstanceState);
	protected void onStart();
	protected void onRestart();
	protected void onResume();
	protected void onPause();
	protected void onStop();
	protected void onDestroy();
}
\end{lstlisting}

Como podemos ver en la figura \ref{fig:cicloActivity} podemos ver que el ciclo completo de una activity es desde el método \lstinline{onCreate(Bundle);} hasta que se realiza la llamada al método \lstinline{onDestroy();}. Como hemos visto antes en \lstinline{onCreate(Bundle);} se genera todo lo necesario para que la activity funcione, como puede ser la inicialización de la interfaz, la creación de un hilo para que realice una operación en background o cualquier otra acción que necesite ser inicializada. El método \lstinline{onDestroy();} se pararía el hilo y se libera la memoria usada por la activity. Entre los procesos \lstinline{onStart();} y \lstinline{onStop();} es donde se mantienen los recursos para que la activity pueda mostrar los datos al usuario. Por ejemplo si tenemos un \lstinline{BroadcastReceiver}, que nos puede cambiar la interfaz de usuario pues lo registramos en el método \lstinline{onStart();} y lo paramos en \lstinline{onStop();}. Estos dos métodos se llaman mucho a lo largo de la ejecución de la activity cada vez que el usuario oculta la activity y vuelve a ejecutarla. Los métodos \lstinline{onResume();} y \lstinline{onPause();} se usan para intercambio de activity, cuando apagamos la pantalla del móvil y volvemos a encenderla, cuando giramos la pantalla, etc. En estos métodos se suelen usar \lstinline{Bundle} para intercambiar información entre los estados y así conseguir por ejemplo restaurar el texto de un cuadro de texto cuando vuelve a generarse.

En esta tabla podemos observar en cada estado del ciclo de vida de una activity puede ser matada y cual sería el próximo estado.
\begin{center}
\begin{tabular}{|l | c | r|}

\hline
Method & ¿Terminable? & Proximo estado\\
\hline
onCreate() & No & onStart()\\
\hline
onRestart() & No & onStart()\\
\hline
onStart() & No & onResume() o onStop()\\
\hline
onResume() & No & onPause()\\
\hline
onPause() & No  & onResume() o onStop()\\
\hline
onStop() & Sí & onRestart() o onDestroy()\\
\hline
onDestroy() & Sí & Ninguna\\
\hline

\end{tabular}
\end{center}

Antes de la versión 3.0 de Android las activity tenían que ocupar toda la ventana y para cambiar o mostrar otra pantalla había que generar una nueva activity de la siguiente forma:

\begin{lstlisting}[style=Java]
Intent intent = new Intent(activity, SplashScreenActivity.class);
startActivity(intent);
\end{lstlisting}

Ese trozo de código se ejecuta en una activity y podemos ver que se crea un objeto \lstinline{Intent} al cual se le dice la activity en la que está y la activity que tiene que iniciar, en este caso la variable \lstinline{activity} es la actual, y \lstinline{SplashScreenActivity} es una activity que tiene la función de inciar todas las variables y realizar las conexiones básicas en el proyecto, acto seguido se usa el procedimiento \lstinline{startActivity(intent);} al que se le pasa el objeto \lstinline{Intent} creado anteriormente y con esto tendríamos la nueva activity ejecutándose.

Desde la versión 3.0 y posteriores las activity siguen ocupando toda la pantalla pero se dio la posibilidad al programador de que usara solo trozos de ella con una clase llamada \lstinline{Fragment} de esta forma no tendría que iniciar una nueva activity cada vez que quiera modificar la interfaz, de este modo se pudieron empezar a usar gestos de scroll laterar para mostrar varias interfaces o en pantallas grandes como una tablet poder modificarla sin tener que generar una nueva activity. 
\begin{figure}
  \centering
    \includegraphics[scale=0.3]{./Android/imagenes/gmailTablet.png}
  \caption{Aplicación de Gmail para tablet.}
  \label{fig:gmailTablet}
\end{figure}

\begin{figure}
  \centering
    \includegraphics[scale=0.4]{./Android/imagenes/gmailMovil.png}
  \caption{Aplicación de Gmail para móvil.}
  \label{fig:gmailMovil}
\end{figure}

\begin{figure}[h]
  \centering
    \includegraphics[scale=0.2]{./Android/imagenes/swype.png}
  \caption{Gesto Swype en la aplicación móvil.}
  \label{fig:swype}
\end{figure}

En la imagen~\ref{fig:gmailTablet} podemos ver la aplicación de Gmail diseñada mediante fragment y en ella si pulsamos algún correo en la parte izquierda de la aplicación nos mostraría el correo en la derecha sin tener que recargar la aplicación. En la imagen~\ref{fig:gmailMovil} podemos ver como en la versión movil no se usa esta forma por falta de espacio en la pantalla. Nosotros hemos realizado un diseño para intercambio de fragment mediante un gesto llamado swype o scroll lateral y como se puede ver en la imagen~\ref{fig:swype} podemos ver que no hay que volver a cargar otra activity ni nada, por lo que dotamos a la aplicación de una mayor fluidez.

Para que una aplicación pueda usar una determinada activity, el programador primeramente tiene que definir el uso y su función en el archivo \textit{AndroidManifest.xml}. Podemos ver un extracto de dicho archivo donde definimos un par de activity usadas en el proyecto.
\newpage
\begin{lstlisting}[style=XML]
<activity android:name=".FirmaDigitalUMA_ICSActivity" />
<activity android:name=".InitialConfiguration" android:noHistory=
	"true" />
\end{lstlisting}

Se puede observar que hemos declarado dos activities una sin ninguna opción y otra en la que no se guardará en la pila de llamadas de activity, por lo que si pulsamos el botón atrás no se abrirá de nuevo. Si no realizamos este proceso nos dará un error en tiempo de ejución la aplicación diciendo que hemos intentado ejecutar una activity que no está declarada.

\subsection{Un Proyecto básico de Android en Eclipse.}\label{cap:proyectoBasico}

A continuación vamos explicar con más detenimiento la estructura que tiene un proyecto básico de Android en Eclipse.

\begin{figure}
  \centering
    \includegraphics[scale=1]{./Android/imagenes/estructuraBasicaAndroid.png}
  \caption{Estructura básica de un proyecto Android.}
  \label{fig:estructuraBasicaAndroid}
\end{figure}

En la figura~\ref{fig:estructuraBasicaAndroid} podemos ver una captura de un proyecto recién creado. Podemos observar que se genera una carpeta principal en la que posteriormente colgarán el resto de carpetas necesarias. Estas carpetas son \textit{src}, \textit{gen}, \textit{assets}, \textit{bin}, \textit{res} y varios archivos sueltos como con \textit{AndroidManifest.xml}, \textit{proguard-project.txt}, \textit{project.properties}, vamos a explicar brevemente que contiene y cual es la función de dichas carpetas y documentos.

\begin{itemize}

\item \textbf{src:} en esta carpeta están todos los paquetes que contiene los archivos de código fuente que se necesitan en el proyecto.

\item \textbf{gen:} esta carpeta es donde se almacena todo lo que el proyecto de Android necesita para funcionar, casi todos los ficheros que se encuentran en el interior se generan cada vez que se construye el proyecto y si los modificamos nosotros, cuando volvamos a construir el proyecto borrarán los cambios. Dentro está la clase \lstinline{R} donde se declaran la mayoría de las constantes con direcciones de memoria que luego en tiempo de ejecución se usarán para realizar la conversión en bytecode del archivo java.

\item \textbf{bin:} es una carpeta donde se almacenan todos los archivos binarios, como puede ser el archivo APK, los archivos DEX, etc.

\item \textbf{res:} esta carpeta la encargada de contener todos los recursos necesarios para nuestra aplicación. Esta carpeta se divide en varias, como por ejemplo \textit{drawable-hdpi}, \textit{drawable-ldpi}, \textit{drawable-mdpi}, \textit{drawable-xhdpi} es donde se añaden todas imágenes usadas, sonidos, vídeos, etc. Pero no todos los recursos son contenido multimedia, hay otras carpetas como por ejemplo la carpeta \textit{layout} donde se almacenan las diferentes interfaces usadas en el formato XML o la carpeta \textit{values} donde se guardan todas las cadenas constantes en un archivo XML.

\item \textbf{AndroidManifest.xml:} como ya hemos explicado anteriormente es el archivo donde se declara todos los permisos e información de interés de la aplicación, como pueden ser las activity, los intent, la versión mínima que tiene que tener el móvil para ejecutar nuestra aplicación, etc.  

\item \textbf{project.properties:} es un archivo donde se pueden configurar diferentes parámetros del proyecto, como puede ser la API sobre la que se va a ejecutar el proyecto, o si queremos usar una herramienta que ofrece Google dentro del SDK para ofuscar el código llamada ProGuard.

\item \textbf{proguard-project.txt:} ProGuard como hemos dicho antes es una herramienta que ofrece Google dentro del SDK de Android para ofuscación de código, ya que hay muchas herramientas de ingeniería inversa que mediante la decompilación de los archivos DEX se puede llegar casi a conseguir el código realizado sin permiso. En este archivo se puede configurar los diferentes valores para el uso de esta herramienta, tales como son: qué tipo de ofuscación queremos utilizar si solo sintáctica o semántica, si queremos que se pueda tracear la salida del archivo, etc. Para ampliar conocimientos sobre dicha herramienta podemos visitar esta web, \url{http://developer.android.com/tools/help/proguard.html} donde está toda la información necesaria.

\end{itemize}

%-----------------------------------------
%Google App Egine
%-----------------------------------------


\section{Google App Engine.}\label{cap:GAE}
En este apartado de la memoria vamos a explicar lo que es, la configuración y como usar la plataforma Google App Engine.

\subsection{Introducción.}
Google App Engine es una conjunto de APIS que proporciona Google para construir tus propias aplicaciones web, que pueden ser alojadas y usadas en su servicio Google App y vendidas en Google Apps Marketplace. Además de alojamiento gratuito, Google ofrece un dominio, que es como el siguiente: \url{http://nombre\_de\_la\_aplicacion.appspot.com} y una base de datos propietaria de Google que se accede transparentemente a través de su API, gestión de usuarios mediante autentificación con cuentas Google del tipo: usuario@gmail.com, autentificación por federación o openID.

Además de todas estas características Google proporciona APIS para el desarrollo con Java, Python y Go, este último un lenguaje experimental propiedad de Google. Para usar dicha API, Google también proporciona un plugins para Eclipse, en caso de que el lenguaje elegido sea Java, que ayuda al despliegue de la aplicación web, auto completado y gestión de de las aplicaciones creadas. 

En el anexo~\ref{cap:configuracionGAEEclipse} se puede ver como instalar el plugins de Google App Engine para Eclipse.

En el proyecto se ha usado Java, por lo que las APIS de Python y Go no se han estudiado.

En general el uso de Google App Engine para desarrollar aplicaciones web es idéntico a crear una aplicación web con Java 2 Enterprise Edition (Java2EE), se pueden desarrollar servlet que recogen valores mediante métodos \lstinline{GET} o \lstinline{POST} y usar clases Java para hacer operaciones con ellos. A su vez para mostrar la información se pueden generar archivos *.jsp, que son archivos HTML con bloques o líneas de código Java incrustadas, que se introducen con estas etiquetas: \lstinline{<\%= linea de codigo Java \%>} o \lstinline{<\% bloque de codigo Java \%>}. A parte de archivos *.java y *.jsp, debemos tener una carpeta llamada war en la que tiene que ir toda la información de la aplicación web que queremos desplegar. En dicha carpeta hay varias subcarpetas como pueden ser css en la que tiene que ir el estilo de la web o WEB-INF en la que están todos los archivos de configuración, como pueden ser los permisos que tenemos que tener para poder acceder al uso de un servlet, si la web tiene conexión https, la configuración de la base de datos, etc. Más adelante se explicará con más detenimiento todas las carpetas de las que se compone un proyecto de Google App Engine.

Para este proyecto hemos tenido que desarrollar dos aplicaciones web, una que es un servidor de timestamp y otra que es una aplicación para gestión de las firmas digitales que realice cada usuario. A continuación vamos a explicar en profundidad la tecnología usada.

\subsection[Aplicación web genérica en GAE]{Explicación de una aplicación web genérica en Google App Engine.}
En esta parte voy a explicar en profundidad que es un servlet, los archivos de configuración, los archivos JSP y el resto de archivos necesarios para poder desplegar una aplicación en Google App Engine.
 
\subsubsection{¿Qué es un servlet?.}
Un servlet es la evolución de los antiguos applets de Java, su uso más común es generar páginas web dinámicamente con los parámetros que recibe mediante una petición realizada por el navegador web y datos que están almacenados en el servidor web.

Un servlet es un objeto Java que tiene que ser ejecutado en un servidor web o contenedor J2EE, que recibe unos parámetros, realiza una o varias acciones y devuelve un resultado que puede ser desde un código HTML, un JSP que genera dinámicamente un código HTML, un JSON o una simple cadena de texto.

Los servlets, junto con JSP, son la solución de Oracle a la generación de contenido dinámico equivalente al lenguaje PHP, ASP de Microsoft, Ruby, etc.

Los servlets forman parte de Java 2 Enterprise Edition (J2EE) que a su vez es una amplicación de Java 2 Standard Edition (J2SE), para su uso es necesario un servidor web que pueda interpretar código Java, el más famoso es Apache Tomcat que está desarrollado y mantenido por Apache Foundation, que son los encargados también de mantener y desarrollar el famoso servidor web Apache, aunque existen otro como JBoss, Jetty o GlassFish, pero como podremos ver no son los únicos, ya que el propio Google App Engine también funciona internamente a base de servlets y JSP.

Para crear un servlet hay que generar una clase Java que implemente la interfaz \lstinline{javax.servlet.Servlet} como puede ser \lstinline{javax.servlet.http.HttpServlet} que es un servlet específico para conexiones HTTP.
 
Una vez generada la clase hay que implementar el método \lstinline{doGet} para peticiones tipo \lstinline{GET} o el método \lstinline{doPost} para peticiones de tipo \lstinline{POST}. En el siguiente trozo de código se puede ver la implementación más básica de los métodos \lstinline{doGet} y \lstinline{doPost}.

\begin{lstlisting}[style=Java] 
@Override
protected void doGet(HttpServletRequest req, 
	HttpServletResponse resp) throws ServletException, IOException {
	// TODO Auto-generated method stub
	super.doGet(req, resp);
}

@Override
protected void doPost(HttpServletRequest req, 
HttpServletResponse resp) throws ServletException, IOException {
	// TODO Auto-generated method stub
	super.doPost(req, resp);
}
\end{lstlisting}

Una vez implementados los métodos que se necesiten se pueden usar el parámetro \lstinline{HttpServletRequest req} para recibir los valores que queramos enviar a la aplicación web y podemos usar \lstinline{HttpServletResponse resp} para enviar lo que queramos desde una redirección a un JSP, una página web, un JSON o una cadena de texto. 

Un ejemplo de como se reciben los parámetros sería: 

\begin{lstlisting}[style=Java]  
String num_sec = req.getParameter("sec");
\end{lstlisting}

Y si queremos devolver algo, por ejemplo un objeto \lstinline{JSONArray}:

\begin{lstlisting}[style=Java]   
PrintWriter out = resp.getWriter();
out.print(jsonArray);
out.flush();
\end{lstlisting}

Como podemos ver el objeto \lstinline{resp} nos da la posibilidad de conseguir un objeto \lstinline{java.io.PrintWriter} por el que podemos enviar lo que necesitemos.

La forma de acceder a un servlet mandándole peticiones \lstinline{GET} sería la siguiente: \url{https://servertimestamp.appspot.com/search?id=63&texto=Prueba}. Como podemos ver la dirección base es: \url{https://servertimestamp.appspot.com/}, el servlets estaría mapeado internamente en el servidor web, como ya veremos, en la dirección \url{/search} y el primer parámetro va precedido de \url{?id\_parametro} y el resto de \url{\&id\_parametro}. En nuestro ejemplo tendría dos parámetros que son \textit{id} y \textit{texto}, con sus valores después del =.

El método \lstinline{POST} es el utilizado para pasar parámetros por medio de formularios.

\subsection{¿Qué es JSP?.}
JSP es el acrónimo de JavaServer Pages y es una tecnología que ayuda a crear dinámicamente páginas web basadas en HTML o XML y es la solución equivalente a PHP de Oracle. En la figura~\ref{fig:modoJSP} podemos ver el proceso que se hace desde que se realiza la petición en el navegador hasta que se muestra un resultado.

\begin{figure}
  \centering
    \includegraphics[scale=0.5]{./GoogleAppEngine/imagenes/JSP_Model.png}
  \caption{Modo de interpretación de un archivo JSP}
  \label{fig:modoJSP}
\end{figure}

Un fichero JSP es la unión de código HTML con código Java, el cual es interpretado en el momento de la visualización de la página web. Un ejemplo es el siguiente:
 
\begin{lstlisting}[style=HTML]   
<!DOCTYPE html>
<html>
<body>
<table>
<tr>
	<th>ID</th>
	<th>Num sec</th>
	<th>Token de tiempo</th>
	<th>Mensaje</th>
	<th>URL para ver la firma</th>
	<th>Fecha</th>
	<th>Usuario</th>
	<th id="filadestino">Destino</th>
	<th>Verificado?</th>
</tr>
<% for (RowRepositorioGeneral row : rows) {%>
<tr>
	<td><%=row.getId()%></td>
	<td><%= row.getNum_sec()%></td>
	<td><%=row.getToken_tiempo()%></td>
	<td><%=row.getTexto_claro()%></td>
	<td><a href=<%=row.getUrl_firma()%>>URL para ver el token
			de tiempo</a></td>
	<td><%=row.getFecha()%></td>
	<td><%=row.getUsuario()%></td>
	<td id="filadestino"><%=row.getDestino()%></td>
	<td>
		<%Boolean confirmado = row.getConfirmado();
		if (!(confirmado == null) && confirmado)  else  %>
	</td>
</tr>
<%}%>
</table>
</body>
</html>
\end{lstlisting}

Como se puede ver en este trozo de código de este archivo JSP genera una tabla que se rellena dinámicamente con los valores que devuelve un objeto Java, se puede observar que se entrelazan trozos de código Java con etiquetas HTML. Si mostramos esta web y acto seguido introducimos otro objeto \lstinline{RowRepositorioGeneral} en la estructura, cuando recarguemos la tabla tendrá una fila nueva.

\subsubsection{La carpeta WAR.}

La carpeta WAR es la carpeta principal para el despliegue de una aplicación web, ya que en ella es donde se almacenan todos los archivos que se necesitan para el funcionamiento de la aplicación web, como pueden ser archivos HTML, CSS, JSP, imágenes, etc. En la figura~\ref{fig:carpetawar} se puede observar la carpeta WAR de una de las aplicaciones web realizadas.

\begin{figure}
  \centering
    \includegraphics{./GoogleAppEngine/imagenes/carpetawar.png}
  \caption{Carpeta WAR}
  \label{fig:carpetawar}
\end{figure}

Se puede observar las diferentes carpetas y ficheros que la forman. Podemos ver que la carpeta css contiene los archivos de estilo que la página web usará, también podemos ver los archivos web.xml y app.yalm, que son archivos de configuración del servidor que se verán en el próximo apartado~\ref{cap:refArchivosConfiguracionGoogleAppEngine} y además los archivos JSP que se usan en la aplicación junto con los archivos HTML y JavaScript que se necesiten.

\subsubsection{Archivos de configuración.\label{cap:refArchivosConfiguracionGoogleAppEngine}}
Los principales archivos de configuración son web.xml y app.yalm, este segundo es solo una forma de escribir de forma más legible XML, para que nos sea más sencillo entenderlo.

Un ejemplo de un archivo web.xml es el siguiente:

\begin{lstlisting}[language=XML]
<?xml version="1.0" encoding="utf-8"?>
<web-app xmlns:xsi="http://www.w3.org/2001/XMLSchema-
	instance"
xmlns="http://java.sun.com/xml/ns/javaee"
xmlns:web="http://java.sun.com/xml/ns/javaee/web-app_2_5.xsd"
xsi:schemaLocation="http://java.sun.com/xml/ns/javaee
http://java.sun.com/xml/ns/javaee/web-app_2_5.xsd" version=
	"2.5">

	<servlet>
		<servlet-name>AddRow</servlet-name>
		<servlet-class>pfc.ServletCreateRow</servlet-class>
	</servlet>
	<servlet-mapping>
		<servlet-name>AddRow</servlet-name>
		<url-pattern>/add</url-pattern>
	</servlet-mapping>

	<welcome-file-list>
		<welcome-file>ServerTimestampApplication.jsp</welcome-file>
	</welcome-file-list>
</web-app>
\end{lstlisting}

Como se puede observar en el código anterior se ha definido un servlet que se llamará \lstinline{AddRow} que usará la clase \lstinline{ServletCreateRow} y que estará mapeado en la dirección web \url{/add}, también podemos observar que el fichero que nos mostrará el servidor será \lstinline{ServerTimestampApplication.jsp} si entramos a la url principal.

A continuación podemos ver el aspecto de un archivo app.yalm:

\begin{lstlisting}[style=YAML]
application: repositoriorecibos
version: 1
runtime: java

handlers:
  - url: /add
    servlet: pfc.ServletCreateRow
    secure: always
welcome_files:
  - RepositorioGeneralApplication.jsp
\end{lstlisting}

Como podemos observar es mucho más fácil de entender y de escribir, el único problema que tienen los archivos YALM es que son sensibles a los espacios en blanco y tabulaciones, por lo que hay que tener cuidado a la hora de redactarlos. En este archivo se crea un servlet en la ruta \url{/add}, que es la clase Java \lstinline{ServletCreateRow} del paquete \lstinline{pfc} y que siempre hay que estar registrado en la aplicación para poder acceder a él. También podemos observar el fichero de bienvenida para cuando accedemos a la aplicación web. 

Al tener el archivo app.yalm en la carpeta WEB-INF el parseador de YALM  interpreta dicho archivo y genera automáticamente un archivo web.xml que usará el servidor web para su configuración.

Para ver todas las opciones de configuración que se pueden modificar en app.yalm\footnote{ Parámetros de configuración en el archivo app.yalm, \url{https://developers.google.com/appengine/docs/java/configyaml/}} o en web.xml\footnote{ Parámetros de configuración en el archivo web.xml, \url{https://developers.google.com/appengine/docs/java/config/}} se puede consultar los enlaces que hay en las notas al pie.












	
	%android
	%\input{./Android/android.tex}
	%google app engine
	%\input{./GoogleAppEngine/googleAppEngine.tex}
	
	%Diseño y arquitectura
	\input{./DisenhoYArquitectura/disenhoYArquitectura.tex}
	
	%Conclusiones y trabajo futuro
	\input{./ConclusionesYTrabajoFuturo/conclusionesYTrabajoFuturo.tex}
		
	\appendix
	%apendice con la configuración de eclipse tanto para Android como para Google App Engine
	\chapter[Configuración de Eclipse.]{Configuracion del entorno de programación Eclipse.}\label{cap:apendiceA}

A continuación vamos a explicar como instalar y configurar Eclipse para que podamos programar para Android y para Google App Engine, vamos a explicar como instalar los plugins que proporciona Google en ambos casos.

\section{Configuración de Eclipse para Android.}\label{cap:configuracionAndroidEclipse}

Lo primero que tenemos que hacer es descargarnos Eclipse y el SDK de Android. Para el primero vamos a la web \url{http://www.eclipse.org/downloads/} y bajamos la versión clasic de Eclipse. Yo recomiendo la versión Eclipse Classic porque es la versión básica que trae todo lo necesario para programar para Android. A continuación bajamos el SDK de Android de la siguiente web \url{http://developer.android.com/sdk/index.html}. Al haber realizado el proyecto en Ubuntu toda esta configuración será para Linux, pero es equivalente a como se debería de realizar en Windows o Mac ya que todo el software es multiplataforma, solo tendríamos que descargar las versiones específicas para nuestro sistema.

Una vez descargados ambos archivos procedemos descomprimirlos, con lo que obtenemos dos carpetas, una con el SDK de Android y otra con Eclipse. A continuación vamos a instalar el SDK de Android, para ello abrimos una terminal y nos desplazamos a la carpeta \textit{\\tools} dentro de la carpeta donde hemos descomprimido el SDK y ejecutamos el siguiente comando: 

\begin{lstlisting}[style=consola]
./android sdk
\end{lstlisting}

El resultado de ejecutar este comando podemos verlo en la figura~\ref[hola]{fig:androidSDK}

\begin{figure}
  \centering
    \includegraphics[scale=0.4]{./ConfiguracionEclipse/imagenes/androidSDK.png}
  \caption{Ventana de instalación del SDK de Android.}
  \label{fig:androidSDK}
\end{figure}

En la pantalla que aparece podemos elegir que versión del SDK queremos instalar dependiendo de la versión sobre la que queramos desarrollar, nosotros instalaremos la versión 4.0. Seleccionamos la versión 4.0 y pulsamos en instalar. Aceptamos las licencias y pulsamos en Install (figura~\ref{fig:licencias}), acto seguido el instalador empieza a descargar e instalar todos los paquetes que hemos seleccionado.

\begin{figure}
  \centering
    \includegraphics[scale=0.6]{./ConfiguracionEclipse/imagenes/licencias.png}
  \caption{Ventana para aceptar las licencias.}
  \label{fig:licencias}
\end{figure}

\begin{figure}
  \centering
    \includegraphics[scale=0.6]{./ConfiguracionEclipse/imagenes/SDKfinalizado.png}
  \caption{Instalación del SDK finalizada.}
  \label{fig:SDKfinalizado}
\end{figure}

Una vez hemos instalado el SDK (figura~\ref{fig:SDKfinalizado}) tenemos que instalar el puglins para Eclipse, para ello abrimos Eclipse y vamos a \textbf{Help -> Install New Software}, pulsamos en \textbf{Add} y copiamos la siguiente dirección web, \url{https://dl-ssl.google.com/android/eclipse/} y pulsamos \textbf{Ok}. En la figura~\ref{fig:instalacionADT} podemos ver el resultado y la opción que tenemos que marcar. Pulsamos \textbf{Siguiente} y aceptamos las licencias y pulsamos \textbf{Finalizar}, a continuación Eclipse empezará a descargar los paquetes que le hemos solicitado y cuando termina los instala.

\begin{figure}
  \centering
    \includegraphics[scale=0.6]{./ConfiguracionEclipse/imagenes/instalacionADT.png}
  \caption{Instalación del ADT en Eclipse.}
  \label{fig:instalacionADT}
\end{figure}

Una vez instalados nos pide que reiniciemos Eclipse, al iniciar de nuevo nos aparece un asistente para la configuración del SDK que vamos a usar en Eclipse, tenemos que saber la ruta donde hemos instalado anteriormente el SDK, podemos ver el asistente en la siguiente figura~\ref{fig:configuracionSDK}.
 
\begin{figure}
  \centering
    \includegraphics[scale=0.6]{./ConfiguracionEclipse/imagenes/configuracionSDK.png}
  \caption{Instalación del SDK en Eclipse.}
  \label{fig:configuracionSDK}
\end{figure}

Como podemos observar en la figura~\ref{fig:nuevoProyectoAndroid} ya tenemos configurado Eclipse para programar en Android.

\begin{figure}
  \centering
    \includegraphics[scale=0.6]{./ConfiguracionEclipse/imagenes/nuevoProyectoAndroid.png}
  \caption{Creación de un nuevo proyecto Android.}
  \label{fig:nuevoProyectoAndroid}
\end{figure}

Para el uso de Git hemos usado el plugins eGit, que se puede descargar gratuitamente de la web \url{http://www.eclipse.org/egit/}. En dicha web también podemos ver la forma de instalación y configuración, que es muy sencilla y solo se necesita la URL del servidor, el nombre de usuario y el password.

\section{Configuración de Eclipse para Google App Engine.}\label{cap:configuracionGAEEclipse}

Al igual que en el apéndice~\ref{cap:configuracionAndroidEclipse} necesitamos tener una versión de Eclipse Classic, yo recomiendo tener un Eclipse diferente para cada plugins que queramos utilizar para evitar incompatibilidades entre los diferentes plugins. Comenzamos como en la instalación del ADT para Eclipse, abriendo Eclipse y pulsamos en \textbf{Help -> Install New Software}, luego en \textbf{Add} y copiamos la siguiente dirección web, \url{http://dl.google.com/eclipse/plugin/3.7} y pulsamos \textbf{Ok}. En la figura~\ref{fig:instalacionGAE} podemos ver el resultado y los paquetes que nos da la posibilidad de instalar, a nosotros nos interesa dentro de \textbf{SDK}, la opción \textbf{Google App Engine Java SDK}, también necesitamos el \textbf{Google Plugin for Eclipse 3.7} y si queremos hacer diseño de la interfaz web de la aplicación podemos instalar \textbf{GWT Designer for GPE}, nosotros no lo usaremos en el proyecto. A continuación pulsamos \textbf{Siguiente}, aceptamos las licencias y pulsamos finalizar. Eclipse acto seguido empezará a descargar los puglins y a instalarlos (figura~\ref{fig:instalacionPlugins}), cuando termine nos pedirá que reiniciemos Eclipse.

\begin{figure}
  \centering
    \includegraphics[scale=0.6]{./ConfiguracionEclipse/imagenes/instalacionGAE.png}
  \caption{Instalación puglins Google App Engine.}
  \label{fig:instalacionGAE}
\end{figure}

\begin{figure}
  \centering
    \includegraphics[scale=0.6]{./ConfiguracionEclipse/imagenes/instalacionPlugins.png}
  \caption{Detalle de la descarga de plugins.}
  \label{fig:instalacionPlugins}
\end{figure}

Cuando se vuelve a abrir Eclipse ya vemos que hay varias cosas que han cambiado como podemos ver en la figura~\ref{fig:eclipseGAE}. Como podemos ver ya estamos logueados con nuestra cuenta de usuario y nos da la posibilidad de crear aplicaciones web en la plataforma. En el botón de Google también hay una acción muy importante que explicaremos posteriormente que es la de desplegar una aplicación web en el servicio de Google App Engine. Todo lo que hagamos podemos probarlo en local simplemente pulsado el botón play y Eclipse deplegará un servidor Tomcat en el que probar la aplicación web que desarrollemos.

\begin{figure}
  \centering
    \includegraphics[scale=0.6]{./ConfiguracionEclipse/imagenes/eclipseGAE.png}
  \caption{Eclipse con el puglin de Google App Engine instalado.}
  \label{fig:eclipseGAE}
\end{figure}

Una vez preparado Eclipse para que podamos programar, hay que genera los dominios donde correrán nuestras aplicaciones web. De forma gratuita Google proporciona 10 dominios diferentes por cuenta. Para crearlos hay que ir a la siguiente dirección web: \url{https://appengine.google.com/}, en ella nos logueamos con nuestra cuenta de Google, podemos ver en la figura~\ref{fig:appEngine} el panel de administración de las aplicaciones.  

\begin{figure}
  \centering
    \includegraphics[scale=0.6]{./ConfiguracionEclipse/imagenes/appEngine.png}
  \caption{Panel de administración Google App Engine.}
  \label{fig:appEngine}
\end{figure}

Pulsando el botón \textbf{Create Application} tenemos acceso a todos los parámetros de configuración para la aplicación web. Podemos ver todos los parámetros que nos pide en la imagen~\ref{fig:createGAE}. Nos pide un nombre que será el dominio que nos proporcionará Google, el título de la aplicación web y la forma de logueo que queremos que tenga nuestra aplicación en caso de necesitarla.

\begin{figure}
  \centering
    \includegraphics[scale=0.6]{./ConfiguracionEclipse/imagenes/createGAE.png}
  \caption{Parámetros para la creación de una nueva aplicación web.}
  \label{fig:createGAE}
\end{figure}

Una vez creada la aplicación web, tenemos acceso a un dashboard donde podemos configurar la aplicación, ver estadísticas, los log, administrar la base de datos, etc. Podemos verlo en la figura~\ref{fig:dashboardGAE}.

\begin{figure}
  \centering
    \includegraphics[scale=0.6]{./ConfiguracionEclipse/imagenes/dashboardGAE.png}
  \caption{Dashboard para la configuración de una aplicación web.}
  \label{fig:dashboardGAE}
\end{figure}

Con esto habríamos terminado la parte de la creación de la aplicación web y a continuación vamos a explicar como hacer el despliegue de una aplicación con Eclipse.

Una vez tenemos un proyecto creado y testeado procederemos a la subida, para eso tendríamos que pulsar en el botón de Google que vimos antes y luego pulsar en la opción \textbf{Deploy to App Engine}. Nos aparece una nueva ventana en la que debemos configurar unos parámetros básicos, pero si es la primera vez que vamos a hacer el despliegue tenemos pulsar en \textbf{App Engine project settings...} (figura~\ref{fig:GAESettings}), en ventana que nos aparece podemos configurar muchos más parámetros, el más importante el \textbf{Application ID} y \textbf{Version}. En \textbf{Application ID} tenemos que poner el nombre con el que hemos creado la aplicación web, en nuestro caso \textit{repositoriorecibos}. También podemos cambiar la versión del SDK que queremos usar, que según hemos podido observar durante el desarrollo del proyecto se ha actualizado varias veces, si la base de datos es altamente replicada, etc.

\begin{figure}
  \centering
    \includegraphics[scale=0.6]{./ConfiguracionEclipse/imagenes/GAESettings.png}
  \caption{Despliegue de una aplicación web.}
  \label{fig:GAESettings}
\end{figure}

Una vez realizado el despliegue ya podemos tener acceso a la aplicación web mediante el link \url{http://nombre\_aplicacion.appspot.com}.























	%apendice para la creación de los certificados usados con XCA
	\input{./AnexoCreacionCertificado/creacionCertificado.tex}
	%apendice con el contenido del CD
	\chapter{Contenido del CD.}

El CD que acompaña a la memoria contine los siguientes archivos:

\begin{itemize}

\item{Archivos fuentes del proyecto Android.}
\item{Archivos fuentes del proyecto Google App Engine.}
\item{Archivos fuente en \LaTeX e imagenes usadas en la memoria.}
\item{Presentación realizada.}

\end{itemize}

	
	\backmatter
		\listoffigures
		%bibliografía
		\input{./Bibliografia/bibliografia.tex}
		
\end{document}
