\documentclass[a4paper,12pt]{book}
\usepackage[utf8]{inputenc}
\usepackage[spanish]{babel}

%\usepackage{listings}
\usepackage{graphicx}
%\usepackage{amssymb}
\usepackage{hyperref}
\usepackage{eurosym}
\usepackage{pdfpages}


%para que la sección sea en mínuscula y no en mayusculas
\usepackage[nouppercase]{scrpage2} % encabezados
\pagestyle{scrheadings} %encabezados
\setlength{\headheight}{1.1\baselineskip} %encabezados


\usepackage{color}
\definecolor{gray97}{gray}{.97}
\definecolor{gray75}{gray}{.75}
\definecolor{gray45}{gray}{.45}

\usepackage{listings}
\lstset{ frame=Ltb,
framerule=0pt,
aboveskip=0.5cm,
framextopmargin=3pt,
framexbottommargin=3pt,
%framexleftmargin=0.4cm,
framesep=0pt,
rulesep=.4pt,
backgroundcolor=\color{gray97},
rulesepcolor=\color{black},
%
stringstyle=\ttfamily,
showstringspaces = false,
basicstyle=\small\ttfamily,
%\footnotesize\ttfamily,
%\small\ttfamily,
commentstyle=\color{gray45},
keywordstyle=\bfseries,
%
%numbers=left,
%numbersep=15pt,
%numberstyle=\tiny,
%numberfirstline = false,
breaklines=true,
tabsize=2,
}

% minimizar fragmentado de listados
\lstnewenvironment{listing}[1][]
{\lstset{#1}\pagebreak[0]}{\pagebreak[0]}

% estilo para poner comandos de consola
\lstdefinestyle{consola}
{basicstyle=\scriptsize\bf\ttfamily,
backgroundcolor=\color{gray75},
}

% estilo para código Java
\lstdefinestyle{Java}
{language=Java,
basicstyle=\footnotesize\ttfamily,
}

% estilo para código XML
\lstdefinestyle{XML}
{language=XML,
basicstyle=\footnotesize\ttfamily,
}

% estilo para código HTML
\lstdefinestyle{HTML}
{language=HTML,
basicstyle=\footnotesize\ttfamily,
}

% estilo para código YAML
\lstdefinestyle{YAML}
{
basicstyle=\footnotesize\ttfamily,
}


\usepackage[left=2.5cm,top=3.5cm,right=2.5cm,bottom=3cm]{geometry} 


\begin{document}
	%\maketitle	
	
	\pagenumbering{roman} % para comenzar la numeracion de paginas en numeros romanos
	%portada
	\includepdf{./primera.pdf}

	\newpage
	\mbox{}

	
	%acta
	\includepdf{./acta.pdf}
		
	\newpage
	\mbox{}


	\chapter*{}

	\begin{flushright}
	\textit{Dedicado a mi familia por todo el apoyo que he recibido \\y a mis amigos por aguantarme a lo largo de la carrera y el proyecto fin de carrera.}
	\end{flushright}
		
	%Table of contents	
	\tableofcontents
	
	%separacion entre parrafos
	\parskip=5mm
	
	\mainmatter
	%introducción
	\input{./Introduccion/introduccion.tex}
	%conocimientos previos sobre las tecnologías
	\chapter[Conocimientos previos]{Conocimientos previos sobre las tecnologías usadas.}\label{cap:conocimientos}
\markboth{CAPÍTULO \ref{cap:conocimientos}. CONOCIMIENTOS PREVIOS.}{}

En este segundo capítulo vamos a explicar cuáles son las ideas previas de las que partimos y los conocimientos que poseíamos sobre las tecnologías que hemos usado en el proyecto. Haciendo una breve explicación sobre su funcionamiento, su uso, su historia y sus recientes versiones.

El principal elemento usado en el proyecto es el lenguaje de programación Java, que lo hemos usado para programar tanto la aplicación web como la aplicación en el móvil. Pero aparte de los diferentes SDK de Android y de Google App Engine, hemos recurrido y empleado muchas otras tecnologías y lenguajes como pueden ser SQL, XML, UML, GIT. Además hemos aplicado los conocimientos básicos sobre criptografía de clave pública necesarios para realizar todo el proceso de firma digital.   

\section{El lenguaje de programación Java}

Java es un lenguaje de programación orientado a objetos que fue diseñado por James Gosling\footnote{ Para más información sobre James Gosling: \url{http://en.wikipedia.org/wiki/James\_Gosling}} para Sun Microsystems y que recientemente ha sido comprado por Oracle Corporation. Fue lanzado en 1995 y ha sido el centro de toda la plataforma Java de Sun Microsystems. Es un lenguaje con una sintaxis muy parecida a C o C++, pero con la gran ventaja de que el manejo de punteros y objetos es automático, al igual que la recogida de basura.

Java es un lenguaje en el que hay que compilar los códigos fuentes para crear unos archivos intermedios llamados bytecodes, los archivos *.class, que luego serán interpretados por la máquina virtual de Java (JVM). Esta dependerá de la arquitectura en la que se quiera ejecutar la aplicación Java. Gracias a esto se puede decir que Java es un lenguaje multiplataforma, lo que significa que un mismo código Java se puede ejecutar en Linux, en Windows, en Mac o cualquier otro sistema para el cual exista una máquina virtual, lo que en inglés se llama ``write once, run anywhere" (WORA). Además de esta importante ventaja Java es un lenguaje de propósito general, concurrente, basado en clases y orientado a objetos. Java es el segundo lenguaje de programación más popular de 2012, gracias a las aplicaciones web cliente-servidor que tienen tanto auge en estos momentos, como podemos ver en la figura \ref{fig:indicetiobe}.

\begin{figure}[h]
  \centering
    \includegraphics[scale=0.9]{./ConocimientosPrevios/imagenes/indiceTiobe.png}
  \caption{Índice Tiobe en septiembre del 2012. \url{http://www.tiobe.com/}}
  \label{fig:indicetiobe}
\end{figure} 

La implementación original y las referencias del compilador de Java, máquinas virtuales y las librerías de clases fueron desarrolladas por Sun en 1995, pero en el 2007 gracias a la contribución de la comunidad, Sun Microsystems cambió la licencia de todas las tecnologías Java a GNU General Public License, por lo que se abría la posibilidad de que se crearan versiones alternativas de compiladores bajo licencia GNU como por ejemplo GNU Compiler para Java o GNU Classpath.

En el proyecto la versión usada fue la versión \textbf{Java SE}.

\subsection{Historia}

Originalmente Java nació como un proyecto de James Gosling, Mike Sheridan y Patrick Naughton en 1991 y estaba diseñado para una televisión interactiva, pero era muy avanzado para lo que la industria televisiva de la época podía necesitar. En su origen fue llamado Oak, pero por problemas con el nombre, ya que era una marca registrada de otra empresa, lo cambiaron a Green y posteriormente ya lo renombraron al definitivo Java. Hay muchas teorías sobre por qué se llama Java, pero una de ellas es que había una cafetería llamada Java Coffe donde James, Mike and Patrick pasaron muchas horas consumiendo café.

La idea de Gosling era crear una máquina virtual donde funcionara un lenguaje de programación con la sintaxis y la estructura de C/C++ para que la curva de aprendizaje fuera muy suave para los programadores que en la época sabían C/C++.

Sun Microsystems lanzó Java 1.0 en 1995, con la principal característica de que una vez escrito un código fuente no había que modificarlo para que funcionara en las diferentes máquinas, lo que anteriormente hemos llamado con el acrónimo en inglés WORA (Write Once, Run Anywhere). Rápidamente todos los navegadores de la época empezaron a soportar applets Java en las páginas web, por lo que Java se volvió muy popular en la época. La nueva versión Java 2 fue lanzada en 1998-1999 y con ella llegaron las distinciones en diferentes plataformas, como por ejemplo Java2EE para aplicaciones corporativas o una versión ligera llamada Java2ME que estaba diseñada para funcionar en los diferentes teléfonos de la época, y el resto se agrupan en la versión Java2SE, que es la versión estándar.

En 1997, Sun Microsystems intentó formalizar Java mediante una norma ISO/IEC pero se retiró del proceso y dio todo el control a la comunidad. Sun ofrecia implementaciones gratuitas y generaba dinero vendiendo algunas licencias de productos como Java Enterprise System. Una cosa importante es que Sun distingue entre el SDK (Kit de desarrollo) y el JRE (Entorno de ejecución) en el que van incluidos los compiladores, debuggers, etc.

El 13 de noviembre del 2006, Sun lanzó Java gratis y como software libre, bajo la licencia GNU General Public License (GPL). El proceso finalizó el 8 de mayo del 2007.

En 2009-2010 Oracle Corporation compró Sun Microsystems por lo que Java actualmente pertenece a Oracle Corporation.

\subsection{Versiones}

\begin{itemize}

	\item \textbf{JDK 1.0} (23 de enero de 1996): Primer lanzamiento
	
	\item \textbf{JDK 1.1} (19 de febrero de 1997): Las primeras características añadidas fueron una reestructuración intensiva del modelo de eventos AWT (Abstract Windowing Toolkit), clases internas (inner classes), JavaBeans, JDBC (Java Database Connectivity), para la integración de bases de datos y RMI (Remote Method Invocation).
	
     \item \textbf{JDK 1.2}(8 de diciembre de 1998): Recibió el nombre en clave Playground. Esta y las siguientes versiones fueron recogidas bajo la denominación Java 2 y el nombre ``J2SE" (Java 2 Platform, Standard Edition), reemplazó a JDK para distinguir la plataforma base de J2EE (Java 2 Platform, Enterprise Edition) y J2ME (Java 2 Platform, Micro Edition). 
    Se añadieron las siguientes mejoras, la palabra reservada strictfp, reflexión en la programación, la API gráfica (Swing) fue integrada en las clases básicas, la máquina virtual (JVM) de Sun fue equipada con un compilador JIT (Just in Time) por primera vez, Java Plug-in, Java IDL, una implementación de IDL (Lenguaje de Descripción de Interfaz) para la interoperabilidad con CORBA y Colecciones.

    \item \textbf{J2SE 1.3} (8 de mayo de 2000): Recibió el nombre en clave Kestrel. Los cambios más notables fueron: la inclusión de la máquina virtual HotSpot JVM, RMI fue cambiado para que se basara en CORBA, JavaSound, se incluyó el Java Naming and Directory Interface (JNDI) en el paquete de bibliotecas principales (anteriormente disponible como una extensión), Java Platform Debugger Architecture (JPDA).

    \item \textbf{J2SE 1.4} (6 de febrero de 2002): Recibió el nombre en clave Merlin. Este fue el primer lanzamiento de la plataforma Java desarrollado bajo el Proceso de la Comunidad Java como JSR 59. Las principales características que se le añadieron fueron palabra reservada assert, expresiones regulares modeladas al estilo de las expresiones regulares Perl, encadenación de excepciones, non-blocking NIO (New Input/Output), logging API, API I/O para la lectura y escritura de imágenes en formatos como JPEG o PNG, parser XML integrado y procesador XSLT (JAXP), seguridad integrada y extensiones criptográficas (JCE, JSSE, JAAS), Java Web Start incluido.
    
    \item \textbf{J2SE 5.0} (30 de septiembre de 2004): Recibió el nombre en clave Tiger. Estos fueron los cambios más importantes, plantillas (genéricos), metadatos, también llamados anotaciones, permite a estructuras del lenguaje, como las clases o los métodos, ser etiquetados con datos adicionales que pueden ser procesados posteriormente por utilidades de proceso de metadatos, autoboxing/unboxing, conversiones automáticas entre tipos primitivos (Como los int) y clases de envoltura primitivas (Como Integer), enumeraciones, varargs (número de argumentos variable), el último parámetro de un método puede ser declarado con el nombre del tipo seguido por tres puntos (por ejemplo \lstinline{void drawtext(String... lines)}). En la llamada al método, puede usarse cualquier número de parámetros de ese tipo, que serán almacenados en un array para pasarlos al método, bucle for mejorado, la sintaxis para el bucle for se ha extendido con una sintaxis especial para iterar sobre cada miembro de un array o sobre cualquier clase que implemente Iterable, como la clase estándar Collection, de la siguiente forma:

\begin{lstlisting}[style=Java]
void displayWidgets (Iterable<Widget> widgets) {
	for (Widget w : widgets) {
		w.display();
	}
}
\end{lstlisting}

    \item \textbf{Java SE 6} (11 de diciembre de 2006): Recibió el nombre en clave Mustang. En esta versión, Sun cambió el nombre ``J2SE" por Java SE y eliminó el ``.0" del número de versión. Los cambios más importantes introducidos en esta versión son un nuevo marco de trabajo y APIS que hacían posible la combinación de Java con lenguajes dinámicos como PHP, Python, Ruby y JavaScript, el motor Rhino, de Mozilla, una implementación de Javascript en Java, un cliente completo de Servicios Web y soporta las últimas especificaciones para Servicios Web, mejoras en la interfaz gráfica y en el rendimiento.
    
    \item \textbf{Java SE 7} (Julio 2011): Su nombre en clave es Dolphin. Y las principales nuevas características son: soporte para XML dentro del propio lenguaje, un nuevo concepto de superpaquete, soporte para closures, e introducción de anotaciones estándar para detectar fallos en el software.

\end{itemize}

\section{El entorno de programación Eclipse.}

Eclipse es un entorno integral de desarrollo que consta de un entorno de desarrollo integrado (IDE) y es extensible mediante plugins que están escritos en Java. Puede ser usado para una larga lista de lenguajes de programación como pueden ser C, C++, Haskell, Perl, PHP, Python, Android y un largo etcétera. Fue originalmente desarrollado por IBM y fue lanzado con la licencia de software Eclipse Public License\footnote{ Para más información visite: \url{http://en.wikipedia.org/wiki/Eclipse\_Public\_License}} la cual es una licencia de software libre. El SDK de Eclipse es libre y tiene licencia Open Source por lo que cualquier persona con los conocimientos necesarios puede programar el plugin que necesite para Eclipse. Fue el primer entorno de programación que funcionó bajo GNU Classpath y que funcionaba sin problemas con IcedTea. En la figura \ref{fig:pantallaEclipse} se puede ver el aspecto que tiene.

\begin{figure}
  \centering
    \includegraphics[scale=0.5]{./ConocimientosPrevios/imagenes/pantallaEclipse.png}
  \caption{Eclipse 4.2 Juno.}
  \label{fig:pantallaEclipse}
\end{figure} 

En el proyecto hemos usado la versión \textbf{Indigo}, que equivale a la versión 3.7 de Eclipse.

\subsection{Historia}

Eclipse comenzó como un proyecto de IBM Canadá. En noviembre de 2001 se creó un grupo de empresas para promover el desarrollo de Eclipse como software libre, los miembros iniciales eran Borland, IBM, Merant, QNX Software Systems, Rational Software, Red Hat, SuSE, TogetherSoft and WebGain. Finalmente en enero de 2004 se creó la Eclipse Foundation. 

\subsection{Versiones}

\begin{itemize}

	\item \textbf{Versión 3.0} (21 de junio de 2004)
	
	\item \textbf{Versión 3.1} (28 de junio de 2005)
	
	\item \textbf{Versión 3.2} (30 de junio de 2006): recibió el nombre de Callisto.
	
	\item \textbf{Versión 3.3} (29 de junio de 2007): recibió el nombre de Europa.
	
	\item \textbf{Versión 3.4} (25 de junio de 2008): recibió el nombre de Ganymede.
	
	\item \textbf{Versión 3.5} (24 de junio de 2009): recibió el nombre de Galileo.
	
	\item \textbf{Versión 3.6} (23 de junio de 2010): recibió el nombre de Helios.
	
	\item \textbf{Versión 3.7} (22 de junio de 2011): recibió el nombre de Indigo.
	
	\item \textbf{Versión 4.2} (27 de junio de 2012): recibió el nombre de Juno.
	
	\item \textbf{Versión 4.3} (26 de junio de 2013): esta será la próxima versión, que saldrá el próximo año y recibirá el nombre de Kepler.

\end{itemize}

\section{Criptografía.}\label{lbl:criptografia}

La criptografía es la ciencia que se encarga del estudio y creación de técnicas para la protección de una comunicación, para que solamente los usuarios autorizados puedan verla, leerla y entenderla. En la actualidad la criptografía es un término que se usa de forma similar a encriptación, que es el proceso para transformar una información mediante diferentes algoritmos, en un mensaje que no pueda entender un atacante que intercepte una comunicación. 

En el proyecto hemos usado una criptografía llamada \textbf{Criptografía de Clave Pública}, que como veremos a continuación en la historia brevemente y posteriormente en el capítulo~\ref{cap:criptografia}, en el que se explicará la criptografía usada a lo largo del proyecto con mas profundidad, consta de dos claves, ambas enlazadas matemáticamente y si conocemos una no podremos de ninguna forma conseguir la otra. La pública es la que tendría la persona que quiera desencriptar el mensaje, que a su vez da nombre a este algoritmo y otra privada que sólo conocerá la persona que quiere encriptar el mensaje.

La criptografía ha evolucionado mucho y actualmente no solo se usan para proteger mensajes, si no que también se usa para proteger la integridad de ellos. Este es uno de los usos más común de la criptografía de clave pública.

\subsection{Historia}

Podemos hacer dos grandes grupos dentro de la historia de la criptografía, la criptografía clásica y la criptografía durante la época de los ordenadores.

\subsection{Criptografía clásica.}

	Durante la época de la criptografía clásica sólo se quería proteger el mensaje que se enviaba de la mirada de curiosos y enemigos por lo que únicamente existían algoritmos de encriptación, la integridad del mensaje no importaba en esa época.  
	
	En dicha época todos los algoritmos de cifrados que existían eran por transposición o sustitución de caracteres. A continuación exponer unos ejemplos de los algoritmos utilizados más famosos. 
\begin{itemize}

	\item \textbf{Cifrado Cesar:} dicho cifrado es famoso porque los usaban las centurias romanas para comunicarse entre ellas de manera que si un mensaje era interceptado no pudiera ser leído. Consiste en sustituir cada carácter del mensaje por el que hay tres lugares a la derecha. Por ejemplo si tenemos el mensaje ``Hola" si lo ciframos con este sistema conseguimos ``Krod", en la figura \ref{fig:cifradoCesar} podemos ver como es el cifrado.
	
	Para desencriptar solo habría que intercambiar por la tercera letra anterior.

\begin{figure}[h]
  \centering
    \includegraphics[scale=0.6]{./ConocimientosPrevios/imagenes/cifradoCesar.png}
  \caption{Ejemplo Cifrado Cesar}
  \label{fig:cifradoCesar}
\end{figure} 

	\item \textbf{Cifrado Homofónico:} Es una evolución del siguiente, pero en vez de sustituir siempre por el mismo carácter lo que se hace es tener la posibilidad de poder realizar varios cambios posibles, por lo que un mismo mensaje podría generar varios textos cifrados, complicando así su desencriptación. En la figura \ref{fig:cifradoHomofonico} podemos ver una tabla sencilla de sustitución para realizar el cifrado. Por ejemplo si ciframos la palabra ``PLATON" nos daría de resultado ``882110772963", pero podríamos sustituir la P no solo por 88 si no por cualquier valor de la tabla dando lugar a que pudiéramos crear varios mensajes cifrados.
	
\begin{figure}[h]
  \centering
    \includegraphics[scale=0.7]{./ConocimientosPrevios/imagenes/cifradoHomofonico.png}
  \caption{Tabla para cifrado homofónico}
  \label{fig:cifradoHomofonico}
\end{figure}

	\item \textbf{Cifrado por Transposición:} consiste en realizar una permutación de las posiciones que ocupan las letras escritas, un ejemplo podría ser escribir todo el texto con una cierta longitud preestablecida y luego leerlo por columnas en vez de por filas. En la figura \ref{fig:cifradoTransposicion} podemos ver un ejemplo del mecanismo de cifrado.  

\begin{figure}[h]
  \centering
    \includegraphics[scale=0.4]{./ConocimientosPrevios/imagenes/cifradoTransposicion.png}
  \caption{Ejemplo de cifrado por sustitución}
  \label{fig:cifradoTransposicion}
\end{figure}	

	\item \textbf{Cifrado Producto:} Es un cifrado que combina sustitución y transposición y se puede considerar como un encadenamiento de varios cifrados. Esto da lugar a cifrados complejos, seguros y difíciles de atacar, ya que tendríamos que averiguar no sólo el método de cifrado utilizado, sino que también tendríamos que saber el orden en el que se ha ejecutado las encriptaciones.

	\item \textbf{Cifrado Vernam:} es un tipo de cifrado que se denomina cifrado de flujo. El texto en claro se combina con una cadena, del mismo tamaño del texto en claro, de número aleatorios o pseudoaleatorio por medio de la función XOR. Lo inventó Gilbert Vernam que era un ingeniero de AT\&T en 1917. Es también conocido como RC4 en internet.  
	
\end{itemize}

\subsection{Criptografía durante la época de los ordenadores.}

La criptografía dio un gran salto en cuanto a calidad en el momento en el que se empezaron a usar ordenadores para encriptar y desencriptar textos, debido a que los ordenadores son máquinas que las tareas repetitivas las hacen muy bien y muy rápidos.

Se empezaron a idear nuevos algoritmos de cifrado mucho más complejos, los cuales se pueden dividir en dos grandes grupos, la criptografía de clave simétrica y la criptografía de clave pública. A continuación vamos a explicar brevemente los algoritmos más famosos de ambos.

\subsubsection*{Criptografía de Clave simétrica}

La principal característica de esta técnica de criptografía es que usan la misma clave para encriptar y desencriptar.

\begin{itemize}

	\item \textbf{Data Encryption Standard (DES):} Fue presentado por IBM en 1974, para generar un estandar de cifrado para transmisión de datos y cifrado de almacenamiento de datos y que fuera usado por gobiernos, empresas privadas o cualquier usuario. IBM comenzó el desarrollo basándose en un dispositivo de cifrado llamado Lucifer el cual tenia una clave de 128 bits. DES es un criptosistema de clave secreta que cifra en bloques de 64 bits del texto en claro y genera otros bloques de 64 bits del texto cifrado. La clave utilizada también es de 64 bits, pero el bit final de cada octeto de los 64 bits de la clave se usa como bit de paridad para control de errores. El cifrado se realiza en 16 iteraciones en las que se usan varias operaciones como son operaciones XOR, permutaciones y sustituciones. El esquema para cifrar se puede ver en la figura \ref{fig:cifradoDes}.
	
\begin{figure}
  \centering
    \includegraphics{./ConocimientosPrevios/imagenes/cifradoDes.png}
  \caption{Iteraciones en el cifrado DES}
  \label{fig:cifradoDes}
\end{figure}

\item \textbf{AES (Rijndael):} Fue presentado al concurso AES el 2 de enero de 1997 y anunciado ganador en 2001. Fue diseñado por dos criptólogos llamados Joan Daemen y Vincent Rijmen, ambos estudiantes de la Katholieke Universiteit Leuven de Bélgica. Al contrario que DES, AES es una red de sustituciones y permutaciones no una red de Feistel, se transformó en estándar efectivo el 26 de mayo de 2002 y en la actualidad es uno de los algoritmos de encriptación más famosos. Opera con bloques de 128 bits y tiene claves de 128, 192 y 256 bits.

\end{itemize}

El mayor problema que tiene este tipo de criptografía es que para que el destinatario pueda leer el mensaje necesita saber la clave y el intercambio de clave puede ser una dificultad muy grande si ambos usuarios no se pueden comunicar directamente, ya que usando cualquier otro método la clave podría ser interceptada y todo el proceso de encriptación sería inutil.

\subsubsection*{Criptografía de Clave Pública}

La criptografía de clave pública fue inventada por Diffie y Hellman y paralelamente por Merkle y ambos grupos aportaron a la criptografía el concepto de la utilización de pares de claves.

La característica principal es que cada usuario posee dos claves, una privada que sólo conoce el dueño de la clave y será usada para descifrar todo lo que otros usuarios cifren con su otra clave y otra la clave pública que es conocida por el resto de usuarios y será la que estos usarán para encriptar el mensaje que queremos que sea secreto. Así de esta forma si una persona quiere comunicarse con otra de forma secreta sólo tiene que conocer su clave pública, cifrar con ella y el destinatario podrá descifrar el mensaje con su clave privada. 

Otra característica es que las claves son imposibles de deducir una a partir de la otra, ambas claves son de una gran longitud y son generadas mediante exponenciación y/o productos de números primos grandes.

En los primeros años de existencia de la criptografía de clave pública se inventaron tres sistemas, Algoritmo de la mochila de Merkle-Hellman que fue roto, el esquema de McEliece que está considerado imposible de llevar a la práctica y un tercero que es el que explicaremos a continuación llamado RSA cuyo uso de ha impuesto actualmente.

\begin{itemize}

	\item \textbf{RSA:} Su nombre proviene de sus creadores que son Rivest, Shamir y Adleman y se basaron en la idea: \textit{``es muy fácil multiplicar dos números enteros primos grandes, pero extremadamente difícil hallar la factorización del producto"}, cuando inventaron el RSA en 1997. Es un algoritmo exponencial. Una característica de RSA es que tanto el mensaje, como el texto cifrado tienen que ser un código decimal, por lo que se tendría que usar el valor ASCII de la letra por ejemplo. Un ejemplo de uso sería el siguiente, lo primero que se debe de hacer antes de enviar el mensaje es acordar el algoritmo que se va a usar, lo siguiente el emisor cifra el mensaje usando la clave pública del receptor y se lo envía. Acto seguido el receptor descifra el mensaje que ha enviado el receptor usando su propia clave privada. La gran ventaja de este método es que en ningún momento la clave privada se tiene que enviar, por lo que solucionamos el gran problema que dijimos que tenían los algoritmos de cifrado simétrico, que antes de nada había que intercambiar la clave con la vulnerabilidad que eso implicaba. 

\end{itemize}

El algoritmo RSA será explicado con más profundidad en el capítulo~\ref{cap:criptografia}.

Los algoritmos de clave pública tienen un gran problema, es que son muy lentos realizando el proceso de cifrado y descifrado, por lo que en la situaciones reales se usan para realizar el intercambio de claves de algoritmos de clave simétrica que son mucho más rápidos y también igual de seguros, de esta forma solucionamos su principal problema.

\section{Android.}

Android es un sistema operativo basado en Linux especialmente diseñado para smartphone, tablet, smart TV y una infinidad de dispositivos, desarrollado por Google con Open Handset Alliance. Android empezó siendo desarrollado por la compañía llamada Android que inicialemente fue financiada y después comprada por Google en 2005. En 2007 cuando se presentó por primera vez Android también se anunció la fundación de Open Handset Alliance que es un conjunto de 86 empresas, entre las que hay compañías de hardware, software y telecomunicaciones, interesadas en el mundo de los dispositivos móviles. Android es código abierto y está distribuido bajo licencia Apache\footnote{ Para saber más sobre la licencia visite: \url{http://en.wikipedia.org/wiki/Apache\_License}}. La tarea del mantenimiento y desarrollo de Android es de Android Open Source Project (AOSP).

Android tiene una gran comunidad de desarrolladores que pueden extender las funcionalidad de los teléfonos o de cualquier dispositivos que pueda ejecutar la máquina virtual de Android, se puede desarrollar tanto en Java usando el SDK o en C++ usando el NDK, posee una tienda online llamada Google Play (anteriormente Android Market), donde se pueden comprar aplicaciones, películas, libros o música y en la que cualquier desarrollador por una pequeña cantidad de dinero (alrededor de 25\euro, por una cuenta vitalicia de desarrollador) puede subir todas las aplicaciones gratuitas o de pago que desee. En Junio de 2012 había alrededor de 600.000 aplicaciones en Google Play.

En el primer cuatrimestre de 2012, Android tenía el 59\% del mercado de smartphones en el mundo, de ahí la importancia de esta plataforma para los desarrolladores, ya que proporciona un mercado muy amplio y una forma muy fácil y barata de conseguir un gran número de usuarios.

Los detalles técnicos de Android se explicarán con más profundidad en el capítulo~\ref{cap:android}.

\subsection{Historia.}

Como hemos dicho anteriormente Android fue diseñado y creado originalmente por una compañía llamada Android que fue fundada en Palo Alto, California en 2003 por Andy Rubin, Rich Miner, Nick Sears y Chris White. Originalmente solo estaba diseñado para funcionar con smartphones, ya que ellos pensaban que un smartphone era algo más que un dispositivo que sirviera para usar el GPS y tener preferencias. 

Google compró Android el 17 de agosto de 2005, con la intención de entrar en el mercado de los teléfonos móviles. Después de varios años de rumores el 5 de noviembre de 2007, Google presentó la Open Handset Alliance, un grupo de empresas que incluian a Broadcom Corporation, Google, HTC, Intel, LG, Marvell Technology Group, Motorola, Nvidia, Qualcomm, Samsung Electronics, Sprint Nextel, T-Mobile and Texas Instruments entre otras muchas empresas que estaban interesadas en generar estándares para dispositivos móviles. Ese mismo día también se lanzó el primer producto Android basado en el kernell de Linux 2.6.

Android ha sido muy criticado por la gran fragmentación que tiene debido al gran número de versiones que posee, que son compatibles hacia versiones abajo pero no hacia versiones posteriores, esto quiere decir que la versión 4.0 es compatible con todo el software que funcionase con las versiones 2.0, 2.3 o cualquiera inferior, pero no será compatibles con el software diseñado para la versión 4.1. Este problema hace necesario que los fabricantes actualicen el software de sus teléfono, lo que es un gran problema debido a que muchos no lo hacen, hubo un pequeño intento de solucionar esta problemática haciendo que los fabricantes estuvieran obligados a actualizar sus terminales al menos en los 18 meses posteriores a la salida al mercado, pero no hubo ningún acuerdo.   

\subsection{Versiones.}

Como curiosidad todas las versiones de Android se denominan con un nombre en clave que es un postre.
\begin{itemize}

	\item \textbf{1.0 (Apple Pie):} primera versión lanzada el 23 de septiembre del 2008.
	
	\item \textbf{1.1 (Banana Bread):} lanzada el 9 de febrero del 2009.
	
	\item \textbf{1.5 (Cupcake):} fue presentada el 30 de abril del 2009, esta fue la primera versión con la que Android empezó a despuntar y entrar en el mundo de los teléfonos móviles, anteriormente apenas si era conocido. Tenía características nuevas muy interesantes como poder grabar y reproducir vídeo, podía subir videos a Youtube e imágenes a Picasa directamente desde el teléfono, un nuevo teclado predictivo, nuevos widget y carpetas para colocar en la pantalla de inicio y transiciones animadas.

	\item \textbf{1.6 (Donut):} fue presentada el 15 de septiembre de 2009. Se le añadieron las siguientes características nuevas como una interfaz integrada para la cámara, la grabadora de vídeo y la galería, se actualizó la búsqueda por voz añadiendo soporte a más aplicaciones nativas y la posibilidad de llamar a contactos, se añadió un buscador general en la pantalla de inicio donde se podía buscar contactos, historiales y páginas web, se añadió un nuevo framework de gestos y las herramientas de desarrollo llamado GestureBuilder.

	\item \textbf{2.0 / 2.1 (Eclair):} la versión 2.0 fue presentada el 26 de octubre de 2009 y la 2.1 fue liberada el 3 de diciembre del 2009. Se añadieron un gran número de mejoras, se optimizó la velocidad de hardware, se soportaron más tamaño de pantallas y resoluciones, se rediseñó la interfaz de usuario, el navegador también fue renovado y se le añadió soporte para HTML5, nueva lista de contactos, se añadió soporte para el flash de la cámara, zoom digital, soporte para bluetooth 2.1, se mejoraron la captura de eventos multi-touch con MotionEvent y fondos de pantalla animados.
	
	\item \textbf{2.2 (Froyo):} fue lanzada el 20 de mayo de 2010. Se optimizó el sistema Android, la memoria y el rendimiento, se mejoró la velocidad de las aplicaciones gracias a la implementación de JIT, se implementó el motor JavaScript V8 de Google Chrome en el navegador del móvil, nueva funcionalidad de WiFi hotspot y tethering por USB, se actualizó el Android market para que tuviera actualizaciones automáticas, marcación por voz y compartir contactos por Bluetooth, soporte para contraseñas numéricas y alfanuméricas, soporte para Adobe Flash 10 y soporte para pantallas de HDPI, como pueden ser pantallas de 4" y resolución de 720p. 

	\item \textbf{2.3 (Gingerbread):} fue presentado el 6 de diciembre del 2010. Cambiaron el diseño de la interfaz de usuario, añadieron soporte para pantallas extra grandes y resoluciones WXGA, soporte nativo para VoIP SIP, reproducción nativa de vídeos WebM/VP8 un formato de vídeo patrocinado por Google que es la alternativa al H264 en la reproducción de vídeo en HTML5 y decodificación de audios en AAC, se añadió soporte a NFC (Near Field Communication), nuevo teclado multitáctil, soporte mejorado para programar en código nativo, soporte nativo de más sensores como pueden ser acelerómetros o barómetros, soporte para múltiples cámaras y cambio del sistema de archivos YAFFS a ext4. La versión 2.3.3 sigue siendo la versión de Android más usada actualmente. 

	\item \textbf{3.0 / 3.1 / 3.2 (Honeycomb):} Esta versión fue diseñada exclusivamente para tablet, por lo que no hubo smartphones que actualizaran a esta versión. Las características principales fueron un escritorio en 3D con widget rediseñado, sistema multitarea mejorado, mejoras en el navegador de internet, videochat mediante Google Talk, mejoras en el soporte de redes WiFi, añadidos soporte para gran cantidad de periféricos y conexión USB.
	
	\item \textbf{4.0 (Ice Cream Sandwich):} Fue una de las actualizaciones más importantes que ha recibido Android y fue lanzada el 19 de octubre de 2011, en ella se unificaron todas las versiones y se tenía una sola versión para smartphne, televisores, tablets, netbooks, etc. Se añadió una nueva versión de interfaz mucho más limpia y usable llamada Holo, una nueva fuente llamada Roboto, se da la opción de utilizar botones virtuales en la interfaz de usuario en vez de botones físicos, soporte para aceleración gráfica por hardware, por lo que la interfaz es manejada y dibujada por la GPU, aumentando notablemente el rendimiento, multitarea mejorada, se ha añadido un nuevo corrector ortográfico, en la lista de notificaciones se pueden eliminar las que no sean interesantes, capturas de pantalla pulsando el botón de encendido y el de bajar volumen, mejorada la aplicación encargada de hacer fotografías, añadida una nueva opción para crear fotos panorámicas, Android Beam, una nueva característica que nos permite compartir contenidos entre teléfonos mediante NFC, reconocimiento de voz del usuario, reconocimiento facial, para bloqueo y desbloqueo del teléfono, añadidas nuevas carpetas que se crean sólo con arrastrar y soltar, un único y nuevo framework para crear aplicaciones y soporte para contenedor MKV.

	\item \textbf{4.1 (Jelly Bean):} Esta es la última versión de Android que hay en el mercado, fue lanzada el 27 de junio de 2012 durantes la última Google I/O. Se mejoró la fluidez y la estabilidad gracias al proyecto ``Project Butter", ajuste automático de widget cuando se añaden al escritorio, se añadió soporte para lenguas no occidentales, mejora de Android Beam para poder enviar video por NFC, dictado de voz mejorada y sin tener que tener conexión a internet para usarlo, nuevas notificaciones en las que se puede añadir botones para controlar o tener acceso a opciones más comunes, como puede ser responder a un email, pulsar pause o pasar de canción, nueva función Google Now que intenta ser el competidor de SIRI del iPhone en Android, cifrado de aplicaciones y nuevas actualizaciones incrementales, en las que no es necesario volver a bajar toda la aplicación para actualizarla, sólo se baja las partes nuevas, Google Chrome se convierte en el navegador por defecto de Android y se pone fin al soporte de Adoble Flash Player, se añade una nueva función llamada Sound Search que permite identificar la canción que está sonando, se ha añadido una nueva función llamada Gestual Mode para personas discapacitadas visualmente.
\end{itemize}

En el proyecto se ha usado la versión \textbf{Android 4.0} para el desarrollo.

En la figura \ref{fig:Android41} se puede ver como es visualemte Android en la actualidad, con la versión 4.1.

\begin{figure}
  \centering
    \includegraphics[scale=0.2]{./ConocimientosPrevios/imagenes/android41.jpeg}
  \caption{Versión de Android 4.1}
  \label{fig:Android41}
\end{figure}


\section{Google App Engine.}

Google App Engine es una plataforma de cloud computing para desarrollar y almacenar aplicaciones web que ofrece Google. Las aplicaciones web pueden ser escalables y si necesitan más recursos automáticamente le son asignados para poder seguir ofreciendo servicio. Google App Engine es gratis para un cierto número de peticiones y almacenamiento y la primera versión fue lanzada en abril de 2008.

Actualmente se puede desarrollar en tres lenguajes, que son Java, Python y Go, este último un lenguaje creado por Google. En el proyecto hemos usado Java para desarrollar en la plataforma. 

Google App Engine para Java soporta muchos estándares y framework, y el Core está hecho con la tecnología Servlet 2.5 usando el servidor de software libre llamado Jetty Web Server, acompañado con otras tecnologías como JSP. El almacén de datos puede ser muy poco intuitivo desde el punto de vista de los desarrolladores, pero se puede acceder fácilmente con JPA (Java Persistence API) y los métodos JDO (Java Data Objects) para escritura y lectura de datos. También se pueden usar tecnologías como Spring Framework.

Google garantiza que aplicación estará disponible el 99.95\% del tiempo, para ello ofrece una alta replicación.

La base de datos a la que nos dan acceso no es una base de datos estándar, como pueden ser MySQL, Oracle o SQLServer, pero tiene una sintaxis muy parecida a SQL, llamada GQL. Una de las principales diferencias es que no admite sentencias join, debido a la ineficiencia de la misma. La versión de Java soporta consultas asíncronas no bloqueantes, ofreciendo así una forma de procesamiento paralelo de datos.

Para más información se puede visitar la web diseñada para desarrolladores que proporciona Google, \url{https://developers.google.com/appengine/}. 

En el capítulo \ref{cap:GAE} veremos más ampliamente todo lo relacionado con Google App Engine que hemos usado en el proyecto.

\section{SQLite.}

SQLite es un sistema gestión de base de datos relacionales compatibles con ACID, ACID es el acrónimo de Atomicity, Consistency, Isolation and Durability, que son las características que debe tener una base de datos para que se consideren base de datos relacional. Su característica principal es que ocupa muy poco espacio, alrededor de 275 Kb y fue escrita en el lenguaje C por Richard Hipp. Está distribuida bajo licencia de dominio público.

A diferencia de los sistemas de gestión de base de datos clientes-servidor, el motor de SQLite pasa a ser parte del programa que quiere usarlo, ya que se integra con él. Esto hace que tenga mayor rendimiento debido a que la comunicación es por medio de funciones, que es mucho más eficiente que mediante comunicación de procesos. La totalidad de la base de datos, tablas, índices y datos, se guardan en un solo fichero estándar en la máquina host. La versión 3 de SQLite permite base de datos de hasta 2 Terabytes y permite campos del tipo BLOB.

SQLite está muy extendido y se puede programar en infinidad de lenguajes de programación como pueden ser C, C++, Perl, Python, PHP, Java, etc.

Es utilizado en infinidad de programas y sistemas, que van desde editores de imagen como puede ser Adobe Photoshop Elements, reproductores de sonido como Clementine o navegadores como Firefox, Chrome u Opera.

Esta es una de las tres formas que proporciona Android para guardar datos, al estar embebida en cada aplicación para mejorar el rendimiento, cada aplicación debe de tener una. 

\section{XML.}

XML son las siglas en inglés de e\textbf{X}tensible \textbf{M}arkup \textbf{L}anguage que es un lenguaje de marcas desarrollado por el World Wide Web Consortium (W3C), deriva del lenguaje SGML y permite definir la gramática de lenguajes específicos para estructurar documentos grandes.

Es una de las formas de intercambio estructurado de información más extendidas en internet, ya que se puede usar en base de datos, hojas de cálculo o en casi cualquier información que se quiera usar.

XML es un lenguaje que puede ser analizado sintácticamente para averiguar si está bien construido o no, por lo que cualquier parser (analizador sintáctico) puede confirmar si tiene la estructura bien definida según el estándar. Todos los documentos tiene que tener las siguientes partes: prólogo, cuerpo, elementos y atributos.

Un ejemplo podemos verlo en el siguiente un trozo de código XML, para almacenar un libro en una librería.

\begin{lstlisting}[language=XML]   
<?xml version="1.0"?>
<libro>
<titulo> A Game of Thrones </titulo>
<disponible tiempo="24" unidad="horas"/>
<autor> George R. R. Martin </autor>
<formato> Rustica </formato>
<publicacion> 1996 </publicacion>
<precio cantidad="9.99" moneda="euro"/>
<descuento cantidad="5"/>
<enlacelibro href="/exec/ISBN/0-553-10354-7"/>
</libro>
\end{lstlisting}

En el proyecto hemos usado XML, en los archivos de configuración o para el diseño de las interfaces en Android. En una de la aplicaciones web realizadas para los archivos de configuración también hemos usado XML, en la otra hemos usado un lenguaje parecido llamado YALM, que es equivalente a XML.

\section{UML.}

UML es un lenguaje de modelado de propósito general más usado en la actualidad para el diseño de software. UML son las siglas de Unified Modeling Language. UML tiene la ventaja de que se puede observar visualmente el diseño del software. Se puede desde especificar, construir o documentar un sistema o un software.

\begin{figure}
  \centering
    \includegraphics[scale=0.4]{./ConocimientosPrevios/imagenes/UMLDiagrams.jpg}
  \caption{Ejemplos de diseños realizados con UML.}
  \label{fig:UMLDiagrams}
\end{figure}

En la figura~\ref{fig:UMLDiagrams} podemos ver los diferentes diagramas que podemos realizar mediantes UML.

Hemos usado UML para diseñar las clases usadas a lo largo de todo el proyecto, además de para mostrar en la memoria del proyecto la relación de las clases, los casos de uso, etc.

\section{GIT.}

Git es un software de control de versiones, a la vez que mantenedor de la coherencia y cohesión del código fuente orientado a la velocidad. Fue desarrollado para el manejo del código fuente de Linux y al principio fue diseñado por Linus Torvalds. GIT es un repositorio con un completo historial y una capacidad de identificación de cada cambio realizado sin depender del acceso a la red o a un servidor central. Está liberado con licencia GNU versión 2.

En el proyecto hemos estado usando GIT como control de versiones, ya que si en algún cambio ocurría algún problema poder volver atrás y para tener una copia de seguridad del proyecto, alojado en todo momento en un servidor externo por si ocurría algún problema en el ordenador donde desarrollamos el proyecto.

A lo largo del proyecto hemos usado dos servicios gratuitos  que proporcionan servidores GIT, como pueden ser \url{https://github.com/} y \url{https://bitbucket.org/}. El primero es muy conocido actualmente y hay una gran comunidad de software libre en dicha plataforma, tiene un pequeño problema para nuestro proyecto y es que el código tiene que ser libre y en la versión gratuita no se pueden tener repositorios privados, cosa que el segundo servicio si te lo ofrece. Para todo el código fuente usando, tanto en la aplicación de Android como en las dos aplicaciones hemos usado bitbucket y para la memoria de proyecto hemos usado github. La dirección del repositorio de la memoria es la siguiente: \url{https://github.com/t321/memoriaPFC}, el repositorio de las aplicaciones web \url{https://bitbucket.org/t321/pfcaeg} y el de la aplicación de Android \url{https://bitbucket.org/t321/pfcandroid}.

La configuración básica para el uso de GIT con Eclipse se puede ver en el apendice~\ref{cap:apendiceA}.  















	%criptografía usada en el proyecto
	\input{./Criptografia/criptografia.tex}
	%lo relacionado con Android y Google App Engine
	\chapter{Android y Google App Engine.}\label{cap:androidYGAE}
\markboth{CAPÍTULO \ref{cap:androidYGAE}. ANDROID Y GOOGLE APP ENGINE.}{}

En este capítulo vamos a explicar más detenidamente las dos principales plataformas usadas durante el proyecto, Android en la parte móvil y Google App Engine en la parte de las aplicaciones web. 

\section{Android}\label{cap:android}

\subsection{Introducción.}

Android es un sistema operativo basado en Linux, libre y multiplataforma. Inicialmente empezó como un sistema operativo solo para móviles pero con el tiempo ya podemos encontrarlo en móviles, tablets, pc, neveras, relojes, cámaras de fotos y una gran cantidad de aparatos.

En el proyecto lo usaremos para diseñar y desarrollar una aplicación móvil, con la que poder firmar digitálmente un texto leido previamente mediante un lector de códigos QR.

Android es propiedad de Google actualmente y es el encargado de dar soporte y ayudar a los desarrolladores. Esta función la realiza muy bien, dando una muy buena API y una gran documentación. Podemos encontrar toda la información que podemos necesitar para desarrollar una aplicación en la siguiente web, \url{http://developer.android.com/index.html}. En la anterior web podemos encontrar la documentación de la API, consejos de diseño para que la aplicación tenga un aspecto bonito a la vez que usable, todas las novedades incluidas en las versiones nuevas del sistema operativo, etc.

\subsection{Arquitectura de la plataforma Android.}

Como hemos dicho anteriormente Android es una plataforma que engloba desde el sistema operativo, al software intermedio que comunica el sistema operativo y las aplicaciones, llamado en inglés middleware y las posibilidad de hacer funcionar las aplicaciones en la plataforma elegida, ya sea un telefono, un tablet o cualquier aparato con Android.

Para los desarrolladores Android proporciona dos kit de desarrollo, uno que usa la tecnología Java (SDK), el que hemos usado en el proyecto y otro que da la posibilidad de programar a más bajo nivel (NDK), este último desarrollado en C++.

El SDK de Android proporciona ayuda para las siguientes características, un navegador basado en WebKit, graficos optimizados en 2D, gráficos en 3D basados en OpenGL ES 1.0 con aceleración gráfica, una base de datos para almacenar datos que necesitemos, llamada SQLite, soporte para ficheros gráficos (JPG, PNG, GIF, etc), vídeo (MPEG4, H.264) y audio (MP3, AAC), telefonía GSM, tecnologías inalámbricas como son Bluetooth, 3G, Wifi, uso de la cámara, GPS, brújulas, etc. Además de todo esto, proporciona ayuda en la reutilización y remplazo de componentes, una máquina virtual optimizada para dispositivos móviles llamada Dalvik y un emulador donde poder probar la aplicación antes del lanzamiento sin tener que poseer un terminal Android. 

\begin{figure}
  \centering
    \includegraphics[scale=0.6]{./Android/imagenes/arquitecturaAndroid.jpg}
  \caption{Arquitectura de Android.}
  \label{fig:arquitecturaAndroid}
\end{figure} 

La arquitectura la podemos ver en la figura~\ref{fig:arquitecturaAndroid}. En la imagen podemos observar las cinco capas principales en las que se divide Android.
\begin{itemize}

\item \textbf{Applications:} en esta capa están todas las aplicaciones que nos proporciona el sistema operativo de base, como pueden ser la lista de contactos, un gestor de SMS, el navegador, el lanzador de aplicaciones, todas las aplicaciones de servicios de Google, como pueden ser Gmail, Google Maps, Google Calendar, Google Reader.   

\item \textbf{Application framework:} en esta capa está todo lo relacionado con los manejadores de activitys, llamadas de telefono, controladores de vistas. Es una capa que hay intermedia para manejo de hardware, con la que se pueden controlar tanto las notificaciones, como servicios que se ejucutan en segundo plano, etc. Los desarrolladores tienen el mismo acceso mediante esta capa de la API a todos los servicios que las aplicaciones nativas que proporciona el sistema el operativo. 

\item \textbf{Libraries:} como su nombre indica en esta capa están todas las librerías que el sistema operativo necesita, para manejo de archivos multimedia, manejo de gráficos 3D, renderizado web, etc.

\begin{itemize}
	\item \textbf{System C library:} una implementación del estandar C, optimizada para funcionar en sistemas móviles para las funciones del kernell de linux.
	\item \textbf{Media Libraries:} librerías basadas en PacketVideo's OpenCORE para grabación y reproducción de formatos de audio y video mas populares del momento como MP3, H.264, JPG o PNG.

	\item \textbf{Surface Manager:} librería para manejo de graficos 2D y 3D para varias aplicaciones.

	\item \textbf{LibWebCore:} motor de renderizado para navegadores embebidos de páginas web.

	\item \textbf{SGL:} motor para renderizado 2D.

	\item \textbf{3D libraries:} librería que implementa la API de OpenGL ES 1.0, que proporciona aceleración 3D por hardware si es posible o una alta optimización para renderizado por software en sistemas que no posean aceleración por hardware.

	\item \textbf{FreeType:} librería para manejo de fuentes, tanto bitmap como vectoriales.

	\item \textbf{SQLite:} librería para el manejo de la base de datos que proporciona Android.

\end{itemize}

\item \textbf{Android Runtime:} es una capa que está al mismo nivel que la capa de librerías. En esta capa se añaden muchas librerías para dotar de la mayoría de las funcionalidades que proporciona Java, también está en esta capa la máquina virtual (Dalvik) encargada de ejecutar el código Smali de los archivos DEX.

\item \textbf{Linux Kernel:} Android está basado en el kernel de Linux 2.6 y todo lo relatico a seguridad, manejo de memoria, control de procesos, pila de protocolos de red y modelo de drivers es el mismo. El kernel es el que proporciona una capa de abstracción entre el hardware y el software que usará Android.

\end{itemize}

Como ya hemos dicho antes Android corre cada aplicación en una máquina virtual, esta máquina virtual recibe el nombre de Dalvik. Dicho nombre viene de un pueblo de Islandía donde viven los familiares del creador de esta, Dan Bornstein. La máquina virtual ejecuta un byte code especial llamdo DEX (Dalvik Executable), que está especialmente diseñado y optimizado para funcionar en sistemas móviles, tablets, etc. En la figura~\ref{fig:maquinaVirtualDalvik} podemos ver todo el proceso desde la creación del archivo JAVA a la ejecución.

\begin{figure}[h]
  \centering
    \includegraphics[scale=0.8]{./Android/imagenes/maquinaVirtualDalvik.png}
  \caption{Proceso de ejecución en Android.}
  \label{fig:maquinaVirtualDalvik}
\end{figure}


Los ficheros DEX meten las cadenas duplicadas y las constantes en un mismo fichero para ahorrar espacio, normalmente los archivos DEX suelen ser más pequeños que los archivos JAR de la máquina virtual Java. Una vez instados los archivos DEX pueden ser modificados en el terminal para añadir optimizaciones, reordenado de byte en ciertos datos, quitado de clases vacias, etc. En la versión 2.2 de Android se añadió una nueva característica llamada JIT (Just-In-Time) que es compilación en tiempo real de los archivos DEX, por lo que se pueden añadir nuevas optimizaciones dependiendo de la plataforma.

Todas las aplicaciones de Android se distribuyen en unos archivos con extensión APK. Estos archivos no son más que archivos ZIP con la extensión cambiada. Todos deben tener una estructura idéntica, que se explica a continuación. Contiene diferentes carpetas en las que se incluyen ficheros de configuración, fiecheros necesarios para el funcionamiento de la aplicación y para comprobar la integridad de los mismos.

\begin{itemize}

\item \textbf{META-INF:} en un directorio que contiene tres archivos, \textit{MANIFEST.MF} que es el archivo de manifest, \textit{CERT.RSA} que es el certificado con el que está firmada la aplicación, \textit{CERT.SF} que contiene el hash en SHA-1 de todos los componentes de la aplicación, un ejemplo del archivo \textit{CERT.SF} es el siguiente:

\begin{verbatim}

Signature-Version: 1.0
Created-By: 1.0 (Android)
SHA1-Digest-Manifest: wxqnEAI0UA5nO5QJ8CGMwjkGGWE=
...
Name: res/layout/exchange_component_back_bottom.xml
SHA1-Digest: eACjMjESj7Zkf0cBFTZ0nqWrt7w=
...
Name: res/drawable-hdpi/icon.png
SHA1-Digest: DGEqylP8W0n0iV/ZzBx3MW0WGCA=
\end{verbatim}

\item \textbf{lib:} esta carpeta puede contener otras dependiendo de la plataforma para la que esté diseñada la aplicación. Si por ejemplo tiene código específicamente diseñado para x86 tendrá una carpeta llamada x86, si tiene código para MIPS una llamada mips donde se encontraría el código especialmente diseñado para esta plataforma. Puede que dicha carpeta no exista.

\item \textbf{res:} este directorio contiene todos los recursos que no tienen que ser compilados, como pueden ser imágenes, sonidos, etc.

\end{itemize}

Además de estas carpetas todos los ficheros APK incluyen estos tres archivos.

\begin{itemize}
\item \textbf{AndroidManifest.xml:} es un archivo que sirve para indicar la versión, los permisos que contiene la aplicación, las referencias a librerías, las activitys que contiene la aplicación. En el siguiente listado podemos observar un extracto de los permisos que usamos en la aplicación del proyecto.

\begin{lstlisting}[style=XML]
<uses-permission android:name="android.permission.INTERNET"/>
<uses-permission android:name="android.permission.WRITE_EXTERNAL_STORAGE"/>
<uses-permission android:name="android.permission.WRITE_SETTINGS"/>
<uses-permission android:name="android.permission.GET_ACCOUNTS"/>
<uses-permission android:name="android.permission.USE_CREDENTIALS" />
<uses-permission android:name="android.permission.MANAGE_ACCOUNTS" />
\end{lstlisting}

Podemos ver que con esto garantizamos que la aplicación pueda conectarse a internet, usar la tarjeta SD y manejar las cuentas almacenadas en el móvil, como puede ser la dirección de correo usada para dar de alta el móvil en Google Play.

\item \textbf{classes.dex:} el archivo DEX donde están todas las clases precompiladas en byte code para la máquina virtual Dalvik.

\item \textbf{resources.arsc:} es un fichero que contiene los recursos precompilados como pueden ser los XML de las interfaces de la aplicación, etc.

\end{itemize}

En la imagen~\ref{fig:estructuraAPK} se pueden observar los ficheros y las carpetas comentadas anteriormente del proyecto.

\begin{figure}[h]
  \centering
    \includegraphics[scale=0.8]{./Android/imagenes/estructuraAPK.png}
  \caption{Estructura del fichero APK del proyecto.}
  \label{fig:estructuraAPK}
\end{figure}

Toda aplicación en Android se construye con unos componentes básicos como pueden ser activities, intents, views, services, contents providers, widgets y un largo etcétera. A continuación vamos a explicar los más importantes.

\begin{itemize}

%poner sección
\item \textbf{Activity:} una activity es la entidad más básica de una interfaz de usuario donde se puede mostrar información en Android, podemos pensar que es como una ventana de cualquier aplicación de escritorio. Cada interfaz que vemos en una aplicación de Android es una activity. Android tiene un ciclo de manejo de activitys bastante complejo, se explicará en la sección~\ref{cap:desarrolandoAndroid} donde pondremos más incapié en el desarrollo de la aplicación móvil del proyecto.

\item \textbf{Intent:} un intent es un componente prácticamente imprescindible en cualquier aplicación de Android, es una forma de comunicación entre cualquier componente. Se pueden definir como mensajes o peticiones, ya que también puede comunicar aplicaciones entre sí. En el proyecto usamos intent para comunicar las activities y poder intercambiar datos entre ellas, también la usamos para abrir el lector de códigos QR que necesitamos, dentro de nuestra aplicación llamamos al intent que nos proporciona la aplicación lectora y cuando termine ella nos devuelve el valor que había en el código QR para que podamos tratarlo.

\item \textbf{View:} son los componentes básicos con los que podemos contruir las interfaces gráficas, como pueden ser botones, barras de texto, campos de texto, spinner, etc. Android pone a nuestra disposición una gran cantidad de estos elementos, y además brinda la posibilidad de crear nuevos, según los vayamos necesitando. Estos objetos normalmente se añaden a una vista y se pueden añadir, modificar o borrar en ella. Estos objetos tienen un ID único en la aplicación por el que podemos controlarlo y añadirle por ejemplo el texto si es un campo de texto o un listener si es un botón para que cuando se pulse realizar la acción que necesitemos.

\item \textbf{Services:} un servicio es un componente de Android que no tiene interfaz gráfica asociada y se ejecuta en segundo plano. Es similar a los servicios que ofrece cualquier sistema operativo. Pueden realizar cualquier acción, tanto recoger o actualizar datos, lanzar notificaciones cada cierto tiempo o mostrar activity para que el usuario introduzca algún valor que necesite.

\item \textbf{Content Provider:} es un mecanismo que posee Android para intercambiar información entre aplicaciones. Por ejemplo cuando usamos la opción de compartir en el teléfono, dentro de una aplicación, nos salen varias aplicaciones con las que podemos compartir directamente, estas son todas las aplicaciones que han implementado el content provider que necesita esta aplicación para compartir la información. En la figura~\ref{fig:contentProvider} podemos ver la opción de compartir de una aplicación y podemos observar como aparece por ejemplo GMail para mandar un email directamente desde aquí sin tener conocimiento de como se produce el intercambio de datos.
 
\begin{figure}[h]
  \centering
    \includegraphics[scale=0.2]{./Android/imagenes/contentProvider.png}
  \caption{Opción compartir de una aplicación.}
  \label{fig:contentProvider}
\end{figure}

\item \textbf{Broadcast Receiver:} es un componente de Android diseñado para actuar cuando ocurre un evento general del sistema, como puede ser la recepción de un SMS, la batería se está agotando, una llamada entrante, etc. También una aplicación puede generar eventos de este tipo para que cualquiera que implemente un Broadcast Receiver pueda recibirlo.

\item \textbf{Widget:} son elementos visuales y generalmente para que el usuario realice alguna acción, tales como poner en pausa la música, pasar de canción, revisar los feed RSS, mirar los correos pendientes, etc. Suele estar en alguna de las pantallas principales de Android. En la figura~\ref{fig:widget} podemos ver un widget de GMail.
 
\begin{figure}[h]
  \centering
    \includegraphics[scale=0.2]{./Android/imagenes/widget.png}
  \caption{Widget de GMail.}
  \label{fig:widget}
\end{figure}

\end{itemize}

\subsection{Desarrollando en Android.}\label{cap:desarrolandoAndroid}

Como ya hemos dicho anteriormente para el desarrollo de la aplicación móvil hemos usado el IDE de programación Eclipse. En el anexo~\ref{cap:apendiceA} podemos ver como realizar la configuración de Eclipse para la programación de aplicaciones Android. En la sección~\ref{cap:proyectoBasico} se explicará con más detenimiento la estructura de un proyecto básico en Android.

En este apartado vamos a explicar las entidades más importantes que existen en Android, como pueden ser las activity, fragment, y view usados en el proyecto.

En el proyecto la entidad más importante que hemos usado es la \lstinline{Activity}. Normalmente las activity están enfocadas a interactuar con el usuario de alguna forma, ya puede ser necesitando de alguna acción del usuario o mostrando información y normalmente ocupan toda la pantalla del terminal, pero puede flotar dentro de otra activity o agruparse con otras usando la clase \lstinline{ActivityGroup}. Todas las subclases que hereden de \lstinline{Activity} tienen que implementar dos métodos imprescindibles. Estos son \lstinline{onCreate(Bundle)} y \lstinline{onPause()}. El primero es donde se inicializa la activity y en el segundo es lo que ocurre cuando el usuario abandona la activity.

Habitualmente toda Activity tiene asociada una interfaz gráfica que se diseña en un fichero XML, como veremos posteriormente, este archivo se le indica a cada Activity con el método \lstinline{setContentView(int);} al cual hay que pasarle una constante que genera automáticamente la clase \lstinline{R} de Android. Una vez se le ha indicado el archivo XML se puede obtener cada uno de los componentes (botones, cuadros de texto, etc) que componen la interfaz gráfica con el método \lstinline{findViewById(int)}. Todo este proceso se tiene que realizar dentro del método \lstinline{onCreate();} y se puede observar en el siguiente trozo de código.

\begin{lstlisting}[style=Java]
@Override
protected void onCreate(Bundle savedInstanceState) {
	super.onCreate(savedInstanceState);
	setContentView(R.layout.initialconfigurescreen);

	TextView tAccountOk = (TextView) findViewById(R.id.tAccountOk);
	EditText tCertificate = (EditText) findViewById(R.id.
		tCertificate);
	ImageView image = (ImageView) findViewById(R.id.imageView1);

	Button bSelectAccount = (Button) findViewById(R.id.bConfigureScreenAccount);
}
\end{lstlisting}

A continuación vamos a explicar las características y su ciclo de vida de las Activity, ya que es un tema muy importante cuando desarrollamos una aplicación Android. En la figura~\ref{fig:cicloActivity} podemos ver todos los estados por los que pasa una Activity desde que se crea hasta que finaliza. 

\begin{figure}
  \centering
    \includegraphics[scale=0.8]{./Android/imagenes/cicloActivity.png}
  \caption{Ciclo de vida de una activity.}
  \label{fig:cicloActivity}
\end{figure}

Todos estos estados se pueden controlar mediante la implementación en la clase que hereda de \lstinline{Activity} de los siguientes métodos.

\begin{lstlisting}[language=Java]
public class Activity extends ApplicationContext {
	protected void onCreate(Bundle savedInstanceState);
	protected void onStart();
	protected void onRestart();
	protected void onResume();
	protected void onPause();
	protected void onStop();
	protected void onDestroy();
}
\end{lstlisting}

Como podemos ver en la figura \ref{fig:cicloActivity} podemos ver que el ciclo completo de una activity es desde el método \lstinline{onCreate(Bundle);} hasta que se realiza la llamada al método \lstinline{onDestroy();}. Como hemos visto antes en \lstinline{onCreate(Bundle);} se genera todo lo necesario para que la activity funcione, como puede ser la inicialización de la interfaz, la creación de un hilo para que realice una operación en background o cualquier otra acción que necesite ser inicializada. El método \lstinline{onDestroy();} se pararía el hilo y se libera la memoria usada por la activity. Entre los procesos \lstinline{onStart();} y \lstinline{onStop();} es donde se mantienen los recursos para que la activity pueda mostrar los datos al usuario. Por ejemplo si tenemos un \lstinline{BroadcastReceiver}, que nos puede cambiar la interfaz de usuario pues lo registramos en el método \lstinline{onStart();} y lo paramos en \lstinline{onStop();}. Estos dos métodos se llaman mucho a lo largo de la ejecución de la activity cada vez que el usuario oculta la activity y vuelve a ejecutarla. Los métodos \lstinline{onResume();} y \lstinline{onPause();} se usan para intercambio de activity, cuando apagamos la pantalla del móvil y volvemos a encenderla, cuando giramos la pantalla, etc. En estos métodos se suelen usar \lstinline{Bundle} para intercambiar información entre los estados y así conseguir por ejemplo restaurar el texto de un cuadro de texto cuando vuelve a generarse.

En esta tabla podemos observar en cada estado del ciclo de vida de una activity puede ser matada y cual sería el próximo estado.
\begin{center}
\begin{tabular}{|l | c | r|}

\hline
Method & ¿Terminable? & Proximo estado\\
\hline
onCreate() & No & onStart()\\
\hline
onRestart() & No & onStart()\\
\hline
onStart() & No & onResume() o onStop()\\
\hline
onResume() & No & onPause()\\
\hline
onPause() & No  & onResume() o onStop()\\
\hline
onStop() & Sí & onRestart() o onDestroy()\\
\hline
onDestroy() & Sí & Ninguna\\
\hline

\end{tabular}
\end{center}

Antes de la versión 3.0 de Android las activity tenían que ocupar toda la ventana y para cambiar o mostrar otra pantalla había que generar una nueva activity de la siguiente forma:

\begin{lstlisting}[style=Java]
Intent intent = new Intent(activity, SplashScreenActivity.class);
startActivity(intent);
\end{lstlisting}

Ese trozo de código se ejecuta en una activity y podemos ver que se crea un objeto \lstinline{Intent} al cual se le dice la activity en la que está y la activity que tiene que iniciar, en este caso la variable \lstinline{activity} es la actual, y \lstinline{SplashScreenActivity} es una activity que tiene la función de inciar todas las variables y realizar las conexiones básicas en el proyecto, acto seguido se usa el procedimiento \lstinline{startActivity(intent);} al que se le pasa el objeto \lstinline{Intent} creado anteriormente y con esto tendríamos la nueva activity ejecutándose.

Desde la versión 3.0 y posteriores las activity siguen ocupando toda la pantalla pero se dio la posibilidad al programador de que usara solo trozos de ella con una clase llamada \lstinline{Fragment} de esta forma no tendría que iniciar una nueva activity cada vez que quiera modificar la interfaz, de este modo se pudieron empezar a usar gestos de scroll laterar para mostrar varias interfaces o en pantallas grandes como una tablet poder modificarla sin tener que generar una nueva activity. 
\begin{figure}
  \centering
    \includegraphics[scale=0.3]{./Android/imagenes/gmailTablet.png}
  \caption{Aplicación de Gmail para tablet.}
  \label{fig:gmailTablet}
\end{figure}

\begin{figure}
  \centering
    \includegraphics[scale=0.4]{./Android/imagenes/gmailMovil.png}
  \caption{Aplicación de Gmail para móvil.}
  \label{fig:gmailMovil}
\end{figure}

\begin{figure}[h]
  \centering
    \includegraphics[scale=0.2]{./Android/imagenes/swype.png}
  \caption{Gesto Swype en la aplicación móvil.}
  \label{fig:swype}
\end{figure}

En la imagen~\ref{fig:gmailTablet} podemos ver la aplicación de Gmail diseñada mediante fragment y en ella si pulsamos algún correo en la parte izquierda de la aplicación nos mostraría el correo en la derecha sin tener que recargar la aplicación. En la imagen~\ref{fig:gmailMovil} podemos ver como en la versión movil no se usa esta forma por falta de espacio en la pantalla. Nosotros hemos realizado un diseño para intercambio de fragment mediante un gesto llamado swype o scroll lateral y como se puede ver en la imagen~\ref{fig:swype} podemos ver que no hay que volver a cargar otra activity ni nada, por lo que dotamos a la aplicación de una mayor fluidez.

Para que una aplicación pueda usar una determinada activity, el programador primeramente tiene que definir el uso y su función en el archivo \textit{AndroidManifest.xml}. Podemos ver un extracto de dicho archivo donde definimos un par de activity usadas en el proyecto.
\newpage
\begin{lstlisting}[style=XML]
<activity android:name=".FirmaDigitalUMA_ICSActivity" />
<activity android:name=".InitialConfiguration" android:noHistory=
	"true" />
\end{lstlisting}

Se puede observar que hemos declarado dos activities una sin ninguna opción y otra en la que no se guardará en la pila de llamadas de activity, por lo que si pulsamos el botón atrás no se abrirá de nuevo. Si no realizamos este proceso nos dará un error en tiempo de ejución la aplicación diciendo que hemos intentado ejecutar una activity que no está declarada.

\subsection{Un Proyecto básico de Android en Eclipse.}\label{cap:proyectoBasico}

A continuación vamos explicar con más detenimiento la estructura que tiene un proyecto básico de Android en Eclipse.

\begin{figure}
  \centering
    \includegraphics[scale=1]{./Android/imagenes/estructuraBasicaAndroid.png}
  \caption{Estructura básica de un proyecto Android.}
  \label{fig:estructuraBasicaAndroid}
\end{figure}

En la figura~\ref{fig:estructuraBasicaAndroid} podemos ver una captura de un proyecto recién creado. Podemos observar que se genera una carpeta principal en la que posteriormente colgarán el resto de carpetas necesarias. Estas carpetas son \textit{src}, \textit{gen}, \textit{assets}, \textit{bin}, \textit{res} y varios archivos sueltos como con \textit{AndroidManifest.xml}, \textit{proguard-project.txt}, \textit{project.properties}, vamos a explicar brevemente que contiene y cual es la función de dichas carpetas y documentos.

\begin{itemize}

\item \textbf{src:} en esta carpeta están todos los paquetes que contiene los archivos de código fuente que se necesitan en el proyecto.

\item \textbf{gen:} esta carpeta es donde se almacena todo lo que el proyecto de Android necesita para funcionar, casi todos los ficheros que se encuentran en el interior se generan cada vez que se construye el proyecto y si los modificamos nosotros, cuando volvamos a construir el proyecto borrarán los cambios. Dentro está la clase \lstinline{R} donde se declaran la mayoría de las constantes con direcciones de memoria que luego en tiempo de ejecución se usarán para realizar la conversión en bytecode del archivo java.

\item \textbf{bin:} es una carpeta donde se almacenan todos los archivos binarios, como puede ser el archivo APK, los archivos DEX, etc.

\item \textbf{res:} esta carpeta la encargada de contener todos los recursos necesarios para nuestra aplicación. Esta carpeta se divide en varias, como por ejemplo \textit{drawable-hdpi}, \textit{drawable-ldpi}, \textit{drawable-mdpi}, \textit{drawable-xhdpi} es donde se añaden todas imágenes usadas, sonidos, vídeos, etc. Pero no todos los recursos son contenido multimedia, hay otras carpetas como por ejemplo la carpeta \textit{layout} donde se almacenan las diferentes interfaces usadas en el formato XML o la carpeta \textit{values} donde se guardan todas las cadenas constantes en un archivo XML.

\item \textbf{AndroidManifest.xml:} como ya hemos explicado anteriormente es el archivo donde se declara todos los permisos e información de interés de la aplicación, como pueden ser las activity, los intent, la versión mínima que tiene que tener el móvil para ejecutar nuestra aplicación, etc.  

\item \textbf{project.properties:} es un archivo donde se pueden configurar diferentes parámetros del proyecto, como puede ser la API sobre la que se va a ejecutar el proyecto, o si queremos usar una herramienta que ofrece Google dentro del SDK para ofuscar el código llamada ProGuard.

\item \textbf{proguard-project.txt:} ProGuard como hemos dicho antes es una herramienta que ofrece Google dentro del SDK de Android para ofuscación de código, ya que hay muchas herramientas de ingeniería inversa que mediante la decompilación de los archivos DEX se puede llegar casi a conseguir el código realizado sin permiso. En este archivo se puede configurar los diferentes valores para el uso de esta herramienta, tales como son: qué tipo de ofuscación queremos utilizar si solo sintáctica o semántica, si queremos que se pueda tracear la salida del archivo, etc. Para ampliar conocimientos sobre dicha herramienta podemos visitar esta web, \url{http://developer.android.com/tools/help/proguard.html} donde está toda la información necesaria.

\end{itemize}

%-----------------------------------------
%Google App Egine
%-----------------------------------------


\section{Google App Engine.}\label{cap:GAE}
En este apartado de la memoria vamos a explicar lo que es, la configuración y como usar la plataforma Google App Engine.

\subsection{Introducción.}
Google App Engine es una conjunto de APIS que proporciona Google para construir tus propias aplicaciones web, que pueden ser alojadas y usadas en su servicio Google App y vendidas en Google Apps Marketplace. Además de alojamiento gratuito, Google ofrece un dominio, que es como el siguiente: \url{http://nombre\_de\_la\_aplicacion.appspot.com} y una base de datos propietaria de Google que se accede transparentemente a través de su API, gestión de usuarios mediante autentificación con cuentas Google del tipo: usuario@gmail.com, autentificación por federación o openID.

Además de todas estas características Google proporciona APIS para el desarrollo con Java, Python y Go, este último un lenguaje experimental propiedad de Google. Para usar dicha API, Google también proporciona un plugins para Eclipse, en caso de que el lenguaje elegido sea Java, que ayuda al despliegue de la aplicación web, auto completado y gestión de de las aplicaciones creadas. 

En el anexo~\ref{cap:configuracionGAEEclipse} se puede ver como instalar el plugins de Google App Engine para Eclipse.

En el proyecto se ha usado Java, por lo que las APIS de Python y Go no se han estudiado.

En general el uso de Google App Engine para desarrollar aplicaciones web es idéntico a crear una aplicación web con Java 2 Enterprise Edition (Java2EE), se pueden desarrollar servlet que recogen valores mediante métodos \lstinline{GET} o \lstinline{POST} y usar clases Java para hacer operaciones con ellos. A su vez para mostrar la información se pueden generar archivos *.jsp, que son archivos HTML con bloques o líneas de código Java incrustadas, que se introducen con estas etiquetas: \lstinline{<\%= linea de codigo Java \%>} o \lstinline{<\% bloque de codigo Java \%>}. A parte de archivos *.java y *.jsp, debemos tener una carpeta llamada war en la que tiene que ir toda la información de la aplicación web que queremos desplegar. En dicha carpeta hay varias subcarpetas como pueden ser css en la que tiene que ir el estilo de la web o WEB-INF en la que están todos los archivos de configuración, como pueden ser los permisos que tenemos que tener para poder acceder al uso de un servlet, si la web tiene conexión https, la configuración de la base de datos, etc. Más adelante se explicará con más detenimiento todas las carpetas de las que se compone un proyecto de Google App Engine.

Para este proyecto hemos tenido que desarrollar dos aplicaciones web, una que es un servidor de timestamp y otra que es una aplicación para gestión de las firmas digitales que realice cada usuario. A continuación vamos a explicar en profundidad la tecnología usada.

\subsection[Aplicación web genérica en GAE]{Explicación de una aplicación web genérica en Google App Engine.}
En esta parte voy a explicar en profundidad que es un servlet, los archivos de configuración, los archivos JSP y el resto de archivos necesarios para poder desplegar una aplicación en Google App Engine.
 
\subsubsection{¿Qué es un servlet?.}
Un servlet es la evolución de los antiguos applets de Java, su uso más común es generar páginas web dinámicamente con los parámetros que recibe mediante una petición realizada por el navegador web y datos que están almacenados en el servidor web.

Un servlet es un objeto Java que tiene que ser ejecutado en un servidor web o contenedor J2EE, que recibe unos parámetros, realiza una o varias acciones y devuelve un resultado que puede ser desde un código HTML, un JSP que genera dinámicamente un código HTML, un JSON o una simple cadena de texto.

Los servlets, junto con JSP, son la solución de Oracle a la generación de contenido dinámico equivalente al lenguaje PHP, ASP de Microsoft, Ruby, etc.

Los servlets forman parte de Java 2 Enterprise Edition (J2EE) que a su vez es una amplicación de Java 2 Standard Edition (J2SE), para su uso es necesario un servidor web que pueda interpretar código Java, el más famoso es Apache Tomcat que está desarrollado y mantenido por Apache Foundation, que son los encargados también de mantener y desarrollar el famoso servidor web Apache, aunque existen otro como JBoss, Jetty o GlassFish, pero como podremos ver no son los únicos, ya que el propio Google App Engine también funciona internamente a base de servlets y JSP.

Para crear un servlet hay que generar una clase Java que implemente la interfaz \lstinline{javax.servlet.Servlet} como puede ser \lstinline{javax.servlet.http.HttpServlet} que es un servlet específico para conexiones HTTP.
 
Una vez generada la clase hay que implementar el método \lstinline{doGet} para peticiones tipo \lstinline{GET} o el método \lstinline{doPost} para peticiones de tipo \lstinline{POST}. En el siguiente trozo de código se puede ver la implementación más básica de los métodos \lstinline{doGet} y \lstinline{doPost}.

\begin{lstlisting}[style=Java] 
@Override
protected void doGet(HttpServletRequest req, 
	HttpServletResponse resp) throws ServletException, IOException {
	// TODO Auto-generated method stub
	super.doGet(req, resp);
}

@Override
protected void doPost(HttpServletRequest req, 
HttpServletResponse resp) throws ServletException, IOException {
	// TODO Auto-generated method stub
	super.doPost(req, resp);
}
\end{lstlisting}

Una vez implementados los métodos que se necesiten se pueden usar el parámetro \lstinline{HttpServletRequest req} para recibir los valores que queramos enviar a la aplicación web y podemos usar \lstinline{HttpServletResponse resp} para enviar lo que queramos desde una redirección a un JSP, una página web, un JSON o una cadena de texto. 

Un ejemplo de como se reciben los parámetros sería: 

\begin{lstlisting}[style=Java]  
String num_sec = req.getParameter("sec");
\end{lstlisting}

Y si queremos devolver algo, por ejemplo un objeto \lstinline{JSONArray}:

\begin{lstlisting}[style=Java]   
PrintWriter out = resp.getWriter();
out.print(jsonArray);
out.flush();
\end{lstlisting}

Como podemos ver el objeto \lstinline{resp} nos da la posibilidad de conseguir un objeto \lstinline{java.io.PrintWriter} por el que podemos enviar lo que necesitemos.

La forma de acceder a un servlet mandándole peticiones \lstinline{GET} sería la siguiente: \url{https://servertimestamp.appspot.com/search?id=63&texto=Prueba}. Como podemos ver la dirección base es: \url{https://servertimestamp.appspot.com/}, el servlets estaría mapeado internamente en el servidor web, como ya veremos, en la dirección \url{/search} y el primer parámetro va precedido de \url{?id\_parametro} y el resto de \url{\&id\_parametro}. En nuestro ejemplo tendría dos parámetros que son \textit{id} y \textit{texto}, con sus valores después del =.

El método \lstinline{POST} es el utilizado para pasar parámetros por medio de formularios.

\subsection{¿Qué es JSP?.}
JSP es el acrónimo de JavaServer Pages y es una tecnología que ayuda a crear dinámicamente páginas web basadas en HTML o XML y es la solución equivalente a PHP de Oracle. En la figura~\ref{fig:modoJSP} podemos ver el proceso que se hace desde que se realiza la petición en el navegador hasta que se muestra un resultado.

\begin{figure}
  \centering
    \includegraphics[scale=0.5]{./GoogleAppEngine/imagenes/JSP_Model.png}
  \caption{Modo de interpretación de un archivo JSP}
  \label{fig:modoJSP}
\end{figure}

Un fichero JSP es la unión de código HTML con código Java, el cual es interpretado en el momento de la visualización de la página web. Un ejemplo es el siguiente:
 
\begin{lstlisting}[style=HTML]   
<!DOCTYPE html>
<html>
<body>
<table>
<tr>
	<th>ID</th>
	<th>Num sec</th>
	<th>Token de tiempo</th>
	<th>Mensaje</th>
	<th>URL para ver la firma</th>
	<th>Fecha</th>
	<th>Usuario</th>
	<th id="filadestino">Destino</th>
	<th>Verificado?</th>
</tr>
<% for (RowRepositorioGeneral row : rows) {%>
<tr>
	<td><%=row.getId()%></td>
	<td><%= row.getNum_sec()%></td>
	<td><%=row.getToken_tiempo()%></td>
	<td><%=row.getTexto_claro()%></td>
	<td><a href=<%=row.getUrl_firma()%>>URL para ver el token
			de tiempo</a></td>
	<td><%=row.getFecha()%></td>
	<td><%=row.getUsuario()%></td>
	<td id="filadestino"><%=row.getDestino()%></td>
	<td>
		<%Boolean confirmado = row.getConfirmado();
		if (!(confirmado == null) && confirmado)  else  %>
	</td>
</tr>
<%}%>
</table>
</body>
</html>
\end{lstlisting}

Como se puede ver en este trozo de código de este archivo JSP genera una tabla que se rellena dinámicamente con los valores que devuelve un objeto Java, se puede observar que se entrelazan trozos de código Java con etiquetas HTML. Si mostramos esta web y acto seguido introducimos otro objeto \lstinline{RowRepositorioGeneral} en la estructura, cuando recarguemos la tabla tendrá una fila nueva.

\subsubsection{La carpeta WAR.}

La carpeta WAR es la carpeta principal para el despliegue de una aplicación web, ya que en ella es donde se almacenan todos los archivos que se necesitan para el funcionamiento de la aplicación web, como pueden ser archivos HTML, CSS, JSP, imágenes, etc. En la figura~\ref{fig:carpetawar} se puede observar la carpeta WAR de una de las aplicaciones web realizadas.

\begin{figure}
  \centering
    \includegraphics{./GoogleAppEngine/imagenes/carpetawar.png}
  \caption{Carpeta WAR}
  \label{fig:carpetawar}
\end{figure}

Se puede observar las diferentes carpetas y ficheros que la forman. Podemos ver que la carpeta css contiene los archivos de estilo que la página web usará, también podemos ver los archivos web.xml y app.yalm, que son archivos de configuración del servidor que se verán en el próximo apartado~\ref{cap:refArchivosConfiguracionGoogleAppEngine} y además los archivos JSP que se usan en la aplicación junto con los archivos HTML y JavaScript que se necesiten.

\subsubsection{Archivos de configuración.\label{cap:refArchivosConfiguracionGoogleAppEngine}}
Los principales archivos de configuración son web.xml y app.yalm, este segundo es solo una forma de escribir de forma más legible XML, para que nos sea más sencillo entenderlo.

Un ejemplo de un archivo web.xml es el siguiente:

\begin{lstlisting}[language=XML]
<?xml version="1.0" encoding="utf-8"?>
<web-app xmlns:xsi="http://www.w3.org/2001/XMLSchema-
	instance"
xmlns="http://java.sun.com/xml/ns/javaee"
xmlns:web="http://java.sun.com/xml/ns/javaee/web-app_2_5.xsd"
xsi:schemaLocation="http://java.sun.com/xml/ns/javaee
http://java.sun.com/xml/ns/javaee/web-app_2_5.xsd" version=
	"2.5">

	<servlet>
		<servlet-name>AddRow</servlet-name>
		<servlet-class>pfc.ServletCreateRow</servlet-class>
	</servlet>
	<servlet-mapping>
		<servlet-name>AddRow</servlet-name>
		<url-pattern>/add</url-pattern>
	</servlet-mapping>

	<welcome-file-list>
		<welcome-file>ServerTimestampApplication.jsp</welcome-file>
	</welcome-file-list>
</web-app>
\end{lstlisting}

Como se puede observar en el código anterior se ha definido un servlet que se llamará \lstinline{AddRow} que usará la clase \lstinline{ServletCreateRow} y que estará mapeado en la dirección web \url{/add}, también podemos observar que el fichero que nos mostrará el servidor será \lstinline{ServerTimestampApplication.jsp} si entramos a la url principal.

A continuación podemos ver el aspecto de un archivo app.yalm:

\begin{lstlisting}[style=YAML]
application: repositoriorecibos
version: 1
runtime: java

handlers:
  - url: /add
    servlet: pfc.ServletCreateRow
    secure: always
welcome_files:
  - RepositorioGeneralApplication.jsp
\end{lstlisting}

Como podemos observar es mucho más fácil de entender y de escribir, el único problema que tienen los archivos YALM es que son sensibles a los espacios en blanco y tabulaciones, por lo que hay que tener cuidado a la hora de redactarlos. En este archivo se crea un servlet en la ruta \url{/add}, que es la clase Java \lstinline{ServletCreateRow} del paquete \lstinline{pfc} y que siempre hay que estar registrado en la aplicación para poder acceder a él. También podemos observar el fichero de bienvenida para cuando accedemos a la aplicación web. 

Al tener el archivo app.yalm en la carpeta WEB-INF el parseador de YALM  interpreta dicho archivo y genera automáticamente un archivo web.xml que usará el servidor web para su configuración.

Para ver todas las opciones de configuración que se pueden modificar en app.yalm\footnote{ Parámetros de configuración en el archivo app.yalm, \url{https://developers.google.com/appengine/docs/java/configyaml/}} o en web.xml\footnote{ Parámetros de configuración en el archivo web.xml, \url{https://developers.google.com/appengine/docs/java/config/}} se puede consultar los enlaces que hay en las notas al pie.












	
	%android
	%\input{./Android/android.tex}
	%google app engine
	%\input{./GoogleAppEngine/googleAppEngine.tex}
	
	%Diseño y arquitectura
	\chapter{Diseño e implementación del proyecto.}\label{cap:DisenhoEImplementacion}
\markboth{CAPÍTULO \ref{cap:DisenhoEImplementacion}. DISEÑO E IMPLEMENTACIÓN DEL PROYECTO.}{}
En este capítulo vamos a explicar el método elegido para el diseño de las aplicaciones del proyecto y la arquitectura que hemos tenido que diseñar para la implementación del proyecto.

\section{Metodología de diseño usada.}

La metodología usada a lo largo de todo el proyecto ha sido una metodología ágil de desarrollo. Esta metodología fue elegida ya que se basa en dar varias iteraciones y en cada iteración se va añadiendo nuevas funcionalidades al proyecto. Normalmente esta metodología se usa por equipo de programadores pero en este caso sólo estaba yo como programador y el director de proyecto fin de carrera como director del proyecto software. Al estar sólo como programador, no hemos podido aplicar ninguna de las metodologías ágiles que ya existen, por lo que hemos tomado las características generales y otras que más nos interesaban de cada una.

La característica general es que es un desarrollo iterativo e incremental. Cada uno de los periodos en los que se divide el proceso de diseño e implementación se llama etapas, que deben de tener una duración de entre una a cuatro semanas y al final de cada una debemos tener una demo funcional del proyecto, esta idea nos gustó ya que después de cada reunión añadíamos cosas nuevas al proyecto y a su vez podíamos ver como iba evolucionando.  

\section{Arquitectura del proyecto.}

La arquitectura básica del proyecto es la que podemos ver en la figura~\ref{fig:arquitecturaBasica}.

\begin{figure}
  \centering
    \includegraphics[scale=0.5]{./DisenhoYArquitectura/imagenes/arquitecturaBasica.png}
  \caption{Arquitectura básica del proyecto.}
  \label{fig:arquitecturaBasica}
\end{figure}

Podemos ver que la idea es tener dos aplicaciones web que son representadas por las nubes, ya que estaría en internet y una aplicación para un terminar Android. Como ya se ha explicado en capítulos anteriores en las aplicaciones web se ha usado Google App Engine para el desarrollo del proyecto y un terminar Android para la realización de las firmas. 

En la figura~\ref{fig:estructura} podemos ver la estructura final que tiene el proyecto, con las dos aplicaciones web diferenciadas y la aplicación del telefono.

\begin{figure}
  \centering
    \includegraphics[scale=0.3]{./DisenhoYArquitectura/imagenes/estructura.png}
  \caption{Estructura del proyecto.}
  \label{fig:estructura}
\end{figure}

A continuación vamos a explicar cada una de las tres partes en las que se divide principalmente el proyecto.

\begin{itemize}
\item \textbf{Servidor de timestamp:} como podemos ver en la figura~\ref{fig:servertimestamp} el servidor de timestamp que hemos diseñado tiene tres partes principales, que son añadir una nueva entrada, listar una entrada determinada y listar todas las entradas para mostrarlas en la aplicación web.  
\end{itemize}

\begin{figure}
  \centering
    \includegraphics[scale=0.3]{./DisenhoYArquitectura/imagenes/servertimestamp.png}
  \caption{Estructura del servidor de timestamp.}
  \label{fig:servertimestamp}
\end{figure}

\begin{figure}
  \centering
    \includegraphics[scale=0.3]{./DisenhoYArquitectura/imagenes/serverRepositorioGeneral.png}
  \caption{Estructura del servidor respositorio general.}
  \label{fig:serverRepositorioGeneral}
\end{figure}

\begin{itemize}
\item \textbf{Servidor Repositorio General:} en la figura~\ref{fig:serverRepositorioGeneral} observamos la estructura básica del servidor donde tendremos el repositorio para todas las firmas realizadas y la funciones principales que nos proporciona, como pueden ser la de añadir una entrada, listar los recibos que tu has firmado o los que están dirigidos para ti, otra función importante es la gestión de certificados, la generación de código QR y la verificación de cualquier firma, para poder ver si la firma es válida o no. De todas estas funciones sólo la de añadir y las de listar ambas firmas se podrán consultar con la aplicación móvil, el resto de funciones sólo serán accesibles desde la aplicación web.  
\end{itemize}


\begin{figure}
  \centering
    \includegraphics[scale=0.3]{./DisenhoYArquitectura/imagenes/aplicacionMovil.png}
  \caption{Estructura de la aplicación móvil.}
  \label{fig:aplicacionMovil}
\end{figure}

\begin{itemize}
\item \textbf{Aplicación Móvil:} podemos ver en la figura~\ref{fig:aplicacionMovil} la estructura general de la aplicación móvil. Como podemos observar tiene varias partes, podemos diferenciar el proceso de lectura del código QR, uso de certificados, realización de la firma digital y subirla a la aplicación web, las conexiones con las aplicaciones web, la parte de listado de las firmas y la conexión con la base de datos. Cada una de ellas tiene relación con las otras como podemos apreciar. El listado de las firmas podrá listarlas desde la base de datos o directamente de la aplicación web, dependiendo del caso en el que nos encontremos.
\end{itemize}

\section{Implementación del proyecto.}

En esta sección vamos a explicar la implementación realizada a lo largo del proyecto, tanto de la aplicación Android como de las aplicaciones web. A continuación vamos a explicar con profundidad el proyecto Android realizado.

\subsection{Proyecto Firma Digital UMA.}

La aplicación Firma Digital UMA es la aplicación móvil que hemos realizado para facilitar la firma y visualización de nuevos documentos. En ella  podemos firmar nuevos documentos o comprobar los anteriormente firmados de una forma fácil, el resto de gestiones se pueden realizar desde la aplicación web, donde se puede verificar, generar nuevos documentos para que sean firmados, etc. En la figura~\ref{fig:pantallaPrincipal} podemos ver el aspecto de la aplicación.

\begin{figure}[h]
  \centering
    \includegraphics[scale=0.2]{./Android/imagenes/pantallaPrincipal.png}
  \caption{Pantalla inicial de la aplicación.}
  \label{fig:pantallaPrincipal}
\end{figure}

\subsubsection{Uso de la aplicación.}

Lo primero que pensamos cuando nos pusimos a diseñar la aplicación era que el método de firma tenía que ser rápido y fácil de realizar, por lo que se decidió usar los códigos QR para agilizar la lectura de información. Sólo hay que pulsar el botón de añadir nuevo recibo (figura~\ref{fig:botonAnhadir}) y se abrirá el lector de códigos QR, una vez leídos el código este se firmará y se subirá automáticamente al servidor sin necesidad de que el usuario realice otra acción.

\begin{figure}[h]
  \centering
    \includegraphics[scale=0.2]{./Android/imagenes/botonAnhadir.png}
  \caption{Detalle del botón añadir.}
  \label{fig:botonAnhadir}
\end{figure}

La aplicación se puede dividir principalmente en dos partes claramente diferenciadas, una en la que se muestra las firmas realizadas y otra en la que se muestran que tienen como destino el usuario que está ejecutando la aplicación. Las dos partes son prácticamente idénticas y tienen el mismo uso. Si pulsamos encima de cualquier recibo de la lista nos mostrará toda la información de dicha firma, podemos ver un ejemplo en la figura~\ref{fig:informacionFirma}. Podemos observar todos los datos necesarios, como pueden ser destino, quien ha realizado la firma, la fecha, el texto, etc y también podemos ver si está verificada con el tick de color verde o una señal de error roja en caso contrario, que podemos ver en la esquina superior derecha. En el caso de los recibos de los que se es destino de la firma, se indica de una forma más clara quien es el que usuario que envía el mensaje, pero el resto de la información es la misma.

En la información que nos proporcionan podemos ver la dirección del servidor de tiempo usado y si pulsamos en ella se abrirá el navegador con una dirección donde podremos confirmar el hash y la fecha en la que se realizó la firma.

\begin{figure}
  \centering
    \includegraphics[scale=0.2]{./Android/imagenes/informacionFirma.png}
  \caption{Información de una firma realizada.}
  \label{fig:informacionFirma}
\end{figure}

\subsubsection{Detalles de la implementación.}

En la figura~\ref{fig:proyectoAndroid} podemos ver el proyecto de eclipse con todas las clases que hemos tenido que desarrollar, desde la conexión a la base de datos, el inicio de la aplicación, el cifrado, etc. La estructura del proyecto es la misma que ya hemos explicado anteriormente.

\begin{figure}
  \centering
    \includegraphics[scale=0.7]{./Android/imagenes/proyectoAndroid.png}
  \caption{Proyecto de Android en Eclipse.}
  \label{fig:proyectoAndroid}
\end{figure}

A continuación vamos a explicar cada una de las clases y la función que tiene dentro del proyecto.

\begin{itemize}

\item \textbf{AuxiliaryFunction.java:} en esta clase se han implementado todas las funciones que son usadas en varias clases en el proyecto, como puede ser la función que verifica si un certificado existe en un ruta determinada y si se puede abrir, también la función que pasa de un array de bytes a una cadena de texto. 

\item \textbf{AuxiliaryVariables.java:} aquí se almacenan todas las variables comunes que vamos a necesitar en la aplicación para que funcione correctamente. En el siguiente trozo de código podemos ver que son variables como el login, con el que se está autentificado, la cookie para autentificación en el servicio, el key store con las claves para realizar las firmas, si la aplicación no tiene internet, para mostrar la base de datos que tengamos almacenados y no intentar bajar recibos nuevos y otras.

\begin{lstlisting}[style=Java]
private static Login LOGIN_GLOBAL;
private static Cookie COOKIE_GLOBAL;
private static KeyStore KEYSTORE_GLOBAL;
private static boolean WITHOUT_INTERNET;
private static List<ReceiptRows> LIST_RECEIPT;
private static String THEME;
\end{lstlisting}

Además de las variables tenemos sus getter and setter correspondientes, que nos sirven para consultar o cambiar los valores.

\item \textbf{ChangePreferences.java:} esta clase es la encargada de realizar todo el manejo del cambio de preferencias, si se producen. En la aplicación hemos usado una clase que proporciona Android para el almacenado de configuración llamada \lstinline{SharedPreferences}. Al usar esta clase Android proporciona listener para que en caso de que las preferencias cambien se activen automáticamente y de esta forma podamos controlar todos los cambios que se realicen y actualizar la interfaz y guardar los cambios.

Un trozo de esta clase, en el que se puede observar como se crea y se implementa un listener que se activará cuando se produzca un cambio en la configuración lo podemos ver a continuación.

\begin{lstlisting}[style=Java]
EditTextPreference textPath = (EditTextPreference) findPreference(getActivity().getResources().getString(R.string.key_cert_path));

// Para que cuando cambie el texto lo cambie tambien en el titulo...
textPath.setOnPreferenceChangeListener(new OnPreferenceChangeListener() {
	public boolean onPreferenceChange(Preference preference, Object newValue) {
		EditTextPreference textPath = (EditTextPreference) preference;
		String s = (String) newValue;

		String pass = settings.getString("certificate_password", "NotValue");
		if (!pass.equals("NotValue")) {
			boolean check = AuxiliaryFunction.checkCert(s, pass);

			if (check) {
				textPath.setSummary(s);
				textPath.setText(s);
				Editor editor = settings.edit();
				editor.putString("path_certificate", s);
				editor.commit();
				return true;
			} else {
				textPath.setSummary("No se ha podido cargar el certificado,\nla ruta no es correcta");
				return true;
			}
		} else {
			textPath.setSummary("Password no guardado");
			return true;
		}

	}

});
\end{lstlisting}

Creamos la variable EditTextPreference donde mostraremos el password del certificado que estamos usando para firmar el texto. A continuación implementamos el listener \lstinline{OnPreferenceChangeListener}, el cual se activará cuando cambiemos el password del certificado. Podemos observar que cuando introducimos un nuevo password hacemos una comprobación para ver si después del cambio podemos seguir usando dicho certificado, si no es posible informamos al usuario mostrando un mensaje en el que se le indicará que no se ha guardado el password.

\item \textbf{ConexionRepositorio.java:} en esta clase se ha implementado todo lo referente a interactuar con la aplicación web Repositorio General. En dicha clase se ha implementado el método para añadir un recibo nuevo, listar todos los recibos que hay guardados, etc. Todos los métodos son estáticos para no tener que crear instancias de esta clase, además se vio que teníamos que crear una nueva conexión cada vez que queríamos añadir un nuevo recibo, por lo que no tendría sentido crea instancias de esta clase. A continuación podemos ver el método que añade un nuevo recibo.

\begin{lstlisting}[style=Java]
public static String addRow(String plainText, String tokenTime, Account account, String to) {
	/*
	 * Primero se le codifican los espacios porque si no la funcion
	 * URLEncoder.encode los cambia por '+' y si el documento tiene '+'
	 * luego los pone como si fueran espacios
	 */
	String textoSinEspacios = plainText.replace(" ", "%20");
	textoSinEspacios = URLEncoder.encode(textoSinEspacios);
	String token = tokenTime.split(";;")[1];
	String fecha = tokenTime.split(";;")[2];
	String url = cadServer + cadAdd + cadServerTimestampSearch + token + "&texto=" + textoSinEspacios + "&token=" + token + "&usuario=" + account.name
			+ "&destino=" + to + "&fecha="+ fecha;
	String response = "";
	try {
		URL u1 = new URL(url);

		// Si queremos usar el proxy inicializarlo de esta forma, donde
		// sa es: SocketAddress sa = new
		// InetSocketAddress("proxy.alu.uma.es", 3128);
		// HttpURLConnection c = (HttpURLConnection)
		// u.openConnection(new Proxy(Proxy.Type.HTTP, sa));

		Cookie cookie = AuxiliaryVariables.getCOOKIE_GLOBAL();
	
		HttpURLConnection con1 = (HttpURLConnection) u1.openConnection();

		con1.addRequestProperty("Cookie", cookie.getName() + "=" + cookie.getValue());
		con1.setRequestMethod("GET");
		con1.connect();

		DataInputStream is1 = new DataInputStream(con1.getInputStream());
		response = is1.readLine();

		con1.disconnect();

	} catch (MalformedURLException e) {
		response = "MalformedURL";
		e.printStackTrace();
	} catch (ProtocolException e) {
		response = "ProtocolException";
		e.printStackTrace();
	} catch (IOException e) {
		response = "IOException";
		e.printStackTrace();
	}
	return response;
}
\end{lstlisting}

La función necesita una serie de parámetros para poder ser invocada, estos parámetros son los datos que queremos almacenar, como pueden ser el texto en claro, el token de tiempo que ha devuelto la aplicación web de timestamp, la cuenta con la que se está usando la aplicación y el destino, el resto de valores o los añade la aplicación web o la aplicación del móvil. A continuación se realiza la conexión con el servidor y se recibe una cadena donde se indica si se ha realizado bien la operación.

\item \textbf{ConexionServerTimestamp.java:} al igual que la clase anterior en esta se ha realizado todo lo relacionado con la conexión con la aplicación web del Servidor Timestamp. Sólo tiene una función que es la de añadir al servidor, que tiene un aspecto similar a la mostrada en el apartado anterior.

\item \textbf{DigitalSignatureOpenHelper.java:} esta clase es la encargada de la creación de la base de datos SQLite, cuando vamos a necesitar tener acceso a ella hay que generar un objeto de dicha clase y llamar al constructor, que devuelve un objeto con el que podremos buscar, insertar o borrar elementos de la base de datos. Al pensar en el diseño de la aplicación se creyó necesario el uso de la base de datos para almacenar todos los recibos que se han realizado hasta la fecha, la aplicación sabe cual es el último recibo que tiene almacenado y sólo pide a la aplicación web que le de los recibos nuevos, de esta forma de ahorra en tiempo de inicio de la aplicación y además en la cantidad de datos móviles que usamos.

Esta clase tiene que heredar de \lstinline{SQLiteOpenHelper} que es la clase genérica que proporciona Android para manejo de la base de datos SQLite que tiene. Hay que implementar el constructor y dos métodos más obligatoriamente \lstinline{public void onCreate(SQLiteDatabase db);} y \lstinline{public void onUpgrade(SQLiteDatabase db, int oldVersion, int newVersion);}, el primero es usado para crear la base de datos en la primera ejecución y el segundo para actualizar la base de datos cuando sea necesario.

La base de datos se crea con la siguiente sentencia SQL, podemos verla en el siguiente trozo de código.

\begin{lstlisting}[style=Java]
static final String KEY_SEQ_NUM = "NUM_SEC";
static final String KEY_SIGN_URL = "URL_FIRMA";
static final String KEY_PLAIN_TEXT = "TEXTO_CLARO";
static final String KEY_TIME_TOKEN = "TOKEN_TIEMPO";
static final String KEY_USER = "USUARIO";
static final String KEY_VERIFY = "VERIFICADO";
static final String KEY_DESTINY = "DESTINO";
static final String KEY_DATE = "FECHA";

private static final String DIGITALSIGNATURE_TABLE_CREATE = "CREATE TABLE " + DIGITALSIGNATURE_TABLE_NAME + " (" +
	BaseColumns._ID + " INTEGER PRIMARY KEY AUTOINCREMENT," +
	KEY_SEQ_NUM + " TEXT, " +
	KEY_SIGN_URL + " TEXT, " +
	KEY_PLAIN_TEXT + " TEXT, " +
	KEY_TIME_TOKEN + " TEXT, " +
	KEY_USER + " TEXT, " +
	KEY_VERIFY + " TEXT, " +
	KEY_DESTINY + " TEXT, " +
	KEY_DATE + " TEXT );";
\end{lstlisting}

\item \textbf{FirmaDigitalUMA\_ICSActivity.java:} esta es la clase principal del proyecto, en ella se crea la activity principal de la aplicación. Se puede ver en el código que no extiende a la clase \lstinline{Activity}, si no que lo hace de \lstinline{FragmentActivity}, esto es así porque como ya hemos explicado anteriormente se han usado \lstinline{Fragment} para el diseño de la aplicación y no sólo \lstinline{Activity}. Además de esto se ha añadido funcionalidad mediante el uso de \lstinline{ViewPager} y \lstinline{TabsAdapter}, el primero para el uso del scroll lateral y el segundo para tener las dos pestañas principales de la aplicación. En el siguiente trozo de código podemos ver la creación de los dos elementos que hemos dicho antes, junto con la action bar.

\begin{lstlisting}[style=Java]
mViewPager = new ViewPager(this);
mViewPager.setId(R.id.pager);
setContentView(mViewPager);

final ActionBar bar = getActionBar();
bar.setTitle("Firma digital UMA");
bar.setNavigationMode(ActionBar.NAVIGATION_MODE_TABS);

mTabsAdapter = new TabsAdapter(this, mViewPager);
mTabsAdapter.addTab(bar.newTab().setText("Repositorio general"), SignaturesGeneral.class, null);
mTabsAdapter.addTab(bar.newTab().setText("Firmas para ti"), SignaturesForYou.class, null);
\end{lstlisting}

Además de la creación de los elementos anteriores esta clase también es la encargada de crear los menús, que en la versión 4.0 de Android tienden a desaparecer debido a la desaparición del botón físico de menú en los nuevos terminales. En las nueva versión el botón de menú se añade a la action bar, junto con las acciones más importantes. A continuación podemos ver como se crean y añaden los botones de añadir un nuevo recibo, el de configuración o el de salir de la aplicación.

\begin{lstlisting}[style=Java]
@Override
public boolean onCreateOptionsMenu(Menu menu) {
	MenuInflater inflater = getMenuInflater();
	inflater.inflate(R.menu.main, menu);
	return true;
}
@Override
public boolean onOptionsItemSelected(MenuItem item) {
	switch (item.getItemId()) {

	case R.id.menuitem_add:
		Log.d(AuxiliaryVariables.TAG_DEBUG, "Add");
		IntentIntegrator intentQR = new IntentIntegrator(this);
		intentQR.initiateScan();
		return true;

	case R.id.menuitem_quit:
		Log.d(AuxiliaryVariables.TAG_DEBUG, "Quit");
		finish();
		return true;
	case R.id.menuitem_about:
		Log.d(AuxiliaryVariables.TAG_DEBUG, "About");
		Toast.makeText(context, "about", Toast.LENGTH_SHORT).show();
		return true;
	case R.id.menuitem_settings:
		Log.d(AuxiliaryVariables.TAG_DEBUG, "Settings");
		Toast.makeText(context, "ajustes", Toast.LENGTH_SHORT).show();
		Intent prefsIntent = new Intent(getApplicationContext(),
		        ChangePreferences.class);
		startActivity(prefsIntent);
		return true;
	}
	return false;
}
\end{lstlisting}

Además de la creación de la activity y de los menús es la encargada de llamar a la función de subir un recibo a la aplicación web después de recibir por medio de un intent el resultado de la lectura del código QR, se puede ver en este trozo de código.

\begin{lstlisting}[style=Java]
public void onActivityResult(int requestCode, int resultCode, Intent intent) {
	IntentResult scanResult = IntentIntegrator.parseActivityResult(requestCode, resultCode, intent);
	String plaintext = "";
	if ((plaintext = scanResult.getContents()) != null) {
		progressDialog = new ProgressDialog(activity);
		progressDialog.setMessage("Subiendo el recibo...");
		progressDialog.show();

		ConexionRepositorio.addRow(handler, plaintext);
		
	}
}
\end{lstlisting} 

\item \textbf{InitialConfiguration.java:} esta clase es la encargada de realizar la configuración de la aplicación cuando es la primera ejecución en un teléfono, en ella se tiene que configurar el certificado y su password y la cuenta. Es la encargada de crear las \lstinline{SharedPreferences} para que la aplicación funcione correctamente.

\item \textbf{InitialConfiguration.java, IntentIntegratorSupportV4.java, IntentIntegratorV30.java, IntentResult.java:} estas clases son las encargadas de la lectura de los códigos QR. Están hechas por ZXing y se distribuyen bajo licencia Apache License, Version 2.0. Proporcionan mediante un intent la posibilidad de lectura de códigos QR con su aplicación Barcode Scanner (figura~\ref{fig:barcodeScanner}) gratuita en Google Play, si no se tiene la aplicación instalada proporciona los métodos para hacerlo.

\end{itemize}

\begin{figure}[h]
  \centering
    \includegraphics[scale=0.3]{./Android/imagenes/barcodeScanner.png}
  \caption{Aplicación Barcode Scanner.}
  \label{fig:barcodeScanner}
\end{figure}

\begin{itemize}

\item \textbf{Login.java:} esta clase es la encargada de realizar el login en la aplicación web Repositorio General. Google proporciona un método de login si se tienen las credenciales, cosa que tenemos gracias a que en Android se necesita tener una cuenta de Google para poder usar Google Play y los diferentes servicios que ofrece. La dirección a la que tenemos que dirigirnos es a: \url{https://repositoriorecibos.appspot.com/\_ah/login}. Esta clase tiene una variable privada que es el token de autentificación que conseguimos mediante la realización del login, para conseguirlo mostramos todas las cuentas de Google que hay configuradas en el terminal Android, cuando el usuario selecciona una de ellas iniciamos la conexión con el servidor y recibimos el token. Esto se hace todo en segundo plano para que la aplicación no se quede parada mientras se realiza el proceso de obtención del token. En el código que sigue podemos ver el proceso de obtención del token.

\begin{lstlisting}[style=Java]
Thread t = new Thread() {
	public void run() {
		try {
			AccountManagerFuture<Bundle> future = manager.getAuthToken(account, "ah", null, activity, null, null);
			Bundle bundle;
			bundle = future.getResult();
			token = bundle.getString(AccountManager.KEY_AUTHTOKEN);
		} catch (OperationCanceledException e) {
			e1.printStackTrace();
		} catch (AuthenticatorException e) {
			e1.printStackTrace();
		} catch (IOException e) {
			e1.printStackTrace();
		}
	}
}.start();
\end{lstlisting} 

La parte más interesante del código es en la llamada \lstinline{manager.getAuthToken(account, "ah", null, activity, null, null);} con esta función conseguimos la cookie donde está el token de acceso para posteriormente con la función \lstinline{bundle.getString(AccountManager.KEY_AUTHTOKEN);} conseguirlo.

La función \lstinline{public Cookie getAuthCookie(String authToken);} se puede ver en el siguiente trozo de código.

\begin{lstlisting}[style=Java]
public Cookie getAuthCookie(String authToken) throws ClientProtocolException, IOException {
	DefaultHttpClient httpClient = new DefaultHttpClient();
	Cookie retObj = null;
	String cookieUrl = gaeAppLoginUrl + "?continue=" + URLEncoder.encode(gaeAppBaseUrl, "UTF-8") + "&auth=" + URLEncoder.encode(authToken, "UTF-8");
	
	HttpGet httpget = new HttpGet(cookieUrl);
	HttpResponse response = httpClient.execute(httpget);
	if (response.getStatusLine().getStatusCode() == HttpURLConnection.HTTP_OK
			|| response.getStatusLine().getStatusCode() == HttpURLConnection.HTTP_NO_CONTENT) {

		if (httpClient.getCookieStore().getCookies().size() > 0) {
			retObj = httpClient.getCookieStore().getCookies().get(0);
		}

	}

	return retObj;
}
\end{lstlisting} 

En ella podemos ver que hacemos una conexión a la dirección \url{http://repositoriorecibos.appspot.com/\_ah/login?continue=http://repositoriorecibos.appspot.com&auth=token}, en la que se le indica el token de acceso y la url a la que tenemos que seguir cuando se realice el login. Una vez realizada la identificación devolvemos la cookie donde irá el token de acceso.

\item \textbf{ReceiptRows.java:} esta clase es la que representa un recibo, con ella se pueden crear objetos para almacenar todos los valores que necesitamos para identificar un recibo, como pueden ser el texto, la url de la firma, la fecha, el destino, etc. Este es el tipo de objeto que se usa para recoger la información de la base de datos y para almacenarla. Durante toda la ejecución de la aplicación tendremos una lista de objetos de esta clase para tener acceso a los recibos. Es una clase simple con un constructor con todos los valores que se guardan como parámetros y los getter y setter de las diferentes variables. También tiene definidas todas las constantes para el nombre de la tabla de la base de datos.

\item \textbf{SignaturesGeneral.java:} esta clase extiende a \lstinline{ListFragment}, es la lista de los recibos que el usuario a firmado. Es una de las dos pantallas principales de la aplicación, se puede ver en la figura~\ref{fig:signaturesGeneral}. Cuando se genera la interfaz se hace una consulta a la base de datos y se recibe un objeto de la clase \lstinline{Cursor}, con el que se puede iterar para obtener todos los recibos que ha firmado el usuario. En el siguiente trozo de código podemos ver como se hace.

\begin{lstlisting}[style=Java]
DigitalSignatureOpenHelper digitalSignatureOpenHelper = new DigitalSignatureOpenHelper(activity);
SQLiteDatabase sqLiteDatabase = digitalSignatureOpenHelper.getReadableDatabase();
Cursor cursor = sqLiteDatabase.query(DigitalSignatureOpenHelper.DIGITALSIGNATURE_TABLE_NAME,new String[] { DigitalSignatureOpenHelper.KEY_PLAIN_TEXT }, null, null, null, null, null);
\end{lstlisting} 

%\end{itemize}

\begin{figure}[h]
  \centering
    \includegraphics[scale=0.2]{./Android/imagenes/signaturesGeneral.png}
  \caption{Detalle fragment generado por la clase SignaturesGeneral.java.}
  \label{fig:signaturesGeneral}
\end{figure}

%\begin{itemize}

\item \textbf{SignaturesGeneralInformation.java:} esta clase es la encargada de mostrar la información cuando pulsamos en algún recibo que hemos firmado. Es una clase que extiende de la clase \lstinline{Activity} y es la encargada de cargar la interfaz modelada en el XML \lstinline{signaturesgeneralinfo.xml} y mostrar toda la información del recibo que ha sido pulsado de la lista. Cuando se pulsa un recibo en la pantalla anterior que es la generada por \lstinline{SignaturesGeneral.java}, antes de cargar esta se le manda la posición que ha sido pulsada para después poder localizarla y acto seguido se muestra la información, esto se hace por medio de un \lstinline{Intent} al cual se le añade un valor entero con la posición. En el siguiente trozo de código podemos ver como se recupera el valor, se realiza la consulta en la base de datos y se rellenan todos los campos donde mostraremos la información del recibo seleccionado.

\begin{lstlisting}[style=Java]
super.onCreate(savedInstanceState);
setContentView(R.layout.signaturesgeneralinfo);

Intent intent = this.getIntent();
int pos = intent.getExtras().getInt("position");

DigitalSignatureOpenHelper digitalSignatureOpenHelper = new DigitalSignatureOpenHelper(this);
SQLiteDatabase sqLiteDatabase = digitalSignatureOpenHelper.getReadableDatabase();
Cursor cursor = sqLiteDatabase.query(DigitalSignatureOpenHelper.DIGITALSIGNATURE_TABLE_NAME, new String[] { DigitalSignatureOpenHelper.KEY_USER, DigitalSignatureOpenHelper.KEY_DESTINY, DigitalSignatureOpenHelper.KEY_SIGN_URL, DigitalSignatureOpenHelper.KEY_PLAIN_TEXT, DigitalSignatureOpenHelper.KEY_VERIFY, DigitalSignatureOpenHelper.KEY_DATE }, null, null, null, null, null);

if (cursor.moveToPosition(pos)) {
	
	final ActionBar bar = getActionBar();
	bar.setTitle("Informacion de la firma.");
	bar.setDisplayHomeAsUpEnabled(true);

	TextView textUser = (TextView) findViewById(R.id.text_sign_general_user);
	TextView textDestiny = (TextView) findViewById(R.id.text_sign_general_to);
	TextView textUrl = (TextView) findViewById(R.id.text_sign_general_url);
	TextView textPlainText = (TextView) findViewById(R.id.text_sign_general_plain_text);
	ImageView imageVerify = (ImageView) findViewById(R.id.image_sign_general_verify);
	TextView textDate = (TextView) findViewById(R.id.text_sign_general_date);
	
	textUser.setText(textUser.getText() + " " + cursor.getString(0));
	textDestiny.setText(textDestiny.getText() + " " + cursor.getString(1));
	textUrl.setText(cursor.getString(2));
	textPlainText.setText(cursor.getString(3));

	String verify = cursor.getString(4);
	if (verify.equals("true")) {
		imageVerify.setImageResource(R.drawable.ok);
	} else {
		imageVerify.setImageResource(R.drawable.cancel);
	}
	textDate.setText(cursor.getString(5));

	sqLiteDatabase.close();
}
\end{lstlisting} 

Podemos ver como se genera el objeto \lstinline{DigitalSignatureOpenHelper} y se realiza la consulta con el método \lstinline{public Cursor query(String table, String[] columns, String selection, String[] selectionArgs, String groupBy, String having, String orderBy);}, a continuación generamos todos los \lstinline{TextView} necesarios para mostrar la información y los rellenamos con los datos.

\item \textbf{SignaturesForYou.java:} esta clase desarrollada es equivalente a la clase \lstinline{SignaturesGeneral.java} y es la encargada de mostrar una lista con todos los recibos de los que el usuario es destinatario, la estructura es prácticamente idéntica, con una única diferencia y es que se realiza un filtrado para que el campo destino sea el mismo que la cuenta de usuario usada en la aplicación. Podemos ver el resultado en la figura~\ref{fig:signaturesForYou}.

%\end{itemize}

\begin{figure}[h]
  \centering
    \includegraphics[scale=0.2]{./Android/imagenes/signaturesForYou.png}
  \caption{Detalle fragment generado por la clase SignaturesForYou.java.}
  \label{fig:signaturesForYou}
\end{figure}

%\begin{itemize}

\item \textbf{SignaturesForYouInformation.java:} es equivalente a la clase \lstinline{SignaturesGeneralInformation.java} y es la encargada de mostrar la información de los recibos que van dirigidos al usuario. El proceso es igual que en \lstinline{SignaturesGeneralInformation.java}, se envía la posición que ha sido marcada y se busca dicho recibo y se muestra toda la información.

\item \textbf{SplashScreenActivity.java:} esta clase es la encargada de generar la primera activity que se muestra en la aplicación cuando se inicia, además de cargar todas las variables que serán usadas durante la ejecución. Esta clase crea una interfaz en la que sólo muestra el logo de la aplicación y una barra progreso que muestra el proceso de carga, pero en background comprueba que no sea la primera ejecución, si lo es llama a la clase \lstinline{InitialConfiguration.java} para generar la configuración. Si no es la primera ejecución abre el archivo de preferencias compartidas y genera todos los objetos necesarios para la ejecución como pueden ser, la cuenta que utilizaremos posteriormente, el tema que estamos usando, hace el login en la aplicación web, el keystore que usaremos para firmar en la aplicación y a parte realizamos la conexión con la aplicación web Repositorio General y conseguimos las filas que se hayan insertado nuevas añadiéndolas a la base de datos de la aplicación para que posteriormente podamos usarlas sin tener que pedirlas al servidor, para ello se utiliza un objeto de tipo \lstinline{JSONArray}, una vez realizadas todas esas acciones se procede a la inicialización de una nueva activity que genera la clase \lstinline{FirmaDigitalUMA_ICSActivity.java} y que dará lugar a la creación de la interfaz principal de la aplicación.

\end{itemize}

A continuación vamos a explicar las dos aplicaciones web realizadas.

\subsection{Servidor de timestamp.}

En este apartado vamos a explicar en profundidad todo lo relacionado con la aplicación de timestamp que he tenido que desarrollar, desde el diseño que se ha seguido hasta los problemas que me han surgido.

En principio me gustaría explicar para que se usa un servidor de timestamp en general. Un servidor de timestamp es un registro donde cualquier persona puede subir un documento y el servidor guarda ese documento añadiéndole la fecha en la que se realizó la subida, dicha aplicación luego ofrece el servicio de consultar a que hora fue subido dicho documento. Un ejemplo podría ser \url{https://seguro.ips.es/servidortimestamp/index.asp} que se puede ver una captura de pantalla en la figura~\ref{fig:server_ips_timestamp}. En dicha captura podemos ver que tiene las opciones básicas de un servidor de timestamping como puede ser generar un sello, consultar su validez, etc. 

\begin{figure}[h]
  \centering
    \includegraphics[scale=0.5]{./GoogleAppEngine/imagenes/server_ips_timestamp.png}
  \caption{Servidor Timestamp https://seguro.ips.es/servidortimestamp}
  \label{fig:server_ips_timestamp}
\end{figure}

La veracidad de que el sellado de dicho documento fue en el instante que dice ser, depende de la confianza que se tenga en ese servicio. Es similar a cuando se necesita que te sellen un documento físico, que dependiendo de quien lo necesite, necesitamos que lo firme un notario, un empleado público, etc. Normalmente suelen existir servidores de timestamping en los que se tiene confianza y los documentos sellados se consideran verdaderos.

Existen tres modelos principales de servidor de timestamping que son los siguientes:

\begin{itemize}

\item \textbf{Solución Arbitrada básica:} En esta solución el usuario que quiere sellar algo mandaría una copia del documento que quiere sellar a la entidad de sellado, que pondría el sello de tiempo y guardaría una copia de dicho documento, este es el modelo más parecido a la vida real. Esta solución tiene un par de grandes problemas como puede ser la privacidad del documento que se pierde totalmente, tenemos que tener en cuenta que el servidor de timestamping puede estar en España, EEUU o en cualquier otro país y a su vez la base de datos para almacenar todos los documentos tendría que ser enorme, por lo que almacenar todos los documentos nos puede acarrear muchos problemas.

\item \textbf{Solución Arbitrada avanzada:} Esta solución es una evolución de la anterior, en ella el cambio que se hace es que el usuario que quiere que le sellen el documento manda el hash de dicho documento y la entidad sólo tendría que almacenar dicho hash junto con el sello de tiempo que se ha generado. Esta solución no tiene los inconvenientes de la anterior, ya que el tamaño de los documentos se reduciría a unos pocos bytes y la privacidad del documento no se ve comprometida. El problema que si persiste es que el usuario conozca a la entidad de certificación y puedan generar timestamp falsos, pero este problema depende de la confianza que queramos darle a ese servicio, supondremos que si es un servicio oficial y serio este problema no va a ocurrir, de todas formas existen otras soluciones que arreglan dicho problema.

\item \textbf{Solución Arbitrada avanzada y distribuida:} Esta forma consigue arreglar el problema de la anterior que se produzca un uso fraudulento del servidor de timestamping. La solución es usar varias entidades de timestamping, por lo que el usuario mandaría el hash a varias entidades de sellado y guardaría los resguardos que están firmados digitalmente de todas las entidades. Así si en una hay un problema tendría varias replicas de que la firma se realizó en ese momento en concreto.

\item \textbf{Solución mediante enlaces:} Esta solución es la más compleja y a su vez la que soluciona todos los problemas anteriores, además tiene la ventaja de que no tiene que usar multitud de entidades de certificación. Consiste en que cuando un usuario quiera sellar un documento, mande el hash del documento, la entidad añade el número de serie del documento anterior, el timestamp y lo firma digitalmente, por lo que el problema de que se introduzcan valores fraudulentos por mitad se anula, ya que cada recibo está enlazado con el anterior.
\end{itemize}

En nuestro caso hemos desarrollado un servidor de timestamping en su versión solución arbitrada avanzada.


\subsubsection{Explicación de la aplicación web.}
%TODO: falta explicar como funciona la BD...
En este capítulo vamos a explicar todas las partes que componen la aplicación web que hemos desarrollado para la implementación del servidor timestamp.

En la figura~\ref{fig:paquete_pfc} se puede ver las clases que forman el paquete \textit{pfc}.

\begin{figure}
  \centering
    \includegraphics[scale=0.6]{./GoogleAppEngine/imagenes/UML_pfc.png}
  \caption{Detalles del paquete pfc}
  \label{fig:paquete_pfc}
\end{figure}

A continuación vamos a explicar una por una las clases desarrolladas para el funcionamiento del servidor de timestamp.

\begin{itemize}

\item \textbf{Dao.java:} esta clase es la encargada de todos los accesos a la base de datos, desde la inserción, el borrado y el listado de las filas, hasta consultas que se necesiten. Se puede ver que las sentencias que son listado de columnas se realizan con una sentencia SQL, se puede ver un ejemplo es el siguiente trozo de código:

\begin{lstlisting}[style=Java]
EntityManager em = EMFService.get().createEntityManager();
Query q = em.createQuery("select t from RowTimerstamp t where t.num_sec = :num_sec");
q.setParameter("num_sec", id);
RowTimerstamp RowTimerstamps = (RowTimerstamp)q.getSingleResult();
\end{lstlisting}

Pero las consultas que implican inclusión o borrado de filas no se realizan mediante sentencias SQL convencionales, se realizan con métodos que proporciona la API, un ejemplo es el siguiente trozo de código:

\begin{lstlisting}[style=Java]
EntityManager em = EMFService.get().createEntityManager();
RowTimerstamp RowTimerstamp = new RowTimerstamp(num_sec, firma, fecha);
em.persist(RowTimerstamp);
em.close();
\end{lstlisting}

\item \textbf{RowTimerstamp.java:} en esta clase se diseña el formato de las filas de la base de datos, que como se ha explicado anteriormente no se crea con sentencias SQL, se usa un modelo de programación llamado JPA. Para dicho modelo hay que crear una clase que contenga como variables de clase, las columnas que formarán la tabla en la base de datos. Como podemos ver, mediante un mecanismo llamado anotaciones Java, se le indica si el campo es la clave primaría, si es auto incrementado y otras opciones que habría que indicar en la creación de la tabla. A continuación podemos ver un trozo de código con las variables que posteriormente serán las filas de la tabla que queremos crear.  

\begin{lstlisting}[style=Java]
@Id
@GeneratedValue(strategy = GenerationType.SEQUENCE)
private Long id;
private Long num_sec;
private String firma;
private Date fecha;
\end{lstlisting}

Se puede ver que el campo \lstinline{id} será la clave primaría, que se indica mediante la anotación \lstinline{@Id} y que será auto incremental, a su vez también podemos ver el resto de datos que se van a almacenar, el campo \lstinline{num_sec} que es el número de secuencia, ya que el campo \lstinline{id} lo usa la base de datos para organizarse internamente, el campo \lstinline{firma} que es hash firmado por el usuario, el campo \lstinline{fecha} como su nombre indica es la fecha en la que se subió el hash firmado. El resto de métodos que tiene esta clase son un constructor, getter para consultar los campos y setter para insertar valores.

\item \textbf{ServletCreateRow.java:} esta clase es un servet que se encarga de recibir todos los parámetros necesarios y añadirlos a la base de datos. Al recibir los parámetros mediante \lstinline{GET} tiene que implementar el método \lstinline{doGet}, casi todos los servlets implementados en el proyecto mandan los parámetros mediante \lstinline{GET}. La forma de recibir parámetros es la siguiente:

\begin{lstlisting}[style=Java]
String firma = req.getParameter("firma");
\end{lstlisting}

El resto de parámetros que se necesitan se generan en el servidor para que no puedan ser falseados, como es el número de secuencia y la fecha. Si la inserción se produce correctamente se devuelve una cadena que tiene el siguiente formato: ``ok;;num\_sec;;fecha", que será interpretado en la aplicación móvil y parseará dicha cadena para conseguir los valores que necesitemos.

%Quitado porque se pueden borrar datos que queramos con el dashboard
%\item \textbf{ServletDeleteAll.java:} es un servlet ``secreto" que se usa para borrar todas las filas del servidor, cosa que no se debería poder para no poder falsear los datos introducidos en el servidor de timestamp. Hay que llamarlo con un parámetro que es \textbf{borrar} con valor \textbf{5}.

\item \textbf{ServletSearchDate.java:} es un servlet que devuelve una cadena con la fecha de una fila que tiene el número de secuencia que se le pasa en el parámetro \lstinline{token}.

\item \textbf{ServletSearchRow.java:} es un servlet que devuelve una página web donde se puede observar en una única fila con toda la información almacenada correspondiente al número de secuencia que se le pasa mediante el parámetro \lstinline{id}. A continuación se puede ver un trozo de código que lo realiza:

\begin{lstlisting}[style=Java]
PrintWriter pw = resp.getWriter();
pw.print("<!DOCTYPE html>");
pw.print("<html><head><title>Lista Time Stamp</title><link rel=\"stylesheet\" type=\"text/css\" " + "href=\"css/main.css\"/> <meta charset=\"utf-8\"> </head>");
pw.print("<body><table><tr><th>ID</th><th>Num sec</th><th>Firma</th><th>Date</th></tr><tr> " + "<td>"+ row.getId() +"</td><td>"+ row.getNum_sec() +"</td><td>"+ row.getFirma() +"</td><td>" +	row.getFecha() + "</td></tr> </table></body>");
pw.flush();
\end{lstlisting}

Como se puede ver se crea una tabla en una web, con la etiqueta \lstinline{<TABLE>} y su fila se rellena dinámicamente dependiendo del número de secuencia que se le pase como parámetro.

\item \textbf{ServletSearchSign.java:} Es un servlet que devuelve una cadena con la firma que corresponde al número de secuencia que se pasa por el parámetro \lstinline{token}.

\end{itemize}

La mayoría de estos servlets son usados por la otra aplicación web o por la aplicación móvil para realizar comprobaciones o mostrar información.

%TODO: falta los jsp

\subsection{Servidor de registro de firmas.}

El servidor de registro de firmas que hemos desarrollado es una aplicación web en la que los usuarios pueden consultar las firmas realizadas, las firmas de las que es destino, gestionar sus certificados de clave pública, verificar una firma realizada por otro usuario, exportar una cadena con la que cualquier usuario pueda consultar si la firma que has realizado es válida, que se usará en comprobaciones en caso de que ocurra algún problema y generar códigos QR para que que otros usuarios puedan firmarlo.

El sistema de gestión de usuarios lo proporciona Google y para entrar en la aplicación web hay que poseer una cuenta de Google Account, si no se está autentificado se produce una redirección a la página de autentificación, que la podemos ver en la figura~\ref{fig:logueoRepoGeneral}. La parte de la seguridad de los usuarios, autenticación y mantenimiento de las base de datos lo realiza Google.

\begin{figure}[h]
  \centering
    \includegraphics[scale=0.5]{./GoogleAppEngine/imagenes/login_repositorio_general.png}
  \caption{Login en Repositorio General}
  \label{fig:logueoRepoGeneral}
\end{figure}

La aplicación web se puede ver en la figura~\ref{fig:repositorio_general}, podemos ver que tiene varias pestañas, que se explicarán posteriormente, pero principalmente cada una de ellas se encarga de realizar una de las funciones que hemos comentado antes.

\begin{figure}[h]
  \centering
    \includegraphics[scale=0.4]{./GoogleAppEngine/imagenes/repositorio_general.png}
  \caption{Repositorio General}
  \label{fig:repositorio_general}
\end{figure}

\subsubsection{Explicación de la aplicación web.}

%TODO: falta explicar como funciona la BD...
En la figura~\ref{fig:clasesReposotorioGeneral} se puede ver las clases que forman el paquete \textit{pfc} de la aplicación web repositorio general. A continuación vamos a explicar una por una las clases desarrolladas.

\begin{figure}[h]
  \centering
    \includegraphics[scale=0.4]{./GoogleAppEngine/imagenes/UML_repositorio.png}
  \caption{Detalles de las clases Repositorio General.}
  \label{fig:clasesReposotorioGeneral}
\end{figure}

\begin{itemize}

\item \textbf{Dao.java:} al igual que en el servidor de timestamp, esta clase es la encargada de hacer todas las operaciones relacionadas con la base de datos.

\item \textbf{DaoUserCert.java:} como hemos implementado dos bases de datos, una para guardar las firmas y otra para guardar los certificados de clave pública que se necesitan, esta clase es la encargada de realizar todas las operaciones en la base de datos donde se guardan los certificados. 

\item \textbf{RowRepositorioGeneral.java:} esta es la clase con la que se crea la tabla en la que se almacenan las firmas de los usuario, tiene los siguientes campos:  

\begin{lstlisting}[style=Java]
@Id
@GeneratedValue(strategy = GenerationType.SEQUENCE) //	 GenerationType.IDENTITY
private Long id;
private Long num_sec;
private String url_firma;
private String texto_claro;
private Long token_tiempo;
private String usuario;
private Boolean confirmado;
private String destino;
private BlobKey blobKey;
private Date fecha;
\end{lstlisting}

Tiene una clave primaría \lstinline{id} que es usada por la base de datos internamente para el almacenado de la información, \lstinline{num_sec} es el número de secuencia dentro la tabla, que va incrementándose automáticamente, \lstinline{url_firma} es la dirección en la que se puede consultar la firma del texto en claro que está en el campo \lstinline{texto_claro}, se guarda el \lstinline{token_tiempo} que es el \lstinline{num_sec} de en la aplicación web servidor de timestamp. La columna \lstinline{usuario} almacena el usuario que ha realizado la firma, y en la columna \lstinline{destino} se guarda a quien va dirigida la firma. En la columna \lstinline{blobkey} se guarda la referencia al certificado de clave pública que estaba en activo cuando se subió la firma a la aplicación web, la columna \lstinline{confirmado} puede valer \lstinline{true} or \lstinline{false} e indica si al subir la firma se pudo verificar, en \lstinline{fecha} está la fecha en la que se almacenó.

\item \textbf{RowUserCert.java:} Esta clase es la encargada de crear la tabla que usamos para guardar los archivos con la clave pública. Los campos que usaremos para almacenarlos serán los que se pueden ver en el siguiente trozo de código.

\begin{lstlisting}[style=Java]
@Id
@GeneratedValue(strategy = GenerationType.SEQUENCE)
private Long id;
private String usuario;
private Date fecha;
private BlobKey certificado;
\end{lstlisting}

Como podemos ver el campo \lstinline{id} será la clave primaria y como hemos explicado será usado por la base de datos internamente para autogestión de las filas, el campo \lstinline{usuario} guardará una cadena con el email de la persona que ha subido ese archivo, el campo \lstinline{fecha} es la fecha en la que se subió el archivo, \lstinline{certificado} es un campo del tipo \lstinline{BlobKey} que es como la ruta al archivo de certificado.

\end{itemize}

Acto seguido vamos a explicar los diferentes servlets que hemos desarrollado para la aplicación web.

\begin{itemize}

\item \textbf{ServletCreateCertificate.java:} Este servlet es el encargado de añadir a la base de datos el certificado de clave pública. Es usado en la pestaña de certificados de la aplicación web y es llamado cuando se pulsa el botón subir certificado. Se puede observar dicho botón en la figura~\ref{fig:pestanhaCertificados}.

\end{itemize}

\begin{figure}[h]
  \centering
    \includegraphics[scale=0.5]{./GoogleAppEngine/imagenes/certificadosRepositorioGeneral.png}
  \caption{Pantallazo de la pestaña certificados.}
  \label{fig:pestanhaCertificados}
\end{figure}

\begin{itemize}

\item \textbf{ServletCreateRowRepositorio.java:} Este servlet está mapeado en la dirección: \url{https://repositoriorecibos.appspot.com/add} y recibe los siguientes parámetros: \url{texto}, \url{url\_firma}, \url{token}, \url{destino} y \url{fecha}. Es el encargado de añadir una fila a la base de datos por cada llamada a dicha dirección, a esta dirección no hay forma de acceder desde la aplicación web. A su vez antes de introducir la fila comprueba que la firma se puede validar y se marca como \lstinline{true} o \lstinline{false} la columna verificado que posteriormente en el archivo \textit{RepositorioGeneralApplication.jsp} se cambiará por una imagen para hacer la verificación más visual. Si hemos podido insertar la fila, el servlet devuelve la cadena ``OK", si no se devuelven varias cadenas con los fallos que han ocurrido.

%\item \textbf{ServletDeleteAll.java:} Servlet ``secreto" que borra todas las filas de firmas almacenadas, hay que llamarlo con un parámetro que es \textit{borrar} con valor \textit{7}

\item \textbf{ServletExport.java:} Este servlet es el encargado de exportar una de las filas para que otra persona pueda comprobar si la firma es válida. Este servlet es llamado cuando se pulsa el botón exportar de la pestaña principal de la aplicación web. Se puede observar en la figura~\ref{fig:botonExportar}

\end{itemize}

\begin{figure}[h]
  \centering
    \includegraphics[scale=0.4]{./GoogleAppEngine/imagenes/botonExportar.png}
  \caption{Detalle del botón exportar.}
  \label{fig:botonExportar}
\end{figure}

\begin{itemize}

El servlet recibe los siguiente parámetros:

\begin{lstlisting}[style=Java]
String mensaje = checkNull(req.getParameter("mensaje"));
String url_firma = checkNull(req.getParameter("token"));
String id_blob = checkNull(req.getParameter("id_blob"));
String user = checkNull(req.getParameter("usuario"));
\end{lstlisting}

Una vez se tienen esos parámetros creamos una cadena de texto en la que unimos los siguiente campos y cada parámetro va separado por el separador: ``;/:".

\begin{lstlisting}[style=Java]
String cadACodificar = mensaje + ";/:" + url_firma + ";/:" + id_blob + ";/:" + user;
\end{lstlisting}

Acto seguido codificamos la cadena con \lstinline{Base64}\footnote{Para más información puede consultar: \url{http://en.wikipedia.org/wiki/Base64}}, que es una forma simple de codificar los caracteres para que no viajen en texto claro.

\begin{lstlisting}[style=Java]
String cadCodificada = Base64.encode(cadACodificar.getBytes("UTF-8"));
\end{lstlisting}

También añadimos unos limitadores para que cuando tengamos que decodificar ese mensaje podamos saber donde empiezan y donde termina la exportación.
\begin{lstlisting}[style=Java]
pw.println("BEGIN EXPORT");
pw.println("--------------------------");
pw.println(cadCodificada);
pw.println("--------------------------");
pw.println("END EXPORT");
\end{lstlisting}

\item \textbf{ServletListRow.java:} Este servlet es el encargado en devolver todas las filas de la tabla que pertenecen a un usuario. La forma de hacerlo es la siguiente, primero se identifica el usuario con el que se ha autentificado de esta forma:

\begin{lstlisting}[style=Java]
UserService userService = UserServiceFactory.getUserService();
User user = userService.getCurrentUser();
\end{lstlisting}

Una vez se consigue el usuario se llama a la función \lstinline{public List<RowRepositorioGeneral> getRowRepositorioGeneralList(String userId, Long num_sec)} de la clase \lstinline{Dao.java}, esta última función nos devuelve una lista con todas las filas. Al llamar al servlet le pasaremos el último número de secuencia que tenemos guardado en el teléfono móvil, para así agilizar las transferencias de datos, de esta forma solo nos devolverá las filas nuevas. La forma de devolvernos las filas será mediante una estructura llamada \lstinline{JSONArray}, que es un objeto que dentro contiene varios objetos \lstinline{JSON}\footnote{Para saber que es un objeto \lstinline{JSON} pueden consultar los siguientes enlaces: \url{http://www.json.org/} o \url{http://en.wikipedia.org/wiki/JSON}}.

La creación de los objetos \lstinline{JSON} la realizamos de la siguiente forma:

\begin{lstlisting}[style=Java]
JSONObject jsonObject = new JSONObject();

jsonObject.put("num_sec", rowRepositorioGeneral.getNum_sec().toString());
jsonObject.put("texto", rowRepositorioGeneral.getTexto_claro());
jsonObject.put("url_firma", rowRepositorioGeneral.getUrl_firma());
jsonObject.put("token_tiempo", rowRepositorioGeneral.getToken_tiempo().toString());
jsonObject.put("usuario",rowRepositorioGeneral.getUsuario());
jsonObject.put("fecha", rowRepositorioGeneral.getFecha().toString());
jsonObject.put("verificado", rowRepositorioGeneral.getConfirmado().toString());
jsonObject.put("destino", rowRepositorioGeneral.getDestino());
\end{lstlisting}

Ese objeto \lstinline{JSON} se añade un objeto \lstinline{JSONArray}, que es el que devolveremos como respuesta final de la ejecución de nuestro servlet y que espera la aplicación Android.

\item \textbf{ServletVerify.java:} Este servlet es el utilizado en la pestaña verificar de nuestra aplicación web, como se puede observar en la figura~\ref{fig:pestanhaVerificar}

\end{itemize}

\begin{figure}[h]
  \centering
    \includegraphics[scale=0.5]{./GoogleAppEngine/imagenes/pestanhaVerificar.png}
  \caption{Detalle de la pestaña verificación.}
  \label{fig:pestanhaVerificar}
\end{figure}

\begin{itemize}

\item Podemos observar que hay un cuadro de texto para introducir la cadena que devuelve el botón exportar. Cualquier usuario puede verificar si una firma es correcta o no. En este servlet se hace el proceso contrario que hicimos en exportar, quitamos los indicadores de inicio y final de exportación, desencriptamos la cadena en \lstinline{Base64} y hacemos varias comprobaciones. Comprobamos que en la fecha en la que se firmó el certificado era válido y que no habíamos revocado dicho certificado, también se comprueba que no fuera reemplazado por otro certificado antes de su expiración, ya que entonces la firma no sería válida. También comprobamos la integridad del mensaje, que la cadena no esté mal formada y que siga el formato que hemos obligado anteriormente.

\end{itemize}

Los archivos JSP que hemos creado en su mayoría solo rellenan tablas dinámicamente haciendo llamadas a funciones de la clase \lstinline{Dao.java}. Solo hay uno que no realiza esas funciones que es el siguiente:

\begin{itemize}

\item \textbf{GenerarQR.jsp:} Este archivo JSP es el que se muestra en la pestaña \textit{Generar QR}, se puede ver en la figura~\ref{fig:pestanhaQR}. Su función es generar un código QR para que pueda ser leído por la aplicación del móvil. Hay que rellenar los campos de destino y el texto que queremos que firme dicha persona. Al darle a \textit{Generar código QR} se hace una llamada a la API Google Chart y se genera un código QR que contiene dichas cadenas y se muestra en la parte de la derecha, como se puede ver en la figura~\ref{fig:codigoQR}.

\end{itemize}

\begin{figure}[h]
  \centering
    \includegraphics[scale=0.5]{./GoogleAppEngine/imagenes/pestanhaQR.png}
  \caption{Pestaña para generar el código QR.}
  \label{fig:pestanhaQR}
\end{figure}

\begin{figure}[h]
  \centering
    \includegraphics[scale=0.4]{./GoogleAppEngine/imagenes/codigoQR.png}
  \caption{Pestaña con el código QR generado.}
  \label{fig:codigoQR}
\end{figure}


	
	%Conclusiones y trabajo futuro
	\input{./ConclusionesYTrabajoFuturo/conclusionesYTrabajoFuturo.tex}
		
	\appendix
	%apendice con la configuración de eclipse tanto para Android como para Google App Engine
	\input{./AnexoConfiguracionEclipse/configuracionEclipse.tex}
	%apendice para la creación de los certificados usados con XCA
	\chapter[Creación certificados digitales.]{Creación de los certificados digitales para la aplicación.}\label{cap:anexoB}
\markboth{ANEXO \ref{cap:anexoB}. CREACIÓN CERTIFICADOS DIGITALES.}{}

En este anexo vamos a explicar como hemos realizado la generación de los certificados digitales usados tanto por la aplicación del móvil, como las aplicación web para la verificación de las firmas.

El programa elegido fue XCA (\url{http://xca.sourceforge.net/}) que es un gestor de certificados y claves, implementado por Christian Hohnstädt. Se eligió dicho programa por ser multiplataforma, por lo que se podrían generar los certificados en cualquier sistema.

XCA está e los repositorios de Ubuntu, por lo que para instalarlo solo hay que proceder a su instalación normal. 

\begin{lstlisting}[style=consola]
sudo apt-get install xca
\end{lstlisting}

Para instalarlo en otro sistema se puede descargar desde: \url{http://sourceforge.net/projects/xca/}.

Una vez instalado, XCA proporciona una interfaz gráfica con la que realizar todo el proceso. Lo primero es generar la CA. Para ello pulsamos en File -> New Database. Elegimos donde queremos guardar toda la información. En el siguiente paso nos pedirá que establezcamos una contraseña para la CA (figura~\ref{fig:password}).

\begin{figure}
  \centering
    \includegraphics[scale=0.6]{./AnexoCreacionCertificado/imagenes/password.png}
  \caption{Password para la CA.}
  \label{fig:password}
\end{figure}

El siguiente paso sería generar el certificado que vamos a usar, para ello vamos a la pestaña Certificates y pulsamos New Certificate, en la nueva pantalla que se abre podemos rellenar los valores que necesitemos en la pestaña Subject y debemos generar una nueva clave privada pulsando en el botón de Generate a new key (figura\ref{fig:clavePrivada}). Una vez generada podemos definir los usos para los que queremos que se pueda usar el certificado en la pestaña Key usage.

\begin{figure}
  \centering
    \includegraphics[scale=0.6]{./AnexoCreacionCertificado/imagenes/clavePrivada.png}
  \caption{Creación de la clave privada.}
  \label{fig:clavePrivada}
\end{figure}

Una vez generado el certificado lo exportamos en formato PKCS\#12 y lo almacenamos en la ruta que queramos (figura~\ref{fig:exportacion}).

\begin{figure}
  \centering
    \includegraphics[scale=0.6]{./AnexoCreacionCertificado/imagenes/exportacion.png}
  \caption{Exportación del certificado.}
  \label{fig:exportacion}
\end{figure}

Una vez creado el archivo *.p12 debemos de meterlo en el móvil y subirlo a la aplicación web en la pestaña Añadir certificados. En el movil hay que colocarlo en el directorio que se configuró la primera vez que se ejecutó la aplicación.

En caso de que el proyecto se llevase a cabo en la realidad este proceso no podría hacerlo el usuario y debería haber un control exhaustivo de dichos certificados con listas de revocación, entidades que firmasen los certificados, etc.









	%apendice con el contenido del CD
	\input{./AnexoContenidoCD/contenidoCD.tex}
	
	\backmatter
	\listoffigures
	%bibliografía
	\input{./Bibliografia/bibliografia.tex}
		
\end{document}
